\documentclass{jsarticle}

\usepackage{amsmath, amssymb}
\usepackage{type1cm}
\usepackage{mathtools}
\usepackage{bm}
\usepackage{amsthm}
\usepackage{color}
\usepackage[dvipdfmx]{graphicx}
\usepackage{listings,jvlisting}

\newtheorem{thm}{Theorem}[section]
\newtheorem{cor}{Corollary}[thm]
\newtheorem{lem}[thm]{Lemma}
\newtheorem{prop}[thm]{Proposition}
\theoremstyle{definition}
\newtheorem{dfn}{Definition}[section]
\newtheorem{ex}{Example}[section]


\lstset{
  basicstyle={\ttfamily},
  identifierstyle={\small},
  commentstyle={\smallitshape},
  keywordstyle={\small\bfseries},
  ndkeywordstyle={\small},
  stringstyle={\small\ttfamily},
  frame={tb},
  breaklines=true,
  columns=[l]{fullflexible},
  numbers=left,
  xrightmargin=0zw,
  xleftmargin=3zw,
  numberstyle={\scriptsize},
  stepnumber=1,
  numbersep=1zw,
  lineskip=-0.5ex
}


\begin{document}


\title{Convex}
\maketitle

\section{Affine sets}


$\bm{x} = (\xi_1,  ... , \xi_n) \subset \mathbb{R}^n$

$\langle x, x^*\rangle = \sum \limits_{i=1}^{n} \xi_i \cdot
\xi_i^* $

$A: m \times n$ real matrix and linear operator from
$\mathbb{R}^n$ to $\mathbb{R}^m$

\begin{math}
f: \mathbb{R}^n  \to \mathbb{R}^m   \colon x \mapsto Ax
\end{math}

\begin{math}
A^* = \overline{A}^\top
\end{math}
 : adjoint matrix (随伴行列、エルミート転置行列)
 
 (the transpose matrix and the corresponding adjoint linear transformation from
$\mathbb{R}^n$ to $\mathbb{R}^m$)

\begin{math}
\langle Ax, y^*\rangle =  \langle x, A^*y^*\rangle
\end{math}
\\



\begin{dfn}[affine set]
A subset $M$ of $\mathbb{R}^n $ is called \textit{affine set }
if
\[
\theta x + (1 - \theta )y \in M, \qquad \forall x, y \in M,
\forall \theta \in \mathbb{R}
\]
\end{dfn}


The empty set $\emptyset$ and the space $\mathbb{R}^n$ itself
are affine sets.
Also coverd by the definition is the case where $M$ consists of
a solitary point.(一点集合もAffine set)
An affine set has to contain, along with any two different
points, the entire line through those points.\\


\begin{thm} 
The subspaces of $\mathbb{R}^n$ are the affine sets which
contain the origin. 
\end{thm}



\begin{dfn}
For $M \subset \mathbb{R}^n$ and $a \in \mathbb{N}$, the
translate(平行移動) of $M$ by $a$ is defined to be the set
\[
M + a = \{x + a \,| \,   x \in M \}
\]
\end{dfn}


An affine set $M$ is said to be parallel to an affine set $L$ if $L = M + a$ for some $a$.
\\

\begin{thm}
Each non-empty affine set $M$ is parallel to a unique subspace
$L$. This $L$ is given by
\[
L = C - C = \{ x - y \,| \, x, y  \in C\} 
\]
\end{thm}

The dimension of a non-empty affine set is defined as the
dimension of the subspace parallel to it. (The dimension of
$\emptyset$ is $-1$ by convention.)

空でないアフィン集合の次元は、それに平行な部分空間の次元と定義される。($\emptyset$の次元は慣習的に$-1$)

Affine sets of dimension 0, 1 and 2 are called points, lines andplanes, respectively.\\



\begin{dfn}[Affine combination]
Let $x_1,...,x_n \in \mathbb{R}^n $, and $\theta_1,..., \theta_n\in \mathbb{R} (n \in \mathbb{N})$.

$\sum \limits_{i=1}^n \theta_i x_i $ is an affine
combination of $x_1, ..., x_n$ if $\sum \limits_{i=1}^n \theta_i= 1$ 
\end{dfn}



\begin{thm}[Hyperplane]
Given $\beta \in \mathbb{R}$ and a non-zero $\bm{b} \in
\mathbb{R}^n$ , the set
\[
H = \{ x \in \mathbb{R}^n\,| \,\langle \bm{x}, \bm{b} \rangle =
\bm{\beta } \}
\]
is a hyperplane in $\mathbb{R}^n$.
\end{thm}


The vector $\bm{b}$ is called a normal to the hyperplane $H$.\\
Every hyper plane has ''two sides,'' loke one's picture of alne
in $\mathbb{R}^2$ or a plane in $\mathbb{R}^3$.\\


\begin{thm} 
Given $\bm{b} \in \mathbb{R}^m$ and an $m \times n$ real matrix
$\bm{B}$, the set
\[
M = \{ x \in \mathbb{R}^n \,| \, \bm{B} \bm{x} = \bm{b}  \}
\]
is an affine set in $\mathbb{R}^n$. 
Moreover, every affine set may be represented in this way.
\end{thm}



Let $\bm{b_i}$ is the $i$th row of $\bm{B}$, $\beta_i$ is the
$i$-th component of $\bm{b}$, and
\[
H_i = \{ \bm{x} \, | \, \langle \bm{x}, \bm{b_i} \rangle =
\beta_i, \} .
\]
Then, 
\[
M = \{ \bm{x} \, | \, \langle \bm{x}, \bm{b_i} \rangle =
\beta_i, i = 1, ... , m\} = \bigcap_{i=1}^{m}{H_i} .
\]

Each $H_i$ is a hyperplane ($\bm{b_i} \neq \bm{0}$).\\


The affine set $M$ in Theorem can be expressed in terms of the
vebtors $b'_1, ..., b'_n$ which form the columns of $\bm{B}$ by
\[
M = \{ \bm{x} = ( \xi_1, ..., \xi_n) \, | \, \sum
\limits_{i=1}^n \xi_i b'_i = \bm{b} \}
\]
\\


任意の集合$S \in \subset
\mathbb{R}^n$に対し、$S$を含む(唯一で)最小のアフィン集合が存在する。これをアフィン包($affine
hull$)という。\\



\begin{dfn}[affine hull]
A subset $C$ of $\mathbb{R}^n $ is callded \textit{affine hull }if
\[
\text{aff} \, S = \{ \sum \limits_{i=1}^n \theta_i x_i \, | \,
x_i \in S (i = 1,...,n), \sum \limits_{i=1}^n \theta_i = 1\} .
\]
\end{dfn}






\begin{dfn}
A set of $k+1$points $b_0, b_1, \dots , b_k$ is  said \textit{affinely independent} if the set 

\begin{align*}
\text{aff}  \{b_0,  b_1,  \dots,  b_k\} &= \text{aff} \{b_0 - b_0,  b_1 - b_0,  \dots,  b_k - b_0\}  + b_0\\
                                           &=  \text{aff} \{0,  b_1 - b_0,  \dots,  b_k - b_0\} + b_0\\
                                           &= L + b_0  \quad \because L = \text{aff} \{0,  b_1 - b_0,  \dots,  b_k - b_0\}
\end{align*}
 is $k$-dimensional.

\end{dfn}


定理1.1より、Lは$b_1 - b_0,  \dots,  b_k - b_0$を含む最小の部分集合と同じになる。
Lの次元が$k$になる必要十分条件は$b_1 - b_0,  \dots,  b_k - b_0$が、線形独立になることである。
従って、$b_0,  b_1,  \dots,  b_k$がアフィン独立である必要十分条件は$b_1 - b_0,  \dots,  b_k - b_0$が線形独立であることである。\\

よって$x_0, x_1, \dots, x_k$がアフィン独立であれば、 $x \in$ aff$\{b_0,  b_1,  \dots,  b_k\}$は次のように表せる。
\[
x = \sum \limits_{i=0}^{k} \theta_i x_i, \quad \sum \limits_{i=0}^{k} \theta_i = 1. 
\]\\


\begin{dfn}[affine transform]
$T: x \to Tx$ from $\mathbb{R}^n$ to $\mathbb{R}^m$ is called
\textit{affine transformation} if
\[
T((1 - \lambda)x + \lambda y) = (1 - \lambda)Tx + \lambda Ty ,
\qquad \forall x, y \in \mathbb{R}^n, \forall \lambda \in
\mathbb{R}
\]
\end{dfn}




\section{Convex Sets and Cones}
 
\begin{dfn}[convex set]
A subset $C$ of $\mathbb{R}^n $ is called \textit{convex} if
\[
\theta x + (1 - \theta )y \in C, \qquad \forall x, y \in C,
\forall \theta \in [0, 1]
\]
\end{dfn}

All affine sets are convex.

\begin{dfn}[convex combination]

An affine combination with a coefficient $\theta_i \in [0,1]$ iscalled a \textit{convex combination}, i.e.

let $x_1,...,x_n \in \mathbb{R}^n $, and $\theta_1,..., \theta_n\in \mathbb{R} (n \in \mathbb{N})$.
$\sum \limits_{i=1}^n \theta_i \cdot x_i $ is an \textit{combex
combination} of $x_1, ..., x_n$ if
\[
\sum \limits_{i=1}^n \theta_i x_i \quad (\theta_i \in
[0,1] (i = 1,...,n)), \qquad \sum \limits_{i=1}^n \theta_i = 1,
\]
\end{dfn}


\begin{dfn}[Half-spaces(半空間)]

For any non-zero $\bm{b} \in \mathbb{R}^n$ and any $\beta$ in
$\mathbb{R}$, the sets
\[
\{ \bm{x} \, | \, \langle \bm{x}, \bm{b}\rangle \le \beta \},
\qquad \{ \bm{x} \, | \, \langle \bm{x}, \bm{b}\rangle \ge \beta\}
\]
are called \textit{closed half-spaces}.  The sets
\[
\{ \bm{x} \, | \, \langle \bm{x}, \bm{b}\rangle < \beta \},
\qquad \{ \bm{x} \, | \, \langle \bm{x}, \bm{b}\rangle > \beta
\}
\]
are called \textit{open half-spaces}. All four sets are plainly
non-empty and convex.
\end{dfn}


\begin{thm}
The intersection of an arbitrary collection of convex sets is convex.
\end{thm}


A set which can be expressed as the inter section of finitely
many closed half spaces of $\mathbb{R}^n$ is called a
\textit{polyhedral convex set}. (凸多面体)\\



$H = \{ x \in \mathbb{R}^n\,| \,\langle \bm{x}, \bm{b} \rangle =\bm{\beta } \} $,
$a \in H$とするとき、半空間($\{ x \in \mathbb{R}^n\,| \,\langle \bm{x},
\bm{b} \rangle \le \bm{\beta} \} $)は
\[
C = \{ x \in \mathbb{R}^n\,| \,\langle \bm{x - a}, \bm{b}
\rangle \le \bm{0} \}
\]
と表せる。$\bm{b}$は点$a$における法線ベクトルである。

これは点$a$から半空間($\{ x \in \mathbb{R}^n\,| \,\langle \bm{x}, \bm{b}
\rangle \le \bm{\beta} \}
$)上の点$a$に向かうベクトルが超平面の法線ベクトル$b$と鈍角をなすことを意味している。

よって、半空間($\{ x \in \mathbb{R}^n\,| \,\langle \bm{x}, \bm{b}
\rangle \le \bm{\beta} \} $)は$b$の反対側に位置する。

一方、半空間($\{ x \in \mathbb{R}^n\,| \,\langle \bm{x}, \bm{b}
\rangle \ge \bm{\beta} \} $)は$b$の同じ側に位置する。\\



\begin{cor}
Let $b_i \in \mathbb{R}^n$ and  $\beta_i \in \mathbb{R}$ for $i \in I$, where $I$ is an arbitrary index set, and consider the set 
\[
C = \{ x \in \mathbb{R}^n\,| \,\langle \bm{x}, \bm{b_i}
\rangle \le \bm{\beta_i}, \forall i \in I \}.
\]

It is crealy convex.\\
\end{cor}

\begin{thm}
A set $C \in \mathbb{R}^n$ is convex if and only if it contains all the convex combinations of its elements.

($C \in \mathbb{R}^n$が凸集合$\, \Longleftrightarrow \, $Cの元からなる全ての凸結合をC自身が含む)\\ 
\end{thm}


\begin{thm}[convex hull]
The convex hull of a given set $X \subset \mathbb{R}^n $ may be defined as
the set satisfying any one (and hence all) of the following
equivalence conditions.
\begin{enumerate}
\item The (unique) minimal convex set containing $X$. ($X$
を含む(唯一の)最小の凸集合)
\item The intersection of all convex sets containing $X$. 
\[
conv X = \cap \{ C : C \text{ is convex set} ,  X \subset C \}
\]
\item The set of all the convex combinations of points in $X$.
($X$ に属する点から得られる凸結合全体の成す集合)
\[
conv X = \{ \sum \limits_{i=1}^n \theta_i x_i \in \mathbb{R}^n\, | \, \exists m \in \mathbb{N}, \, \exists x_1, ... , x_m \in X,    \, \exists \theta_1, ..., \theta_m \in [0, 1],  \, \sum \limits_{i=1}^n \theta_i = 1 \}
\]

\item (Carath$\acute{e}$odory's Theorem)The union of all simplices with vertices in $X$. ($X$
に属する点を頂点とする単体全ての合併)

\[
conv X = {\{ \sum \limits_{i=1}^{n+1} \theta_i x_i \quad |\quad n \in \mathbb{N}, x_i \in X, \theta_i > 0, \sum
\limits_{i=1}^{n+1} \theta_i = 1} \}
\]
\end{enumerate}
\end{thm}



A set which is a convex hull of a finite number of points is called a polytope(超多面体).\\

\begin{dfn}
A subset K of $\mathbb{R}^n$ is called \textit{cone} if it is closed under positive scalar multiplication, i.e. $\lambda x \in K$ for $x \in K$ and $  \lambda > 0$.  
\end{dfn}


\begin{dfn}
A subset K of $\mathbb{R}^n$ is called \textit{convex cone} if it is closed under positive scalar multiplication, i.e. $\lambda x + (1 - \lambda) y \in K$ for $x, y \in K$ and $  \lambda > 0$.  
\end{dfn}

If K contains the origin, we call it a convex cone.

A convex cone should not necessarily be considered ``pointed''.
A subspace of $\mathbb{R}^n$ is a particularly convex cones.
The same is true is the open and closed half-space corresponding to a hyperplane through the origin.

Non-negative \textit{orthant}  of $\mathbb{R}^n$ is convex cone.
\[
\mathbb{R}^n_{+} = \{x = (\xi_1,  \dots, \xi_n) \in \mathbb{R}^n \, | \, \xi_i \ge 0 \, \text{for}\,  i = 1,  \dots,  n\} = \{x \in \mathbb{R}^n \, | \, x \ge 0\}
\]
\[
\mathbb{R}^n_{> 0} = \{x = (\xi_1,  \dots, \xi_n) \in \mathbb{R}^n \, | \, \xi_i > 0 \, \text{for}\,  i = 1,  \dots,  n\} = \{x \in \mathbb{R}^n \, | \, x > 0\}
\]


\begin{thm}
The intersection of an arbitrary collection of convex cones is a convex cone.
\end{thm}

\begin{cor}
Let  $b_i \in \mathbb{R}^n$ for $i \in I$, where I is an arbitrary index set.  Then
\[
K = \{x \in \mathbb{R}^n \, | \, \langle \bm{x}, \bm{b_i} \rangle \le 0, i \in I \}
\]
is a convex cone.

\end{cor}




\section{Convex Functions}

\textcolor{red}{This note is excluded the value $f(x) = - \infty$.}

\begin{dfn}
A function $f:C \to \mathbb{R}$,  where $C$ is a convex set.  Then $f$ is $convex$ on $C$ if and only if
\[
f((1 - \lambda)x + \lambda y) \le (1 - \lambda) f(x) + \lambda f(y), \quad \forall x, y \in C, \quad \lambda \in [0, 1].
\] 
\end{dfn}

Let
$f:$ convex function,  $f \in C \ne \emptyset$.
Extending $f$ to $\mathbb{R}^n$ by setting $f(x) = +\infty$ for $x \notin C \subset \mathbb{R}^n$.
Then the above definition is equivalent to the following definition.

\begin{dfn}
A function $f:\mathbb{R}^n \to \mathbb{R} \cup \{+ \infty\}$,  not identically $+ \infty$,  is convex if and only if
\[
f((1 - \lambda)\bm{x} + \lambda \bm{y}) \le (1 - \lambda) f(\bm{x}) + \lambda f(\bm{y}), \quad \forall \bm{x}, \bm{y} \in \mathbb{R}^n, \quad \lambda \in [0, 1].
\] 
\end{dfn}


$f: \mathbb{R}^n \to \mathbb{R} :$ $concave$  $\Leftrightarrow$ $- f$ is convex.

\begin{dfn}
$f: \mathbb{R}^n \to \mathbb{R} \cup \{ +\infty\},  C \subset \mathbb{R}^n $. We define epigraph of $f$ as
\[
\text{epi} f := \{(\bm{x},  y) \, | \, \bm{x} \in C,  y \in \mathbb{R},  y \ge f(\bm{x}) \}.
\]
Note that $f(x) = +\infty$ for $x \notin C$.

In othe words, 

$f: \mathbb{R}^n \to \mathbb{R} \cup \{ +\infty\}$, not identically equal to $+ \infty$ .  We define epigraph of $f$ is the non-empty set 
\[
\text{epi} f := \{(\bm{x},  y) \, | \, \bm{x} \in \mathbb{R}^n,  y \in \mathbb{R},  y \ge f(\bm{x}) \}.
\]

\end{dfn}


%We define $f$ to be convex function on $C$ if epi$f  (\subset \mathbb{R}^{n+1})$ is convex set.

\begin{dfn}
The \textit{effective domain} of function f on C is the set
\[
\text{dom} f = \{\bm{x} \in \mathbb{R}^n \, | \,  \exists y,  (\bm{x}, y) \in \text{epi} f \} = \{ \bm{x} \in \mathbb{R}^n \, | \, f(\bm{x}) < + \infty \}.
\]
\end{dfn}


\begin{dfn}
A convex function $f$ is said to be $proper$ if 
$f(x) < +\infty$ at least one $x \in C$ and $f(x) > - \infty$ for all $x \in C$.

In other words,  $f$ is $proper$ if its epigraph is non-empty and does not contain ``vertical lines''.
\end{dfn}

Let $f: $ convex function, $C \subset \mathbb{R}^n$ is non-empty convex set.

$f \in C$ is proper
$\Leftrightarrow$ 
 
dom $f = C$ where $f \in C$ is finite. \\
 
 
 $f \in \mathbb{R}^n$ is proper
$\Leftrightarrow$ 

 $f \in \mathbb{R}^n$ is convex function and $f \in C$ is finite. $ \Rightarrow$ 
 $f(x) = 
\begin{cases}
f(x) & (x \in C)\\
+ \infty & (x \notin C)
\end{cases}
$\\

\textcolor{red}{In this note,  we excluded the value $f(x) = - \infty$, so from the definition 3.2,  every convex function $f$ is ``proper''.}


\begin{thm}
$f: \mathbb{R}^n \to \mathbb{R} \cup \{+ \infty \}$.

$f$ is a convex funciton if and only if  \textup{epi} $f$ is convex set. 
\end{thm}

\begin{thm}
Let $f: \mathbb{R}^n \to (-\infty, + \infty], \lambda_i \ge 0$ and $\sum \limits_{i=0}^{k} \lambda_i = 1$. Then f is convex if and only if 
\[
f(\sum \limits_{i=0}^{k} \lambda_i x_i) \le \sum \limits_{i=0}^{k} \lambda_i f(x_i).
\]
\end{thm}


Neighborhood of $x^{\prime}$ is defined by
\[
B(x^{\prime}, \delta) := \{\bar{x} \, | \, d(x^{\prime},  x) \le \delta \}.
\]
\[
V \in \mathcal{N}(x^{\prime}) := \text{the collection of all neiborhoods of $x^{\prime}$}.
\]

\begin{dfn}(lower limits)
    Let $f: \mathbb{R}^n \to \mathbb{R} \cup \{+ \infty\}$ and let $x^{\prime}$ is a limit point of $f$. Then the lower limit of function $f$ is defined by 
    \begin{align*}
        \liminf_{x \to x^{\prime}}{f(x)} &= \lim_{\delta \, \searrow \, 0} {[\inf_{x \in B(x^{\prime}, \delta)}{f(x)}]} \\
                                          &= \sup_{\delta \, > \, 0} {[\inf_{x \in B(x^{\prime}, \delta)}{f(x)}]} 
                                          = \sup_{V \in \mathcal{N}(x^{\prime})} {[\inf_{x \in V}{f(x)}]}. 
    \end{align*}
\end{dfn}

\begin{dfn}(lower semi-continuous)
    Let $f: \mathbb{R}^n \to \mathbb{R} \cup \{+ \infty\}$ and let $x^{\prime}$ is a limit point of $f$. Then $f$ is lower semi-continuous at $x^{\prime}$ if and only if
    \[
        \liminf_{x \to x^{\prime}} {f(x)} \ge f(x^{\prime}), \text{\, or \,} \liminf_{x \to x^{\prime}} {f(x)} = f(x^{\prime}) 
    \]
\end{dfn}

\begin{dfn}(level set)
The lower level sets $lev_{\le \alpha}{f}$ is defined by
\[
lev_{\le \alpha} f := \{x \in \mathbb{R}^n \, | \, f(x) \le \alpha \}
\]
\end{dfn}

\begin{figure}[htbp]
\begin{center}
\includegraphics[width=50mm]{level_set.JPG}
\caption{Level set}
\end{center}
\end{figure}


\begin{prop}
Let arbitrary function $f: \mathbb{R}^n \to [- \infty, + \infty]$, the following are equivalent:

\begin{enumerate}
\renewcommand{\labelenumi}{(\roman{enumi})}
\item $f$ is lower semi-continuous at all $x^{\prime} \in \mathbb{R}^n$;
\item $lev_{\le \alpha} f := \{x^{s\prime} \in \mathbb{R}^n \, | \, f(x^{\prime}) \le \alpha \}$ is closed for every $\alpha$;
\item \text{epi}$f$ is a closed set in $\mathbb{R}^{n+1}$.
\end{enumerate}

\end{prop}

\begin{dfn}(level boundedness)
    A funciton  $f : \mathbb{R}^n \to \mathbb{R} \cup \{+ \infty \}$ is \textit{lower level bounded} 
    if the set $lev_{\le \alpha}{f}$ is bounded (possibly empty) for every $\alpha \in \mathbb{R}.$
\end{dfn}

\begin{thm}(conditions $\inf f = \min f$)
    Suppose $f : \mathbb{R}^n \to \mathbb{R} \cup \{+ \infty \}$ is lower semi-continuous, level-bounded and proper. 
    Then the set $argmin f$ is nonenpty and compact and the value $\inf f$ is finite.
    In other words, $\inf f = \min f$.
\end{thm}

\begin{cor}(lower bounds)
    Suppose $f: \mathbb{R}^n \to \mathbb{R} \cup \{+ \infty \}$ is lsc and proper.
    Then it is bounded from below on each bounded subset of $\mathbb{R}^n$. 
    Actually, it is minimum with respect to any compact subset of $\mathbb{R}^n$ that satisfies dom f.
\end{cor}


\section{}





\end{document}