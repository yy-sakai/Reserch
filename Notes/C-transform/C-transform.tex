\documentclass{jsarticle}

\usepackage{amsmath, amssymb}
\usepackage{type1cm}
\usepackage{mathtools}
\usepackage{bm}
\usepackage{amsthm}
\usepackage{color}
\usepackage[dvipdfmx]{graphicx}
\usepackage{listings,jvlisting}

\newtheorem{thm}{Theorem}[section]
\newtheorem{cor}{Corollary} [thm]
\newtheorem{lem}[thm]{Lemma}
\newtheorem{prop}[thm]{Proposition}
\theoremstyle{definition}
\newtheorem{dfn}{Definition}[section]
\newtheorem{ex}{Example}[section]


\lstset{
  basicstyle={\ttfamily},
  identifierstyle={\small},
  commentstyle={\smallitshape},
  keywordstyle={\small\bfseries},
  ndkeywordstyle={\small},
  stringstyle={\small\ttfamily},
  frame={tb},
  breaklines=true,
  columns=[l]{fullflexible},
  numbers=left,
  xrightmargin=0zw,
  xleftmargin=3zw,
  numberstyle={\scriptsize},
  stepnumber=1,
  numbersep=1zw,
  lineskip=-0.5ex
}


\begin{document}


\title{C-transform}

\maketitle

\section{Convex hull (Convex envelope)}


\begin{enumerate}
   \item $f(x)$を離散化する.
   \item $l = [0, - \infty ]$(点と2点の傾きを保存する.) 
   \item 傾き更新(繰り返し)
   \begin{enumerate}
   	\item ''一つ前の傾き''$ < $''現在の傾き''$\Rightarrow$ OK $l$に点と傾きを追加
   	\item ''一つ前の傾き''$ > $''現在の傾き''$\Rightarrow$ 一つ前の点と傾きを消去
   	
   	傾き計算:$c_i = \tfrac{f_{i+1} - f_i}{x_{i+1}- x_i}$
   \end{enumerate}
\end{enumerate}


\begin{figure}[htbp]
 \begin{minipage}{0.5\hsize}
  \begin{center}
   \includegraphics[width=70mm]{convex_hull-1.jpeg}
  \end{center}
  \caption{convex hull-1}
  \label{fig:one}
 \end{minipage}
 \begin{minipage}{0.5\hsize}
  \begin{center}
   \includegraphics[width=70mm]{convex_hull-2.JPG}
  \end{center}
  \caption{convex hull-2}
  \label{fig:two}
 \end{minipage}
\end{figure}

\section{Legendre Fenchel transform}
Legendre Fenchel transformとは、凸関数のある点$chi[i]$における接線の傾き(微分)$p[j]$を求めるもの。

$s[i]: chi[i]$と$chi[i+1]$を結んできる線分の傾きとすると, 一般に以下が成り立つ。
\[
s[i] < p[j] \leq s[i+1]
\]

\begin{figure}[htbp]
  \begin{center}
    \includegraphics[width=50mm]{legendre_fenchel-1.JPG}
    \caption{Legendre Fenchel-1}
  \end{center}
\end{figure}

\begin{enumerate}
   \item $p[j]$($x$座標を分割した値 )を固定(微分の値)
   \item 分割を十分小さくすると、$\nabla f(x) \approx s[i]$となるので、$i$を大きくすることで, $p[j] \leq s[i]$となるs[i]を探す。
   
             $s[i] < p[j] \leq s[i+1]$となるものを見つける。この$p[j]$が$chi[i]$での微分に当たる.
  \item 1,2を繰り返す。( $j$を大きくすることにより) 
\end{enumerate}

\begin{math}
f:\Omega \to \mathbb{R}
\end{math}

\begin{math}
f^{*}(x) = \displaystyle \sup_{y\in \Omega}{[x \cdot y - f(y)]}
\end{math}

 \color{red}狭義凸関数ならば, $f^*(x)$が単調増加なので、$s[i]$は常に増加.よって、  $s[i] < p[j] \leq s[i+1]$となるものを見つけることができる.\color{black}

\begin{ex}

$f(x) = \alpha |x|^2$


\begin{align*}
f^*(x) &= \displaystyle \sup_{y\in \Omega}{[x \cdot y - f(y)]} \quad (g(y) = x \cdot y - f(y))\\
           &=\displaystyle \sup_{y\in \Omega}{[x \cdot y - \alpha |y|^2]}\\
           &= \displaystyle \sup_{y\in \Omega}{[- \alpha(y^2 - \frac{1}{\alpha}xy)]}\\
           &= \displaystyle \sup_{y\in \Omega}{[- \alpha(y - \frac{1}{2 \alpha}x)^2 +  \frac{1}{4 \alpha}x^2]}\\
           &=  \frac{1}{4 \alpha}|x|^2 + \sup_{y\in \Omega}{[- \alpha(y - \frac{1}{2 \alpha}x)^2]}\\
           &= \frac{1}{4 \alpha}|x|^2 \qquad (\because  \text{$y = \frac{1}{2 \alpha}x$のとき,$g(y)$はsup.  $\Rightarrow f^*$は $y = \frac{1}{2 \alpha}x$のとき成立})\\
           &= \frac{1}{4 \alpha^2}f(x)\\
\end{align*}

よって、\color{red}$\alpha = \frac{1}{2}$のとき、$f^*(x) = f(x).$ \color{black}

また、$g(y)$が最大になるのは$y = \frac{1}{2 \alpha}x$のとき。$(g( \frac{1}{2 \alpha}x) = x \cdot \frac{1}{2 \alpha}x - \alpha  \frac{1}{4 \alpha^2}x^2 =  \frac{x^2}{4 \alpha} = f^*(x))$

$g(y) = x \cdot y - f(y)= x \cdot y - \alpha |y|^2$

$g'(y) = 0 \Leftrightarrow x - 2 \alpha y = 0 \Leftrightarrow x - f'(y) = 0 \, (\because f'(x) = 2 \alpha x) \Leftrightarrow f'(y) = x$

\color{red}一般に, $f(x) = \alpha |x| ^2$のとき, $f^*(x)$となる$x$は $f'(y) = x$より, ある点$y$での$f$の微分になる. かつ$x = f'(y) = 2\alpha y. $

つまり,'' $f^*(x)$を求めること''は ``$f'(y)$を求めること''に等しい.よって, Legendre Fenchel transformとは,凸関数のある点$chi[i]$における接線の傾き(微分)$p[j]$を求めるもの.
\color{black}

$\alpha = \frac{1}{2}$のとき, $f^*(x) = f(x),  かつf'(y) = x = 2\alpha y = y$.

従って, $s[i]: $''$chi[i+1]$と$chi[i]$を結んだ線の傾き'' = ``$chi[i]$での接線の傾き(微分)''.
つまり, 以下のプログラムで, $\alpha = \frac{1}{2}$のとき, $p$は$x$座標の分割であるので, $p[j] = s[i]$となる$p$が必ず存在する.

$\Rightarrow f^*_d(y) = f^*(y) = f(y)$, if $\alpha = \frac{1}{2}$. $\,  (f^*_d: f^*$を離散化したもの)(Legendre-Fenchel is an identity for $0.5 x$**$2$の意味)
\end{ex}

\begin{figure}[htbp]
 \begin{minipage}{0.5\hsize}
  \begin{center}
   \includegraphics[width=70mm]{legendre_fenchel-2.JPG}
  \end{center}
  \caption{convex hull-1}
  \label{fig:one}
  
 \end{minipage}
 \begin{minipage}{0.5\hsize}
  \begin{center}
   \includegraphics[width=70mm]{legendre_fenchel-3.JPG}
  \end{center}
  \caption{convex hull-2}
  \label{fig:two}
 \end{minipage}
\end{figure}


\lstinputlisting[caption=キャプション,label=ラベル]{/Users/sakaiyukito/Downloads/LABO/back-and-forth/reserch_git/code/legendre_fenchel.py}


\section{C-transform}
\begin{math}
\Omega \subset \mathbb{R}^n
\end{math}

\begin{math}
C:\Omega \times \Omega \to \mathbb{R}, C=C(x,y):=\frac{1}{2}|x-y|^2
\end{math}

\begin{math}
\phi = \phi(y), \phi :\Omega \to \mathbb{R}
\end{math}

\begin{math}
\phi^{c}(x)=\displaystyle \inf_{y\in \Omega}{[C(x, y)-\phi(y)]}
\end{math}

\begin{equation*}
C(x,y) = \frac{1}{2}|x-y|^2 = \frac{|x|^2}{2}-x \cdot y + \frac{|y|^2}{2}
\end{equation*}
\begin{align*}
\phi^{c}(x) &= \displaystyle \inf_{y\in \Omega}{[C(x, y)-\phi (y)]} \\
            &= \displaystyle \inf_{y\in \Omega}{[\frac{|x|^2}{2} -x \cdot y + \frac{|y|^2}{2} - \phi(y)]} \\
            &= \frac{|x|^2}{2} + \displaystyle \inf_{y\in \Omega}{[-(x \cdot y - \frac{|y|^2}{2} + \phi(y))]} \\ 
            &= \frac{|x|^2}{2} - \displaystyle \sup_{y\in \Omega}{[x \cdot y -( \frac{|y|^2}{2} - \phi(y))]} \\
                  \shortintertext{ここで、 $f(y) := \frac{|y|^2}{2} - \phi(y),  f^{*}(x) := \displaystyle \sup_{y\in\Omega}{[x \cdot y - f(y)]}$とすると、}
            &= \frac{|x|^2}{2} - f^*(x)
\end{align*}


\section{Property of L-F transform and C-transform}
\begin{math}
f:\Omega \to \mathbb{R}
\end{math}

\begin{math}
f^{*}(x) = \displaystyle \sup_{y\in \Omega}{[x \cdot y - f(y)]}
\end{math}

\begin{ex}
If 
\begin{math}
f(x) = \frac{|x|^2}{2}
\end{math}
,  
$f^*(x) = f(x)$.
\end{ex}

\begin{align*}
f^{*}(x) &= \displaystyle \sup_{y\in \Omega}{[x \cdot y - f(y)]} \\
              &= \displaystyle \sup_{y\in \Omega}{[x \cdot y - \frac{|y|^2}{2}]} \\
              &= \displaystyle \sup_{y\in \Omega}{[- \frac{1}{2}(y - x)^2 + \frac{x^2}{2}]} \\
              &= \frac{x^2}{2} + \displaystyle \sup_{y\in \Omega}{[- \frac{1}{2}(y - x)^2 ]} \\
              &= \frac{x^2}{2} + 0 = \frac{x^2}{2} = f(x)
\end{align*}


\begin{lem}
$f^{**}(x) \leq f(x)$ for every $x \in \Omega$.
\end{lem}


\begin{proof}
\begin{math}
f^{*}(x) = \displaystyle \sup_{y\in \Omega}{[x \cdot y - f(y)]}, 
- f^{*}(x) = \displaystyle \inf_{y\in \Omega}{[f(y) - x \cdot y]}. 
\end{math}
\begin{align*}
f^{**}(x) &= \displaystyle \sup_{y\in \Omega}{[x \cdot y - f^{*}(y)]} \\
          &= \displaystyle \sup_{y\in \Omega}{[x \cdot y - \displaystyle \sup_{z\in \Omega}{[y \cdot z - f(z)]}]} \\
          &= \displaystyle \sup_{y\in \Omega}{[x \cdot y + \displaystyle \inf_{z\in \Omega}{[f(z) - y \cdot z]}]} \\
          &\leq \displaystyle \sup_{y\in \Omega}{[x \cdot y - x \cdot y + f(x)]} \\
          &= f(x)
\end{align*}

Consider $f(x)$ is affine map $s. t.  f(x) = ax + b$.
\begin{align*}
f^{*}(x) &= \displaystyle \sup_{y\in \Omega}{[x \cdot y - f(y)]} \\
         &= \displaystyle \sup_{y\in \Omega}{[x \cdot y - (ay + b)]} \\
         &= \displaystyle \sup_{y\in \Omega}{[(x - a) y ]} - b \\
         &=\begin{cases*}
                   - b & if $x = a$, \\
                   + \infty & if $x \ne a$.
                   \end{cases*}
\end{align*}
Thus, 
\begin{align*}
f^{**}(x) &= \displaystyle \sup_{y\in \Omega}{[x \cdot y - f^{*}(y)]} \\
          &= ax + b \qquad ( \because \text{ $\displaystyle \sup_{y\in \Omega} {[-f^{*}(y)]} = b$ \quad if $y \ne a$)}\\
          &= f(x) 
\end{align*}

\end{proof}

\begin{lem}
$\phi^{cc}(x) \geq \phi(x)$ for every $x \in \Omega$.
\end{lem}

\begin{proof}

$\phi^{c}(x)=\displaystyle \inf_{y\in \Omega}{[C(x, y)-\phi(y)]}$. 
\begin{align*}
\phi^{cc}(x) &= \displaystyle \inf_{y\in \Omega}{[C(x, y)-\phi^c(y)]}\\ 
             &= \displaystyle \inf_{y\in \Omega}{[C(x, y)- \displaystyle \inf_{z\in \Omega}{[C(y, z)-\phi^c(z)]}]}\\
             &\geq \displaystyle \inf_{y\in \Omega}{[C(x, y)- (C(y, x)-\phi^c(x))]}\\
             &= \phi^c(x)
\end{align*}
\end{proof}


\begin{ex}



\begin{math}
C(x,y) = \frac{1}{2}|x-y|^2 = \frac{|x|^2}{2}-x \cdot y + \frac{|y|^2}{2}
\end{math}
とするとき,  Lemma 4.2.は

\begin{align*}
\phi^{c}(x) &= \displaystyle \inf_{y\in \Omega}{[C(x, y)-\phi (y)]} \\ 
            &= \displaystyle \inf_{y\in \Omega}{[\frac{1}{2}|x-y|^2 - \phi(y)]} \\
            &= \frac{|x|^2}{2} - \displaystyle \sup_{y\in \Omega}{[x \cdot y -( \frac{|y|^2}{2} - \phi(y))]} \\
            &= \frac{|x|^2}{2} - f^*(x)
\end{align*}



\begin{align*}
\phi^{cc}(x) &= \frac{|x|^2}{2} - \displaystyle \sup_{y\in \Omega}{[x \cdot y -( \frac{|y|^2}{2} - \phi^c(y))]} \\
             &= \frac{|x|^2}{2} - \displaystyle \sup_{y\in \Omega}{[x \cdot y -\{\frac{|y|^2}{2} - \displaystyle \inf_{z\in \Omega}{[\frac{1}{2}|y-z|^2 - \phi(z)]} \}]} \\
             &= \frac{|x|^2}{2} - \displaystyle \sup_{y\in \Omega}{[x \cdot y -\{\frac{|y|^2}{2} + \displaystyle \sup_{z\in \Omega}{[\phi(z) - \frac{1}{2}|y-z|^2]} \}]} \\
             &= \frac{|x|^2}{2} - \displaystyle \sup_{y\in \Omega}{[x \cdot y -\{\frac{|y|^2}{2} +  \displaystyle \sup_{z\in \Omega}{[\phi(z) - \frac{|y|^2}{2} + y \cdot z - \frac{|z|^2}{2}]} \}]} \\
             &= \frac{|x|^2}{2} - \displaystyle \sup_{y\in \Omega}{[x \cdot y -\{\frac{|y|^2}{2} - \frac{|y|^2}{2} +  \displaystyle \sup_{z\in \Omega}{[y \cdot z - (\frac{|z|^2}{2} - \phi(z)} )]\}]} \\
             &= \frac{|x|^2}{2} - \displaystyle \sup_{y\in \Omega}{[x \cdot y -  \displaystyle \sup_{z\in \Omega}{[y \cdot z - (\frac{|z|^2}{2} - \phi(z)} )]]} 
             \geq \frac{|x|^2}{2} - \displaystyle \sup_{y\in \Omega}{[x \cdot y - x \cdot y + \frac{|x|^2}{2} - \phi(x)} )] \\
             &= \frac{|x|^2}{2} + \displaystyle \inf_{y\in \Omega}{[ \displaystyle \sup_{z\in \Omega}{[y \cdot z - (\frac{|z|^2}{2} - \phi(z)} )]- x \cdot y]} 
             \geq \frac{|x|^2}{2} + \displaystyle \inf_{y\in \Omega}{[x \cdot y  - \frac{|x|^2}{2} + \phi(x)} - x \cdot y] \\
             &= \frac{|x|^2}{2} - \frac{|x|^2}{2} + \phi(x)\\
             &= \phi(x)
\end{align*}

\end{ex}



\end{document}


