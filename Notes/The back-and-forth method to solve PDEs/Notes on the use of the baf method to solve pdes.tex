\documentclass{jsarticle}
%
\usepackage{type1cm}
\usepackage{amsthm}
\usepackage{color}

\usepackage[dvipdfmx]{graphicx}
\usepackage{listings,jvlisting}
\usepackage{float}
\usepackage{here, amsmath, latexsym, amssymb, bm, ascmac, mathtools, multicol, tcolorbox, subfig, graphicx, comment, pgfplots}
%

% 「%」は以降の内容を「改行コードも含めて」無視するコマンド
\usepackage[%
 dvipdfmx,% 欧文ではコメントアウトする
 setpagesize=false,%
 bookmarks=true,%
 bookmarksdepth=tocdepth,%
 bookmarksnumbered=true,%
 colorlinks=true,%
 citecolor=green,%
 urlcolor=magenta,%
 linkcolor=blue,%
 pdftitle={},%
 pdfsubject={},%
 pdfauthor={},%
 pdfkeywords={}%
]{hyperref}
% PDFのしおり機能の日本語文字化けを防ぐ((u)pLaTeXのときのみかく)
\usepackage{pxjahyper}



\makeatletter
\@addtoreset{equation}{section}
\def\theequation{\thesection.\arabic{equation}}% renewcommand でもOK
\makeatother


\newtheorem{thm}{Theorem}[section]
\newtheorem{cor}{Corollary} [thm]
\newtheorem{lem}[thm]{Lemma}
\newtheorem{prop}[thm]{Proposition}
\theoremstyle{definition}
\newtheorem{dfn}{Definition}[section]
\newtheorem{ex}{Example}[section]
\newtheorem{rem}{Remark}[section]
\newtheorem{exer}{Exercise}[section]
\renewcommand{\labelenumi}{(\roman{enumi})}

%
\lstset{
  basicstyle={\ttfamily},
  identifierstyle={\small},
  commentstyle={\smallitshape},
  keywordstyle={\small\bfseries},
  ndkeywordstyle={\small},
  stringstyle={\small\ttfamily},
  frame={tb},
  breaklines=true,
  columns=[l]{fullflexible},
  numbers=left,
  xrightmargin=0zw,
  xleftmargin=3zw,
  numberstyle={\scriptsize},
  stepnumber=1,
  numbersep=1zw,
  lineskip=-0.5ex
}

\title{Note "THE USE OF THE BACK-AND-FORTH METHOD FOR WASSERSTEIN GRADIENT FLOWS TO SOLVE PDES"}

\author{坂井幸人}

\date{\today}

\begin{document}
\maketitle

\section{Dual Problem}
\subsection*{Condition}

$\mathcal{P} (\mathbb{R}^n):  (\mathbb{R}^n) \text{上の確率測度, 非負の測度で質量(mass)1}$

$U: \text{lsc(lower semi continuous) on} \mathcal{P} (\mathbb{R}^n)$ 

簡単のため、積分が$1$となる非負の$L^1(\mathbb{R}^n)$で考える。

\subsection*{$JKO$スキーム}
$JKO$スキームについて考える:
$$
\min_{\rho \in \mathcal{P}} U(\rho) + \frac{1}{2\tau} W_2^2(\rho, \mu),
$$


古い論文[JL]からの記号を使用すると、
2-Wasserstein距離のKantorovichの双対公式は次のように表されます:

$$
\frac{1}{2\tau} W_2^2(\rho, \mu) = \sup_{(\phi, \psi) \in \mathcal{C}} \left( \int \phi d\rho + \int \psi d\mu \right),
$$

ここで、$\mathcal{C}$は制約
$$
    \mathcal{C}  := \{(\phi, \psi) \in C(\Omega) \times C(\Omega) : \psi(x) + \phi(y) \leq \frac{1}{2 \tau} |x - y|^2 \}. 
$$
を満たす関数$(\phi, \psi)$の集合を表します。重要な点として、集合$\mathcal{C}$は凸であることに注意してください。[Exercise \ref{exer: convex}]


\begin{exer}
  \label{exer: convex}
  \hypertarget{Proof:convex}
  $\mathcal{C}$がconvex set であることを示せ。(Proof \ref{proof:convex})
\end{exer}

よって,
\begin{align*}
  \min_{\rho \in \mathcal{P}} U(\rho) + \frac{1}{2\tau} W_2^2(\rho, \mu) &= \min_{\rho \in \mathcal{P}} \left(U(\rho) + \sup_{(\varphi, \psi) \in \mathcal{C}} \left(\int \varphi \, d\rho + \int \psi \, d\mu\right)\right)\\
                                                                          &= \min_{\rho \in \mathcal{P}} \sup_{(\varphi, \psi) \in \mathcal{C}} \left(U(\rho) + \int \varphi \, d\rho + \int \psi \, d\mu\right)
\end{align*}

次に、\(\mathcal{P}\)と\(\mathcal{C}\)が凸であり、\(U\)が凸であるとすると、関数
\[
  \label{eq:L}
  L(\rho, (\varphi, \psi)) := U(\rho) + \int \varphi \, d\rho + \int \psi \, d\mu
\]
は\hypertarget{rho_convex}{\(\rho\)に関して凸関数}(\((\varphi, \psi)\)が固定された場合)(\hyperlink{Proof:rho_convex}{Proof})、
および\hypertarget{varphipsi_concave}{\((\varphi, \psi)\)に関しては凹関数}(実際には線形)です(\hyperlink{Proof:varphipsi_concave}{Proof})(さらにおそらく、いくつかのconditionが必要である).

このため、[ET, Ch. VI, Prop. 2.4 (p176)]のような最小最大の定理を適用して、\(\min\)と\(\sup\)の順序を交換することができます。
したがって、次のように結論付けることができます:
\begin{align*}
  \min_{\rho \in \mathcal{P}} U(\rho) + \frac{1}{2\tau} W_2^2(\rho, \mu) &= \sup_{(\varphi, \psi) \in \mathcal{C}} \min_{\rho \in \mathcal{P}} \left(U(\rho) + \int \varphi \, d\rho + \int \psi \, d\mu \right),\\
                                                                          &= \sup_{(\varphi, \psi) \in \mathcal{C}} \left(\min_{\rho \in \mathcal{P}} \left(U(\rho) + \int \varphi \, d\rho\right) + \int \psi \, d\mu\right),\\
                                                                          &= \sup_{(\varphi, \psi) \in \mathcal{C}} \left(\int \psi \, d\mu - U^*(- \varphi)\right).\\
\end{align*}
ここで、$U^*(\varphi)$は以下のように定義している。 
\[
  U^*(\varphi) := \sup_{\rho \in \mathcal{P}} \left(\int \varphi \, d\rho - U(\rho) \right).
\]

{\color{gray}
$\bold{detail:}$

\begin{align*}
  U^*(\varphi)   &= \sup_{\rho \in \mathcal{P}} \left(\int \varphi \, d\rho - U(\rho) \right).\\
  U^*(- \varphi) &= \sup_{\rho \in \mathcal{P}} \left(\int - \varphi \, d\rho - U(\rho) \right),\\
                 &= \sup_{\rho \in \mathcal{P}} \left(- \int \varphi \, d\rho - U(\rho) \right),\\
                 &= \sup_{\rho \in \mathcal{P}} \left\{ -  \left(\int \varphi \, d\rho + U(\rho) \right) \right\} ,\\
                 &= - \inf_{\rho \in \mathcal{P}} \left\{ \left(\int \varphi \, d\rho + U(\rho) \right) \right\} .\\
\end{align*}
よって、
\[
  \min_{\rho \in \mathcal{P}} \left(U(\rho) + \int \varphi \, d\rho\right) + \int \psi \, d\mu = - U^*(- \varphi)\\
\]
}

注意しておきますが、
{\color{teal}
与えられた \((\varphi, \psi) \in C\) に対して、{\color{red}\(\psi^c \geq \varphi\) }が成り立ちます。
}
ただし、
\[
  \psi^c(x) := \inf_y \left( \frac{1}{2\tau}|x-y|^2 - \psi(y)\right)
\]
は \(\psi\) の c-変換(c-transform)です。なぜなら、
$
    \mathcal{C}  := \{(\phi, \psi) \in C(\Omega) \times C(\Omega) : \psi(x) + \phi(y) \leq \frac{1}{2 \tau} |x - y|^2 \}. 
$
であり,
$
  \varphi(x) \le \frac{1}{2\tau}|x-y|^2 - \psi(y)
$
なので、$\varphi(x)$の中での$\sup$が$\psi^c$になるためである。

また、\(\rho \geq 0\) の場合、\(- U^*(-\varphi)\) は \(\varphi\) に関して増加する関数です。
よって、{\color{red}$-U^*(- \varphi) \le -U^*(- \psi^c)$}である。


{\color{teal}したがって、以下のようになります。}
\[
\sup_{(\varphi, \psi) \in C} \left(\int \psi \, d\mu - U^*(- \varphi)\right) \le \sup_\psi \left(\int \psi \, d\mu - U^*(- \psi^c)\right)
\]

また、$(\varphi, \psi) \in \mathcal{C} \implies (\psi^c, \psi) \in \mathcal{C}$
であるので,
\[
\sup_{(\varphi, \psi) \in C} \left(\int \psi \, d\mu - U^*(- \varphi)\right) \ge \sup_\psi \left(\int \psi \, d\mu - U^*(- \psi^c)\right)
\]
が成立する。よって,
\begin{equation}
  \label{eq:psi^c}
\sup_{(\varphi, \psi) \in C} \left(\int \psi \, d\mu - U^*(- \varphi)\right) = \sup_\psi \left(\int \psi \, d\mu - U^*(- \psi^c)\right)
\end{equation}

同様に、\(\mu \geq 0\) であるため、

\begin{equation}
  \label{eq:phi^c}
  \sup_{(\varphi, \psi) \in C} \left(\int \psi \, d\mu - U^*(- \varphi)\right) = \sup_\varphi \left(\int \varphi^c \, d\mu - U^*(- \varphi)\right)
\end{equation}
となります。\\

正規化アルゴリズムの主要なアイデアは、上記の右辺の$\phi$と$\psi$の2つの関数に対して交互に勾配上昇ステップを実行することであり、
$c$-変換を使用して$\phi$と$\psi$を変換します。
勾配は適切な重み付きSobolev空間で計算されます。

\section{Porous medium equation(多孔質媒体方程式)}

多孔質媒体方程式(PME)は、固定された \(m > 1\) に対して以下の偏微分方程式(PDE)のことをいう:
\[
\rho_t = \frac{{\partial \rho}}{{\partial t}} = \Delta (\rho^m)
\]
ここで、非負の解 \(\rho \geq 0\) に興味があります。
このPDEは、エネルギー関数
\[
U(\rho) := \frac{1}{{m-1}} \int \rho^m \, dx
\]
に基づくWasserstein勾配フローとして表現することができます。

$\rho \in \mathcal{P}(\mathbb{R}^n) \setminus L^m(\mathbb{R}^n)$のとき、$U(\rho)$は$+\infty$と定義されているとする。
ただし、これは$s \mapsto s^m$が$[0, \infty)$上で凸関数であるため、$U(\rho)$は$P(\mathbb{R}^n)$上で凸な汎関数です。
[JLL]のアルゴリズムでは、$\delta U^*$を計算する必要があります。ここで、

\begin{align*}
  U^*(\varphi) &= \sup_{\rho \in P} \left( \int \varphi \, d\rho - U(\rho) \right).\\
               &= \sup_{\rho \in P} \int \left(- \frac{1}{m-1}\rho^m + \rho\varphi\right) \, dx
\end{align*}

となります。
$\varphi$の前の符号が[JLL]とは異なることに注意してください、しかしこれは問題ありません。\\

$U^*$は、実質的には$U$のLagrange-Fenchel変換を$-\varphi$に対して行ったものですが、
重要な違いとして、$\mathcal{P}$がヒルベルト空間の部分集合ではなく、
測度と連続関数の間の双対性を使って内積の代わりに$\int \varphi d \rho$の積分を扱っています。\\

{\color{teal}
どちらにせよ、$\delta U^*(\varphi)$は通常の設定における$\partial U^*(\varphi)$に類似しており、
その場合、下半連続(つまり閉じている)凸関数$f$について、次の関係が成り立ちます。}
\[
  x \in \partial f^*(y) \iff z \cdot y - f(z) \text{が} z = x \text{において最大値を取る}
\]
言い換えると、
\[
  \partial f^*(y) = \operatorname{argmax}_x (x \cdot y - f(x))
\]
{\color{teal}
よって、$\delta U^*$を見つけるために, 以下の最大値を求める必要がある。
\[
  V(\rho) := \int \left(- \frac{1}{m-1}\rho^m + \rho\varphi\right) \, dx.
\]
}
\vspace\baselineskip 

\begin{lem}
  $\varphi \in \mathcal{C}$ と仮定し、以下のように定義されるとする.
  $$
    \rho_*(x) := \left( \frac{m-1}{m}(C + \varphi)_+ \right)^{\frac{1}{m-1}} 
  $$
  ただし、$C \in \mathbb{R}$は$\int \rho_* = 1$となる。
  $(s)_+ := max(s, 0)$と定義している。

  この時、$\rho_*$ は$\mathcal{P}(\mathbb{R}^n)$上の関数$V$の最大化関数。
\end{lem}

\vspace\baselineskip 
\begin{proof}
  $C$の選び方により、 $\rho_* \in \mathcal{P}(\mathbb{R}^n)$であることがわかる。
  次に、以下を示す。
  $$
  V(\rho) \leq V(\rho_*) \qquad \text{for all } \, \rho \in \mathcal{P}(\mathbb{R}^n) \cap L^1(\mathbb{R}^n).
  $$
  $\rho$を固定し、 
  $$
  \mu(x) := \rho(x) - \rho_*(x)
  $$
  とします。
  注意点として、
  \begin{equation}
    \label{eq:intmu}
     \int \mu \, dx = 0
  \end{equation}
  であり、また$\mu(x) \geq 0$は$\rho_*(x) = 0$の場所、つまり$\varphi(x) \geq C$の場所で成り立ちます。

  よって、
  \begin{align*}
    V(\rho) - V(\rho_*) &= V(\rho_* + \mu) - V(\rho_*)\\
                        &= \int \left(- \frac{1}{m-1} ((\rho_* + \mu)^m - \rho_*^{m}) + \mu \varphi \right) \, dx
  \end{align*}
  ここで、関数 $s \mapsto s^m$ は $[0, \infty)$ 上で凸であるため、$s, t \geq 0$ に対して 
  $$
    (s + t)^m \geq s^m + ms^{m-1}t, \qquad s + t \geq 0
  $$ 
  が成り立ちます。この不等式を適用することで、さらに次のように簡略化できます:
  \begin{align*}
    V(\rho) - V(\rho_*) &\leq \int \left(- \frac{1}{m-1} ((\rho_*^m + m \rho_*^{m-1} \mu) - \rho_*^{m}) + \mu \varphi \right) \, dx\\
                        &\leq \int \left(- \frac{m}{m-1} \rho_*^{m-1}\mu + \mu\varphi\right) \, dx\\
  \end{align*}

  $\rho_*$の式を利用することで、
  \begin{align*}
    \int \left(\frac{m}{m-1} \rho_*^{m-1}\mu + \mu\varphi\right) \, dx &= \int \left(-(C + \varphi)_+ \mu + \mu\varphi\right) \, dx \\
                                                                       &= \int \left(-(C + \varphi)_+ \mu + \mu \varphi + C \mu\right) \, dx \\
                                                                       &= \int \left(-(C + \varphi)_+ + \varphi + C\right)\mu \, dx \\  
  \end{align*}
  ただし、(\ref{eq:intmu})を利用する。

  ここで、以下のような観察をします:

  \[
    - (C + \varphi)_+ + \varphi + C = - (C + \varphi)_- \begin{cases} = 0 & \text{if } \varphi > - C \\ 
                                                                  \leq 0 & \text{if } \varphi \leq - C \end{cases}
  \]

  また、\(\mu(x) \geq 0\) は、\(\varphi(x) \geq C\) のとき成り立ちます。したがって、

  \[
     V(\rho) - V(\rho_*) \leq \int (- (C + \varphi)_+ + \varphi + C) \mu \, dx \leq 0
  \]

  したがって、\(\rho_*\) は \(V\) の最大化点であることがわかります。
\end{proof}

\section{GRADIENT ASCENT}

関数(\ref{eq:psi^c}),(\ref{eq:phi^c})の交互に勾配上昇を行う。
\[
  J(\varphi) = \int \varphi^c \, d\mu - U^*(-\varphi),
\]
\[
  I(\psi) = \int \psi \, d\mu - U^*(-\psi^c).
\]

$\varphi,  \psi$が $c$-凹の場合、の時、$H$空間での勾配は以下のようになる。
\[
  \nabla_H J(\varphi) = (\theta_1 I - \theta_2 \Delta)^{-1} (\delta U^*(- \varphi) - T_{\varphi \#} \mu),
\]
\[
  \nabla_H I(\psi) = (\theta_1 I - \theta_2 \Delta)^{-1} (\mu - T_{\psi \#} \delta U^* (- \psi^c)).
\]
また、以下が成り立つ。
\[
  T_\varphi(x) = x - \tau \nabla \varphi^c(x),
\]
\[
  T_\varphi^{-1}(x) = x - \tau \nabla \varphi(x)
\]

[JLL]のProposition 2.4.を参照すること。\\

もし$\mu$と$\varphi$が十分に滑らかであり、$\varphi$が$c$-凹関数である場合,
写像の変数変換の公式を用いて、押し出し密度(pushforward density)は以下のように求められる.

\begin{align*}
  T_{\varphi \#} \mu(x) &= \mu(T_\varphi^{-1}(x)) | \det \nabla T(T^{-1}(x)) |^{-1},\\
                        &= \mu(x - \tau \nabla \varphi(x)) | \det(I - \tau D^2 \varphi(x) |.
\end{align*}






%%%%%%%%%%%%%%%%%%%%%%%%%%%%%%%%%%%%%%%%%%%%%%%%%%%%%%%%%%%%%%%%%%%%%%%%%%%%%%%%%%%%%%%%%%%%%%%%%%%%%%
\section{appendix}


\begin{proof}[Proof of Exercise \ref{exer: convex}]
  \label{proof:convex}

  集合$\mathcal{C}$が凸であることを示すためには、
  $\mathcal{C}$内の任意の2点\((\varphi_1, \psi_1)\)と\((\varphi_2, \psi_2)\)を結ぶ線分、すなわち
  \[
    [(t\varphi_1 + (1-t)\varphi_2, t\psi_1 + (1-t)\psi_2)]
  \]
  が$\mathcal{C}$に含まれることを示せばよい。

  $\mathcal{C}$内の2点\((\varphi_1, \psi_1)\)と\((\varphi_2, \psi_2)\)を考え、制約条件\(\varphi_1(x) + \psi_1(y) \leq \frac{1}{2\tau}|x - y|^2\)および\(\varphi_2(x) + \psi_2(y) \leq \frac{1}{2\tau}|x - y|^2\)をすべての\(x, y \in \mathbb{R}^n\)に対して満たすとする。

  線分\([(t\varphi_1 + (1-t)\varphi_2, t\psi_1 + (1-t)\psi_2)]\)上の点\((\varphi, \psi)\)を考える。この点は\(t \in [0, 1]\)に対して\((\varphi, \psi) = (t\varphi_1 + (1 - t)\varphi_2, t\psi_1 + (1 - t)\psi_2)\)とパラメーター化できる。

  次に、\((\varphi, \psi)\)が制約条件\(\varphi(x) + \psi(y) \leq \frac{1}{2\tau}|x - y|^2\)をすべての\(x, y \in \mathbb{R}^n\)に対して満たすかどうかを確認する:
  \begin{align*}
    \varphi(x) + \psi(y) &= (t\varphi_1 + (1 - t)\varphi_2)(x) + (t\psi_1 + (1 - t)\psi_2)(y)\\
                        &= t(\varphi_1(x) + \psi_1(y)) + (1 - t)(\varphi_2(x) + \psi_2(y))\\
                        &\leq t\left(\frac{1}{2\tau}|x - y|^2\right) + (1 - t)\left(\frac{1}{2\tau}|x - y|^2\right)\\
                        &= \frac{1}{2\tau}|x - y|^2\\.
  \end{align*}

  よって、\((\varphi, \psi)\)が制約条件を満たすことがわかり、\((\varphi, \psi) \in C\)となります。
  したがって、集合Cが凸であることが示された。
\end{proof}

%%%%%%%%%%%%%%%%%%%%%%%%%%%%%%%%%%%%%%%%%%%%

\begin{proof}[Proof of \hyperlink{rho_convex}{$\rho$が変数とした時$L(\rho, (\varphi, \psi)$が凸関数}]
  \hypertarget{Proof:rho_convex}
  関数 \(L(\rho, (\varphi, \psi)) := U(\rho) + \int \varphi \, d\rho + \int \psi \, d\mu\) が、\(\rho\) を変数として固定された \((\varphi, \psi)\) の関数として凸関数であることを証明します。

まず、\(L(\rho, (\varphi, \psi))\) が \((\varphi, \psi)\) に関して線形であることを示します。つまり、
\begin{align*}
    L(\rho, (t\varphi_1 + (1-t)\varphi_2, t\psi_1 + (1-t)\psi_2) &= L(\rho, t(\varphi_1, \psi_1) + (1-t)(\varphi_2, \psi_2))\\ 
                                                                 &= tL(\rho, (\varphi_1, \psi_1)) + (1-t)L(\rho, (\varphi_2, \psi_2))
\end{align*}

を示します。
\begin{align*}
  L(\rho, (t\varphi_1 + (1-t)\varphi_2, t\psi_1 + (1-t)\psi_2) &= L(\rho, t(\varphi_1, \psi_1) + (1-t)(\varphi_2, \psi_2))\\
                                                               &= U(\rho) + \int (t\varphi_1 + (1-t)\varphi_2) \, d\rho + \int (t\psi_1 + (1-t)\psi_2) \, d\mu \\
                                                               &= t(U(\rho) + \int \varphi_1 \, d\rho + \int \psi_1 \, d\mu) + (1-t)(U(\rho) + \int \varphi_2 \, d\rho + \int \psi_2 \, d\mu) \\
                                                               &= tL(\rho, (\varphi_1, \psi_1)) + (1-t)L(\rho, (\varphi_2, \psi_2))
\end{align*}

次に、\(L(\rho, (\varphi, \psi))\) が凸関数であることを示します。つまり、\(\rho_1, \rho_2 \in \mathcal{P}\) および \(0 \leq \lambda \leq 1\) に対して、以下の不等式が成り立つことを示します:

\[
L(\lambda\rho_1 + (1-\lambda)\rho_2, (\varphi, \psi)) \leq \lambda L(\rho_1, (\varphi, \psi)) + (1-\lambda) L(\rho_2, (\varphi, \psi))
\]
\begin{align*}
  L(\lambda\rho_1 + (1-\lambda)\rho_2, (\varphi, \psi)) &= U(\lambda\rho_1 + (1-\lambda)\rho_2) + \int \varphi \, d(\lambda\rho_1 + (1-\lambda)\rho_2) + \int \psi \, d\mu \\
                                                      &= \lambda(U(\rho_1) + \int \varphi \, d\rho_1 + \int \psi \, d\mu) + (1-\lambda)(U(\rho_2) + \int \varphi \, d\rho_2 + \int \psi \, d\mu) \\
                                                      &= \lambda L(\rho_1, (\varphi, \psi)) + (1-\lambda) L(\rho_2, (\varphi, \psi))\\
\end{align*}
したがって、関数 \(L(\rho, (\varphi, \psi))\) は \((\varphi, \psi)\) を固定した場合に凸関数。
\end{proof}

%%%%%%%%%%%%%%%%%%%%%%%%%%%%%%%%%%%%%%%%%%%%%%%%%%%%%%%%%%%%%%%%%%%%%%%%%%%%%%%%%%%%%%%%

\begin{proof}[Proof of \hyperlink{varphipsi_concave}{\((\varphi, \psi)\)に関して$L(\rho, (\varphi, \psi))$は凹関数}]
  \hypertarget{Proof:varphipsi_concave}
  関数 \(L(\rho, (\varphi, \psi)) := U(\rho) + \int \varphi \, d\rho + \int \psi \, d\mu\) が \((\varphi, \psi)\) を変数として固定された \(\rho\) の関数として凹関数(実際には線形関数)であることを示します。

  まず、\(L(\rho, (\varphi, \psi))\) が \(\rho\) に関して線形であることを示します。
  つまり、
  \[
    L(t\rho_1 + (1-t)\rho_2, (\varphi, \psi)) = tL(\rho_1, (\varphi, \psi)) + (1-t)L(\rho_2, (\varphi, \psi))
  \]
  を示す。
  \begin{align*}
    L(t\rho_1 + (1-t)\rho_2, (\varphi, \psi)) &= U(t\rho_1 + (1-t)\rho_2) + \int \varphi \, d(t\rho_1 + (1-t)\rho_2) + \int \psi \, d\mu \\
                                              &= tU(\rho_1) + (1-t)U(\rho_2) + \int \varphi \, d(t\rho_1 + (1-t)\rho_2) + \int \psi \, d\mu \\
                                              &= t(U(\rho_1) + \int \varphi \, d\rho_1 + \int \psi \, d\mu) + (1-t)(U(\rho_2) + \int \varphi \, d\rho_2 + \int \psi \, d\mu) \\
                                              &= tL(\rho_1, (\varphi, \psi)) + (1-t)L(\rho_2, (\varphi, \psi))
  \end{align*}

  次に、\(L(\rho, (\varphi, \psi))\) が凹関数であることを示します。
  つまり、\(\rho_1, \rho_2 \in \mathcal{P}\) および \(0 \leq \lambda \leq 1\) に対して、以下の不等式が成り立つことを示す:
  \[
    L(\lambda\rho_1 + (1-\lambda)\rho_2, (\varphi, \psi)) \geq \lambda L(\rho_1, (\varphi, \psi)) + (1-\lambda) L(\rho_2, (\varphi, \psi))
  \]
  \begin{align*}
    L(\lambda\rho_1 + (1-\lambda)\rho_2, (\varphi, \psi)) &= U(\lambda\rho_1 + (1-\lambda)\rho_2) + \int \varphi \, d(\lambda\rho_1 + (1-\lambda)\rho_2) + \int \psi \, d\mu \\
                                                          &= \lambda(U(\rho_1) + \int \varphi \, d\rho_1 + \int \psi \, d\mu) + (1-\lambda)(U(\rho_2) + \int \varphi \, d\rho_2 + \int \psi \, d\mu) \\
                                                          &= \lambda L(\rho_1, (\varphi, \psi)) + (1-\lambda) L(\rho_2, (\varphi, \psi))
  \end{align*}
  以上より、関数 \(L(\rho, (\varphi, \psi))\) は \((\varphi, \psi)\) を変数として固定された \(\rho\) の関数としては凹関数(実際には線形関数)であることが示されました。
\end{proof}

\begin{dfn}(lower limits)
  Let $f: \mathbb{R}^n \to \mathbb{R} \cup \{+ \infty\}$ and let $x^{\prime}$ is a limit point of $f$. Then the lower limit of function $f$ is defined by 
  \begin{align*}
      \liminf_{x \to x^{\prime}}{f(x)} &= \lim_{\delta \, \searrow \, 0} {[\inf_{x \in B(x^{\prime}, \delta)}{f(x)}]} \\
                                        &= \sup_{\delta \, > \, 0} {[\inf_{x \in B(x^{\prime}, \delta)}{f(x)}]} 
                                        = \sup_{V \in \mathcal{N}(x^{\prime})} {[\inf_{x \in V}{f(x)}]}. 
  \end{align*}
\end{dfn}

\begin{dfn}(lower semi-continuous)
  Let $f: \mathbb{R}^n \to \mathbb{R} \cup \{+ \infty\}$ and let $x^{\prime}$ is a limit point of $f$. Then $f$ is lower semi-continuous at $x^{\prime}$ if and only if
  \[
      \liminf_{x \to x^{\prime}} {f(x)} \ge f(x^{\prime}), \text{\, or \,} \liminf_{x \to x^{\prime}} {f(x)} = f(x^{\prime}) 
  \]
\end{dfn}

\end{document}