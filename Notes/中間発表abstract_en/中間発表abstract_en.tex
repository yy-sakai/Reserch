\documentclass[a4j,10pt, twocolumn, dvipdfmx]{article}

\usepackage{graphicx}
\usepackage{float}
\usepackage{here, amsmath, latexsym, amssymb, bm, ascmac, mathtools, multicol, tcolorbox, subfig, graphicx, comment, pgfplots, listings,jvlisting, type1cm, amsthm, comment, algorithm, algpseudocode, url}

%---------------------------------------------------
% ページの設定
%---------------------------------------------------
\setlength{\pdfpagewidth}{8.5in}  % DO NOT CHANGE THIS
\setlength{\pdfpageheight}{11in}  % DO NOT CHANGE THIS
\pagestyle{empty}
\setlength{\headheight}{0truemm}
\setlength{\parindent}{1zw}

\newtheorem{thm}{Theorem}[section]
\newtheorem{cor}{Corollary} [thm]
\newtheorem{lem}[thm]{Lemma}
\newtheorem{prop}[thm]{Proposition}
\theoremstyle{definition}
\newtheorem{dfn}{Definition}[section]
\newtheorem{ex}{Example}[section]




\lstset{
    basicstyle={\ttfamily},
    identifierstyle={\small},
    commentstyle={\smallitshape},
    keywordstyle={\small\bfseries},
    ndkeywordstyle={\small},
    stringstyle={\small\ttfamily},
    frame={tb},
    breaklines=true,
    columns=[l]{fullflexible},
    numbers=left,
    xrightmargin=0zw,
    xleftmargin=3zw,
    numberstyle={\scriptsize},
    stepnumber=1,
    numbersep=1zw,
    lineskip=-0.5ex
}


%---------------------------
%1段組
\begin{document}
\twocolumn
[
\begin{center}
{\huge Application to Partial Differential Equations for Optimal Transport Problem}
\end{center}
\begin{flushright}
\begin{tabular}{rr}
{\Large \ } &\ \\
2215011016&Yukito Sakai\\
{\Large \ } &\ \\
\end{tabular}
\end{flushright}

%\begin{abstract}
%    Recently, a new method for solving partial differential equations has been proposed by utilizing the solution of the optimal transportation problem.
%    This method is particularly effective for solving nonlinear partial differential equations. 
%    In this paper, we explain an efficient way to solve the optimal transportation problem with strictly convex cost functions.
%    \newline
%\end{abstract}

\vspace{2truemm}
]
%-----------------------------------

\section{Introduction}

Interest in the optimal transportation problem has increased in recent years due to the discovery of a deep relationship between the optimal transportation problem and partial differential equations.
The back-and-forth method (Matt Jacobs, Flavien Léger, 2020) is a new method for solving partial differential equations using the solution of the optimal transportation problem. 
This method is particularly effective for solving nonlinear partial differential equations and can solve a wider range of problems than previous methods because it is faster and does not require stability conditions. 
I am interested in the back-and-forth method and my ultimate goal is to create a program that solves partial differential equations using this method. 

In this paper, I will explain the back-and-forth method, which efficiently solves the optimal transportation problem with strictly convex costs.

\section{Background}

%--------------Optimal transform problem---------------------
\subsection{Optimal transport problem}
\textbf{Optimal transform problem(Monge,1781).}
Find a method to minimize cost, 
which depends on weight and distance, 
for transporting sand from a sandpit with measure $\mu$ to a hole with the same volume and measure $\nu$ using a mapping $T$.

\begin{dfn}[pushforward measure]
    Given a mapping $T$ that transports from measure $\mu$ to measure $\nu$ $(T_\#\mu = \nu)$, 
    the pushforward measure is defined as:
    \begin{equation*}
        \nu (A) =  T_\#\mu (A) := \mu (T^{-1} (A)) \qquad A \subset \Omega.
    \end{equation*}
\end{dfn}

\begin{figure}[htbp]
    \begin{center}
        \includegraphics[width=0.5\textwidth]{images/transport_map2.JPG}
        \caption{transport map}
    \end{center}
\end{figure}

The optimal transportation problem can be formalized as follows.
\newline
%-----------------------------------------------------------

%--------------Optimal transform problem formalize---------------------
\textbf{Optimal transform problem.}
Let $\Omega \subset \mathbb{R}^d$ be a convex set,
$c: \Omega \times \Omega \to \mathbb{R}$ be a cost function that represents the cost of transporting from $x$ to $y$, 
$\mu, \nu : \Omega \to \mathbb{R}$ be probability measures on $\Omega$, and $T: \Omega \to \Omega$ be a mapping such that $T_\#\mu = \nu$, 
meaning it transforms measure $\mu$ to measure $\nu$.

The minimum cost to transport $\mu$ to $\nu$ is denoted by $C(\mu, \nu)$ and is given by:

\begin{equation*}
    C(\mu, \nu) = \inf_T \int_\Omega c(x, T(x)) d \mu (x).
\end{equation*}

The optimal cost $C$ can be treated as a problem of maximizing the volume of sand transported, 
rather than fixing the transportation location.
\newline
%-----------------------------------------------------------

%--------------Kantorovich dual problem---------------------
\textbf{Kantorovich dual problem.}
When $\nu$ and $\mu$ are probability measures and $c(x, y)$ represents the cost of transporting from $x$ to $y$, the cost C can be expressed as maximizing the volume of sand transported as follows:
\begin{equation*}
    C(\mu, \nu) = \sup_{\phi, \psi} \int \phi d\nu + \int \psi d\mu
\end{equation*}
where $\phi(y)$ and $\psi(x)$ are the Kantorovich potentials and satisfy the inequality:
\begin{equation*}
    \phi(y) + \psi(x) \le c(x, y).
\end{equation*}

If there exists an optimal map from $\mu$ to $\nu$, the maximum value of the dual problem $\phi_(y), \psi_(x)$ can be restored. Then,
\begin{equation*}
    \phi_*(y) + \psi_*(x) = c(x, y).
\end{equation*}

%-----------------------------------------------------------
\subsection{c-transform}

\begin{dfn}[$c$-transform]
    Given a continuous function $\phi: \Omega \to \mathbb{R}$, 
    we define its $c$-transform $\phi^c: \Omega \to \mathbb{R}$ as follows:
    \begin{equation*}
        \phi^c(x) := \inf_{y \in \Omega} (c(x, y) - \phi(y))
    \end{equation*}
    
    A function $\phi$ is called a $c$-concave function if there exists a continuous function $\psi: \Omega \to \mathbb{R}$ such that $\phi = \psi^c$.
    Additionally,  pair of functions $(\phi, \psi)$ is called $c$-conjugate if $\phi = \psi^c$ and $\psi = \phi^c$.
\end{dfn}

If $(\phi, \psi)$ is c-conjugate, then the maximum values $\phi_*$ and $\psi_*$ are given by:
\begin{align*}
    \phi_*(y) = \psi^c_*(y) = \inf_{x \in \Omega}{c(x, y) - \psi_*(x) },\\
    \psi_*(x) = \phi^c_*(x) = \inf_{y \in \Omega}{c(x, y) - \phi_*(y) }.
\end{align*}


\section{The back-and-forth method}
The Kantorovich dual problem, given by
\begin{equation*}
    C = \sup_{\phi, \psi} \int \phi d\nu + \int \psi d\mu,
\end{equation*}
can be expressed as the $\sup$ of 
\begin{equation*}
    J(\phi) = \int \phi d \nu + \int \phi^c d \mu,
\end{equation*}
and 
\begin{equation*}
    I(\psi) = \int \psi^c d \nu + \int \psi d \mu.
\end{equation*}
using c-transformation. In other words, $c = \sup J = \sup I$.

The back-and-forth method solves the Kantorovich dual problem rapidly by finding the supremum of $J$ and $I$ using gradient ascent, 
and by alternating back and forth between $J$ and $I$ through c-transformations.

\begin{algorithm}
    \caption{The back-and-forth method}
    \begin{algorithmic}
        \State Given probability densities $\mu$ and $\nu$, set $\phi_0 = 0, \psi_0 = 0$, and iterate:
        \State \begin{align*}
            \phi_{n + \frac{1}{2}} &= \phi_{n} + \sigma \nabla_{\dot{H}^1} J(\phi_{n}),\\
            \psi_{n + \frac{1}{2}} &= (\phi_{n + \frac{1}{2}})^c,\\
            \psi_{n + 1} &= \psi_{n + \frac{1}{2}} + \sigma \nabla_{\dot{H}^1} I(\psi_{n + \frac{1}{2}}),\\
            \phi_{n + 1} &= (\psi_{n + 1})^c.
        \end{align*}
    \end{algorithmic}
\end{algorithm}

Where,
\begin{align*}
    \nabla_{\dot{H}^1} J(\phi) = (- \Delta)^{-1} (\nu - T_{\phi \#} \mu),\\
    \nabla_{\dot{H}^1} I(\psi) = (- \Delta)^{-1} (\mu - T_{\psi \#} \nu).\\
\end{align*}

\section{Results}
An example of $\nu$ and $\nu$ optimal transport using the back-and-forth method is shown.

\begin{ex}
    Initial conditions

    $\mu = \left\{
        \begin{array}{ll}
            1 & 0.3 \le x \le 0.8\\
            0 & otherwise \\
        \end{array},
    \right.
    $

    $\nu = \left\{
        \begin{array}{ll}
            1 & - 0.8 \le x \le - 0.3 \\
            0 & otherwise \\
        \end{array}.
    \right.
    $


\end{ex}
\begin{figure}[htb]
    \begin{center}
        \begin{minipage}{0.45\hsize}
            \includegraphics[width=\linewidth]{images/back-and-forth-update_ex1_0.png}
        \end{minipage}
        \begin{minipage}{0.45\hsize}
            \includegraphics[width=\linewidth]{images/back-and-forth-update_ex1_19.png}
        \end{minipage}
    \end{center}
    \caption{Example 4.1.}
\end{figure}
\label{Example 1}

\begin{ex}
    Initial conditions

    $\mu = \left\{
        \begin{array}{ll}
            0.5 & 0 \le x \le 0.5 \\
            0 & otherwise \\
        \end{array}
    \right.
    $

    $\nu = \left\{
        \begin{array}{ll}
            1 & - 0.5 \le x \le - 0.25 \\
            0 & otherwise \\
        \end{array}
    \right.
    $

    \begin{figure}[htb]
        \begin{center}
            \begin{minipage}{0.45\hsize}
                \includegraphics[width=\linewidth]{images/back-and-forth-update_ex2_0.png}
            \end{minipage}
            \begin{minipage}{0.45\hsize}
                \includegraphics[width=\linewidth]{images/back-and-forth-update_ex2_19.png}
            \end{minipage}
        \end{center}
        \caption{Example 4.2.}
    \end{figure}

\end{ex}

\section{Issues}

\begin{figure}[htbp]
    \begin{center}
        \includegraphics[width=0.7\linewidth]{images/x[iopt]explanation1.JPG}
        \caption{Transport map $T_{\phi \#} \mu$}
        \includegraphics[width=0.7\linewidth]{images/x[iopt]explanation2.JPG}
        \caption{Transport map $T_{\psi \#} \nu$}
    \end{center}
\end{figure}
\label{x[iopt]explanation}

Since the $x$-coordinates before and after transportation have a one-to-one correspondence, 
the mass (value) of $\nu$ and $\mu$ at a certain $x$ is reflected directly to the destination of the transportation.

Therefore, even though the calculation results in $T_{\phi \#} \mu = \nu, T_{\psi \#} \nu = \mu$ , a large error occurs.
\section{Next steps}
\begin{enumerate}
    \item Creation of a program for transport maps without a one-to-one correspondence (for improved accuracy)
    \item Creation of a program that solves partial differential equations using the back-and-forth method
    \item Development of a program to find new solutions for nonlinear equations
\end{enumerate}

\bibliography{references}
\begin{thebibliography}{99}
    \bibitem{1}
    Matt Jacobs, Flavien Léger.
    \newblock A fast approach to optimal transport: the back-and-forth method.
    \newblock Numerische Mathematik, 2020. 
    \bibitem{2}
    Shin-ichi OHTA.
    \newblock 最適輸送理論とその周辺.
    \newblock \url{http://www4.math.sci.osaka-u.ac.jp/~sohta/jarts/kino09.pdf}.
    \bibitem{3}
    Shinichiro KOBAYASHI.
    \newblock 距離コストに対する最適輸送問題について.
    \newblock \url{https://www.math.sci.hokudai.ac.jp/~wakate/mcyr/2020/pdf/kobayashi_shinichiro.pdf}.

    \end{thebibliography}

\end{document}