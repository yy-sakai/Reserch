\chapter{The back-and-forth method} 
\label{ch:The back-and-forth method}
本論文では多孔質勾配方程式について考える。
多孔質媒体方程式(PME)は、以下の偏微分方程式のことをいう$:$
\[
\rho_t = \frac{{\partial \rho}}{{\partial t}} = \gamma\Delta (\rho^m)
\]
ここで、本論文では非負の解 \(\rho \geq 0\) を考えている。
このPDEは、エネルギー関数
\[
    U(\rho) := \frac{\gamma}{{m-1}} \int \rho^m \, dx
\]
に基づくWasserstein勾配流として表現することがきる。
ただし、$\gamma > 0, m > 1$とする。
ここで、
\begin{align*}
    U^*(\varphi) &= \sup_{\rho \in P} \left( \int \varphi \, d\rho - U(\rho) \right).\\
                &= \sup_{\rho \in P} \int \left(- \frac{\gamma}{m-1}\rho^m + \rho\varphi\right) \, dx
\end{align*}
となる。

\section{The back-and-forth method algorithm}
\label{sect:The back-and-forth method algorithm}
The back-and-forth methodは標準的な勾配上昇法より収束速度が大幅に速い。
The back-and-forth methodを用いてKantorovich双対問題を解くことで一般化最適輸送問題
(\ref{eq: GOT})
\begin{equation*}
        \rho_* = \underset{\rho \in L^1(\Omega)} {\operatorname{argmin}} \left( U(\rho) + \frac{1}{2 \tau}W_2^2(\rho, \mu) \right), 
\end{equation*}
を求め、JKOスキームを求めることで多孔質勾配方程式を効率的に解くことができる。
多孔質勾配方程式
\[
    \partial_t \rho - \gamma\Delta(\rho^m) = 0  \quad(m > 1, \gamma > 0)
\]
はエネルギー関数$U$に基づくWasserstein 勾配流として表現できる。
ここで$U$は
\[
    U(\rho) := \frac{\gamma}{{m-1}} \int \rho^m \, dx
\]
であり、$\Omega$に対して凸である。
ここで、$U$は凸である場合には、Thm. \ref{thm: duality}より一般化最適輸送問題は双対性を持つ。
具体的には、$U$が凸であるとき、一般化最適輸送問題は$J$(\ref{eq:J})と$I$(\ref{eq:I})を用いて次のような双対性を表せる$:$
$$
    \inf_{\rho \in L^1(\Omega)} U(\rho) + \frac{1}{2 \tau} W^2_2(\rho, \mu) = \sup_\varphi J(\varphi) = \sup_\psi I (\psi)
$$
さらに一般化最適輸送問題の解は以下の性質を持つ$:$
\begin{equation}
    \label{eq:dualJI}
    \rho_* \in \delta U^*(\varphi_*), \quad \varphi_* \in \delta U(\rho_*), \quad \rho_* = T_{\varphi_* \#} \mu,
\end{equation}
ただし、$\rho$は(\ref{eq: GOT})の最小化解であり、$(\varphi_*, \psi_*)$は(\ref{eq:dualJI})の最大化される関数である。
$J, I$はどちらも制約がない汎関数であるので、いずれかの関数の最大化解を標準的な勾配上昇法で見つけることができる。
しかし$I$または$J$だけを使用することは問題の対称性を崩すため、関数のいずれかにだけ焦点を当てるのではなく、back-and-forthメソッドでは$I$と$J$の交互の勾配上昇ステップを行うことで対称性を保存することができる。
$I$と$J$は異なる変数を使用しているが、$c$-変換を使用することで$\psi$と$\varphi$を変換できる。


次に、問題(\ref{eq:dualJI})の$(\varphi_*, \psi_*)$を見つけるthe back-and-forth methodのアルゴリズムについて考える。
アルゴリズムは以下のアイデアに基づいている$:$
\begin{enumerate}
    \item Back-and-Forth スキーム:$I$と$J$における勾配上昇ステップを交互に繰り返す。
    \item Sobolev空間における$H^1$-型ノルム$H$における勾配上昇ステップは以下のようになる:
        \begin{equation*}
            \nabla_H J(\varphi) = (\Theta_1 \text{Id} - \Theta_2 \Delta)^{-1} \left[\delta U^*(- \varphi)-  T_{\varphi \#} \mu \right], \\
        \end{equation*}
        \begin{equation*}
            \nabla_H I(\psi) = (\Theta_1 \text{Id} - \Theta_2 \Delta)^{-1} \left[\mu - T_{\psi \#} (\delta U^*(\psi^c))\right].
        \end{equation*}
    ただし、$H$-勾配は(\ref{eq:J-gradient}),(\ref{eq:I-gradient})、$\delta F(\varphi)$は(\ref{eq:delta J}),(\ref{eq:delta I})を参照すること。
    また、適切な$\Theta_1, \Theta_2$については\cite{MR4238775}を参照すること。
\end{enumerate}

上記を踏まえ、the back-and-forth methodのアルゴリズムをAlgorithm \ref{al:baf-method}に示す。

\begin{algorithm}[tb]
    \caption{The back-and-forth scheme for solving(\ref{eq:J}) and (\ref{eq:I})}
    \label{al:baf-method}
    \begin{algorithmic}
    \State \textbf{Input:} $\mu$ and $\varphi_0$, iterate:
    \State{
        \begin{align*}
            \varphi_{k + \frac{1}{2}} &= \varphi_k + \nabla_H J(\varphi_k)\\
            \psi_{k + \frac{1}{2}} &= (\varphi_{k + \frac{1}{2}})^c\\
            \psi_{k + 1} &= \psi_{k + \frac{1}{2}} + \nabla_H I(\psi_{k + \frac{1}{2}})\\
            \varphi_{k + 1} &= (\psi_{k + 1})^c
        \end{align*}
    }
    \State \textbf{return} $\varphi_*$
    \end{algorithmic}
\end{algorithm}

\((\varphi, \psi) \in C\) に対して、\(\psi\) の c-変換(c-transform)は
$
    \psi^c(x) = \inf_{y \in \Omega} \left( \frac{1}{2\tau}|x-y|^2 - \psi(y)\right)
$
であるので、
\(\psi^c \geq \varphi\)が成り立つ。
また、\(\rho \geq 0\) の場合、\(- U^*(-\varphi)\) は \(\varphi\) に関して減少しない関数である。
このことと、Lemma. \ref{lem:c-transform}から、以下の関係が確認でき、$J, I$のそれぞれの関係性を保持したまま値を上昇できていることがわかる。
\[
    J(\varphi_{k + \frac{1}{2}}) \leq I((\varphi_{k + \frac{1}{2}})^c), \quad I(\psi_{k+1}) \leq J((\psi_{k+1})^c). 
\]


$I$および$J$の双対問題が十分な精度で解かれた後、(\ref{eq: GOT})式における最適密度\(\rho_*\)は(\ref{eq:dualJI})式の双対関係を通じて復元することができる。
本論文では多孔質勾配方程式について考えるため、$\rho_*$は一意に定まる。
そのため、単に\(\rho_* = \delta U^*(- \varphi_*)\)と同一視することができる。
この方法で\(\rho_*\)を回復することは、押し出し測度の関係\(\rho_* = T_{\varphi_* \# }\mu\)とは異なり、\(\varphi^*\)の数値微分を計算する必要がないためとても有効である。

これらの結果を組み合わせることで、偏微分方程式の解$\rho_*$を求めるJKOスキームのAlgorithm \ref{al:JKO}が得られる。

\begin{algorithm}[tb]
    \caption{JKO scheme}
    \label{al:JKO}
    \begin{algorithmic}
    \State \textbf{Input:} $\rho^{(0)}$, initialize $\varphi^{(0)} = - \delta U(\rho^{(0)}).$
    \For{$n = 0 \, \ldots \, N$}
        \State{$\varphi^{(n+1)} \leftarrow \,$ Run Algorithm \ref{al:baf-method} with $\mu = \rho^{(n)}$ and $\varphi_0 = \varphi^{(n)}$,}
        \State{$\rho^{(n+1)} = \delta U^*(- \varphi^{(n+1)})$.}
    \EndFor
    \State \textbf{return} $\rho_*$
    \end{algorithmic}
\end{algorithm}

%%%%%%%%%%%%%%%%%%%%%%%%%%%%%%%%%%%%%%%%%%%%%%%%%%%%%%%%%%%%%%%%%%%%%%%%%%%%%%%%%%%%%%%%%%%%%%%%%%

\section{最適輸送写像}
\label{sect:最適輸送写像}

$I$および$J$の勾配(\ref{eq:J-gradient})(\ref{eq:I-gradient})を計算するには、押し出し測度も計算する必要がある。
密度$\mu$および可逆な写像$T: \Omega \to \Omega$が与えられた場合、ヤコビアンの公式を使用して押し出し測度 $T_\# \mu$を計算できる。

The back-and-forth methodの構造のおかげで、$\varphi, \psi$がどちらも$c$-凹の場合にのみ、$T_{\varphi \#} \mu$および$T_{\psi \#} \delta U^*(- \psi^c)$を計算すればよい。
Proposition. \ref{prop:transport map}より、
\[
    T_\varphi ^{-1}(y) = y - \tau \nabla \varphi(y), T_\psi^{-1}(x) = x - \tau \nabla \psi(x),
\]
\[
    y = T_\varphi(x), \quad x = T^{-1}_\varphi (y), \quad dx = |\det \nabla T^{-1}_\varphi (y)| dy 
\]

$I$および$J$の勾配(\ref{eq:J-gradient})(\ref{eq:I-gradient})、さらに$J$と$I$ の第一変分(\ref{eq:delta J}), (\ref{eq:delta I})から、
$T_{\varphi \#} \mu$, $T_{\psi \#} \delta U^*(- \psi^c)$は以下のように求められる。
\begin{align*}
    T_{\varphi \#} \mu (A) &= \int_A T_{\varphi \#} \mu (x)\, dx = \mu(T^{-1}_\varphi (A))\\
                        &= \int_{T^{-1}_\varphi (A)} \mu(x) \, dx\\
                        &= \int_A \mu \left( T^{-1}_\varphi(y) \right) |\det \nabla T^{-1}_\varphi (y)|\, dy = \int_A \frac{\mu \left( T^{-1}_\varphi(y) \right)}{|\det \nabla T\left(T^{-1}_\varphi (y)\right)|}\, dy\\
                        &= \int_A \mu(y - \tau \nabla \varphi(y))|\det (I - \tau D^2 \varphi(y))|\, dy\\
                        &= \mu(y - \tau \nabla \varphi(y))|\det (I - \tau D^2 \varphi(y))|
\end{align*}
\begin{align*}
    T_{\psi \#} \delta U^* (- \psi^c) (A)   &= \int_A T_{\psi \#} \delta U^* (- \psi^c) (y)\, dy =  \delta U^* (- \psi^c)(T^{-1}_\psi (A))\\
                                            &= \int_{T^{-1}_\psi (A)} \delta U^* (- \psi^c) (y) \, dx\\
                                            &= \int_A  \delta U^* (- \psi^c)  \left(T^{-1}_\psi (x) \right) |\det \nabla T^{-1}_\psi (x)|\, dy = \int_A \frac{\ \delta U^* (- \psi^c)  \left( T^{-1}_\psi(x) \right)}{|\det \nabla T\left(T^{-1}_\psi (x)\right)|}\, dy\\
                                            &= \int_A \delta U^* (- \psi^c)(x - \tau \nabla \psi(x))|\det (I - \tau D^2 \psi(x))|\, dx\\
                                            &= \delta U^* (- \psi^c)(x - \tau \nabla \psi(x))|\det (I - \tau D^2 \psi(x))|
\end{align*}
よって、まとめると以下のようになる。
\begin{equation}
    \label{eq:pushforward Tphi}
    T_{\varphi \#} \mu (y) =\mu(y - \tau \nabla \varphi(y))|\det (I - \tau D^2 \varphi(y))|
\end{equation}
\begin{equation}
    \label{eq:pushforward Tpsi}
    T_{\psi \#} \delta U^* (- \psi^c)(x) =\delta U^* (- \psi^c)(x - \tau \nabla \psi(x))|\det (I - \tau D^2 \psi(x))|
\end{equation}
