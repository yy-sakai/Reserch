\chapter{予備知識} \label{ch:background}
本研究内で偏微分方程式の解法に用いるthe back-and-forth methodの導入に必要な要素を説明する。
\section{最適輸送}
\label{sect:最適輸送}
まず、the back-and-forth methodが開発された元の問題である最適輸送問題を導入する。
\subsection{Mongeの最適輸送問題}
\label{sect:Mongeの最適輸送問題}
back-and-forth methodは本来最適輸送問題を求めるアルゴリズムである。
まず、最適輸送問題について説明する。
1781年、G.Mongeによって最適輸送問題が提唱された。

\begin{monge}
    ある砂山から砂山(測度$\mu$)と同じ体積の穴(測度$\nu$)に砂を運ぶ(写像$T$)。
    輸送にかかるコストは重さと移動距離に依存する時,コストを最小にする方法を求めよ。
\end{monge}

最適輸送問題を数学的に解釈するため、押し出し測度(pushforward measure)を定義する。
\begin{dfn}
    $\mu$ から $\nu$ へ輸送する写像を$T$とするとき$(T_\#\mu = \nu)$,
    押し出し測度は
    \begin{equation*}
        \nu (A) =  T_\#\mu (A) := \mu (T^{-1} (A)) \qquad A \subset \Omega
    \end{equation*}
    で定義される(図 \ref{fig:pushforward measure})。
    また、押し出し測度をテスト関数$f: \Omega \to \mathbb{R}$に対する押し出し測度eの積分として定義すると以下のようになる:
    \begin{equation}
        \label{def:pushforward_int}
        \int_{\Omega} f(y)d\, T_\# \mu (y) = \int_\Omega f(T(x)) d \mu(x).
      \end{equation}
\end{dfn}

\begin{figure}[htbp]
    \begin{center}
        \includegraphics[width=0.8\textwidth]{images/transport_map2.JPG}
        \caption{輸送写像}
        \label{fig:pushforward measure}
    \end{center}
\end{figure}

上記の押し出し測度を用いて最適輸送問題の定式化を行う。

\begin{monge}
    $\Omega \subset \mathbb{R}^d$を凸集合とし、
    $c : \Omega \times \Omega \to \mathbb{R}$を$x$から$y$への輸送コストを表すコスト関数、
    $\mu, \nu : \Omega \to \mathbb{R}$を$\Omega$上の確率測度、
    $T : \Omega \to \Omega$を写像とする。
    ここで$T_\#\mu = \nu$は、測度$\mu$を$\nu$に変換することを意味する。

    $\mu$から$\nu$への輸送の最小コストは、$C(\mu, \nu)$で表され、次のように計算される:

    \[
    C(\mu, \nu) = \int_{\Omega} c(x, T(x)) \,dx
    \]
\end{monge}

\subsection{Kantorovich双対問題}
\label{sect:Kantorovich双対問題}
最適コスト$C$は、輸送の位置を固定するのではなく、輸送される砂の体積を最大化する問題として考えることができる。
この考えはKantorovichによって導入された。\cite{MR0096552}
これをKantorovich双対問題という。


\begin{Kantorovich}
$\mu$と$\nu$が確率測度であり、$c(x, y)$が$x$から$y$への輸送コストを表す場合、コスト$C$は以下のように、砂を輸送する体積を最大化する問題として表現できる:
\[
    C(\mu, \nu) = \sup_{\varphi, \psi} \left( \int \varphi \,d\nu + \int \psi \,d\mu \right)
\]
ここで、$\varphi(x)$および$\psi(y)$はKantorovich ポテンシャルであり、以下の不等式を満たす:
\[
    \varphi(x) + \psi(y) \leq c(x, y)
\]
\end{Kantorovich}
さらに多孔質媒体方程式は固定された$m > 1$に対して、エネルギー関数$U(\rho) = \frac{1}{m - 1} \int \rho^m \, dx$に基づくWasserstein 勾配流として表現できる。
ただし、2-Wasserstein距離のKantorovichの双対公式は次のように表される:

\begin{equation}
    \label{eq:wasserstein dual}
    \frac{1}{2\tau} W_2^2(\rho, \mu) = \sup_{(\varphi, \psi) \in \mathcal{C}} \left( \int \varphi d\rho + \int \psi d\mu \right),
\end{equation}

ここで、$\mathcal{C}$は制約
\[
    \mathcal{C}  := \{(\varphi, \psi) \in C(\Omega) \times C(\Omega) : \varphi(x) + \psi(y) \leq \frac{1}{2 \tau} |x - y|^2 \}. 
\]
を満たす関数$(\varphi, \psi)$の集合、$\tau$はスキーム内の時間ステップを表す。


Mongeの最適輸送問題を直接解く代わりにKantorovich双対問題を解くことで、最適写像を求めることができる。
双対問題を解く利点として以下3点があげられる。
\begin{itemize}
    \item 圧力変数$\varphi$は密度変数$\rho$よりも正則性が高いため、離散的な近似スキームに適している。
    \item 双対汎関数の微分を計算するための明示的な式があるため、勾配上昇法で解くことができる。
    \item 双対問題は制約のない形で表現されるため、厳しい制約を持つ問題にも適用できる。    
\end{itemize}

back-and-forth methodはこの2-Wasserstein距離におけるKantorovich双対問題を解くことで最適写像を求める。

\section{c-変換(c-transform)}
\label{sect:c-変換(c-transform)}
最適輸送問題における制約条件を解消するため、c-変換を導入する。
\subsection{凸包(convex hull)}
\label{sect:凸包(convex hull)}
\begin{dfn}[凸集合]
    $\mathbb{R}^n$の部分集合 $X$ が\textit{凸集合}であるとは、すべての $x, y \in C$ および $\forall \theta \in [0, 1]$ に対して次が成り立つときである:
    \[
        \theta x + (1 - \theta )y \in X
    \]
\end{dfn}

\begin{dfn}
    集合 $X \subset \mathbb{R}^n$ の凸包(convex hull)とは、$X$ を含む(唯一で)最小の凸集合を指し、$conv X$と表す。

\end{dfn}

\begin{dfn}
    関数 $f$ が \textit{proper} であるとは、少なくとも一つの $x \in $ に対して $f(x) < +\infty$ であり、
    すべての $x \in $ に対して $f(x) > - \infty$ であるときをいう。
\end{dfn}

\begin{dfn}(下半連続)
    関数 $f: \mathbb{R}^n \to \mathbb{R} \cup \{+ \infty\}$ とし、$x^{\prime}$ を $f$ の収束点とする。
    このとき、$f$ が $x^{\prime}$ で下半連続であるとは、次が成り立つことである:
    \[
        \liminf_{x \to x^{\prime}} {f(x)} \ge f(x^{\prime})
    \]
\end{dfn}

\begin{dfn}
    $\mathcal{C}$が凸集合であるとき、関数 $f \in \mathcal{C}$ が$C$上で\textit{凸関数}であるとは、次が成り立つ場合に限る:
    \[
        f((1 - \lambda)x + \lambda y) \le (1 - \lambda) f(x) + \lambda f(y), \quad \forall x, y \in \mathcal{C}, \quad \lambda \in [0, 1].
    \] 
\end{dfn}

任意の関数$f$を凸関数とし、$f \in \mathcal{C} \ne \emptyset$とする。
$f$を$\mathcal{C} \subset \mathbb{R}^n$に含まれない点では$f(x) = +\infty$として$\mathbb{R}^n$に拡張する。
このとき、上記の定義は次の定義と等しい。

\begin{dfn}
関数$f: \mathbb{R}^n \to \mathbb{R} \cup \{+ \infty\}$が、常に$+ \infty$でない場合、および任意の$\bm{x}, \bm{y} \in \mathbb{R}^n$、$\lambda \in [0, 1]$ に対して次が成り立つとき、$f$は凸関数である。
\[
f((1 - \lambda)\bm{x} + \lambda \bm{y}) \le (1 - \lambda) f(\bm{x}) + \lambda f(\bm{y})
\]
さらに、$f: \mathbb{R}^n \to \mathbb{R}$に対して、$f$が凹関数ならば$-f$は凸関数である。
\end{dfn}

\begin{dfn}
    $f: \mathbb{R}^n \to \mathbb{R} \cup \{ +\infty\},  C \subset \mathbb{R}^n$ とする。
    この時$f$のepigraphを次のように定義する:
    \[
    \text{epi} f := \{(\bm{x},  y) \, | \, \bm{x} \in C,  y \in \mathbb{R},  y \ge f(\bm{x}) \}.
    \]
    ここで注意として、$f(x) = +\infty$は$x \notin C$のときである。
\end{dfn}


関数 \( f : \mathbb{R}^n \to \mathbb{R} \) に対して、凸包の概念も同様に存在する。
$conv f$は$f$よりも大きくない最大の凸関数である。
ここで、関数 $f: \mathbb{R}^n \to \mathbb{R}$ が関数 $g: \mathbb{R}^n \to \mathbb{R}$ によってmajorizeされるとは、
すべての $x$ に対して $f(x) \leq g(x)$ であることを指す。
また、$conv f$はすべての凸関数 $g \leq f$ の点ごとの上限(pointwise supremum)であり、点ごとの上限は凸関数である。
すなわち、
\[
    conv f = \sup_g \{ g \le f | g \text{ is convex function}\}
\]
と表すことができる。


%Legendre-Fenchel transform %%%%%%%%%%%%%%%%%%%%%%%%%%%%%%%%%%%%%%%%%%%%%%%%%%%%%%%%%%%%%%%%%%%%%%%%%%%%%%%%%%%

\subsection{Legendre-Fenchel変換(Legendre-Fenchel transform)}
\label{sect:Legendre-Fenchel変換(Legendre-Fenchel transform)}
\begin{dfn}
    関数$f \in \mathcal{C}$を凸で微分可能な関数とするとき、$f$のLegendre-Fenchel transformは以下のように定義される。
    \begin{equation}
        f^*(p) := sup_x\{p \cdot x - f(x) \} 
    \end{equation}
    また、
    \begin{equation}
        (f^*)^*(x) = f^{**} (x) := sup_p\{p \cdot x - f^*(p) \}
    \end{equation}
    も同様に定義される。
\end{dfn}
ここで、
\[
    f^{**} = \text{cl(conv} f)  
\]
であることから、常に$f^{**} \le f$がわかる(証明については\cite[Thm. 11.1(p474)]{MR1491362}を参照)。

\subsection{$c$-変換($c$-transform)}
\label{sect:c-変換(c-transform)}
\begin{dfn}
    $\varphi$, $\psi$が$\Omega$上の連続関数の空間$\mathcal{C}(\Omega)$内の関数のとき、それぞれの$c$-変換は以下のように定義される。
    \begin{equation}
        \label{dfn:backward-c-transform}
        \psi^c(x) := \inf_{y \in \Omega} \left( \frac{1}{2\tau}|x-y|^2 - \psi(y)\right) = \frac{1}{\tau}\inf_{y \in \Omega} \left( \frac{1}{2}|x-y|^2 - \tau\psi(y)\right)
    \end{equation}
    \begin{equation}
        \label{dfn:forward-c-transform}
        \varphi^c(y) := \inf_{x \in \Omega} \left( \frac{1}{2\tau}|x-y|^2 - \varphi(x)\right) = \frac{1}{\tau}\inf_{x \in \Omega} \left( \frac{1}{2}|x-y|^2 - \tau\varphi(x)\right)
    \end{equation}
    また、 $\phi$ が $c$-凹関数とは、 $\varphi = \psi^c$ となる$\psi \in C(\Omega)$ が存在することをいう。
    さらに関数の組 $(\varphi, \psi) \in \mathcal{C}$が $c$-共役であるとは, $\varphi = \psi^c$ かつ $\psi = \varphi^c$のときをいう。
\end{dfn}


\begin{lem}
    \label{lem:c-transform}
    $\varphi, \psi \in C(\Omega)$のとき、
    $\forall x \in \Omega \text{に対し,}\phi^{cc} \ge \varphi$
    が成り立つ。また、$\phi^{cc} = \varphi$の必要十分条件は$\varphi$が$c$-concaveの時である。
    ただし、$\varphi^{ccc} = \varphi^c$は常に成立する.[JL](Lemma 1(i))
\end{lem}


\section{輸送写像(Transport map)}
\label{輸送写像(Transport map)}
\begin{prop}
    \label{prop:transport map}
    $c$-concaveな関数$\varphi \in C(\Omega)$に対して、写像$T_\varphi(x)$がwell-definedかつalmost everywhereにおいて一意である。
    
    \begin{equation}
        T_\varphi(x):= \underset{y \in \Omega} {\operatorname{argmin}}\left( \frac{1}{2 \tau} |x - y|^2 - \varphi(y) \right) 
    \end{equation}
    言い換えれば、almost everyの$x$に対して最小化問題 $\inf_{y \in \Omega} \frac{1}{2 \tau} |x - y|^2 - \varphi(y)$ は唯一の最小値をもつ関数$T_\varphi(x)$を提供する。
    さらに、$u \in C(\Omega)$の場合、almost everyな$x, y \in \Omega$に対して次の$c$-transformの摂動公式が成り立つ。
    
    \begin{equation}
        \lim_{\varepsilon \to 0} \frac{(\varphi + \varepsilon u)^c(x) - \varphi^c(x)}{\varepsilon} = - u(T_\varphi(x))
    \end{equation}
    
    最後に、以下の式も成り立つ。
    \[
        T_\varphi(x) = x - \tau \nabla \varphi^c(x),
    \]
    \[
        T_\psi(y) = y - \tau \nabla \psi^c(y).
    \]
    また、$\varphi, \psi$が$c$-共役の場合、$T_\varphi(T_\psi(y)) = y$および$T_\psi(T_\varphi(x)) = x$ はalmost everywhereで成り立つ。
    具体的には、
    \[
        T_\varphi^{-1}(x) = x - \tau \nabla \varphi(x),
    \] 
    \[
        T_\psi^{-1}(y) = y - \tau \nabla \psi(y) .
    \] 
\end{prop}

\begin{prop}
    \label{prop:wasserstein}
    $\mu$と$\nu$は$L^1(\Omega)$上の非負密度関数で、質量が等しく$1$である。すなわち、
    \[
        \int_{\Omega} \mu(x)dx = \int_{\Omega} \nu(y)dy = 1
    \]
    であるのでば、次の式が成り立ちます。
    \[
        \frac{1}{2\tau}W_2^2(\mu, \nu) = \sup_{\varphi \in C(\Omega)} \left( \int_{\Omega} \varphi^c(x) \mu(x)dx + \int_{\Omega} \varphi(y) \nu(y)dy \right) 
    \]

    \[
        \frac{1}{2\tau}W_2^2(\mu, \nu) = \sup_{\psi \in C(\Omega)} \left( \int_{\Omega} \psi(x) \mu(x)dx + \int_{\Omega} \psi^c (y) \nu(y)dy \right) 
    \]
\end{prop}
この結果により、最適輸送写像の存在と一意性が保証される。
さらに、$\frac{1}{2\tau}W_2^2(\mu, \nu)$を$\varphi$もしくは$\psi$の式として表すことができることがわかる。

\begin{thm}
    \label{thm:pushforward measure}
    $\mu$と$\nu$は$L^1(\Omega)$上の非負密度関数で、質量が等しく$1$であるので、次の条件を満たす$c$-共役のペア$(\varphi_*, \psi_*)$が存在する:
    \begin{align*}
        \varphi_* &\in \underset{\varphi \in C(\Omega)} {\operatorname{argmax}}\left\{\int_{\Omega} \varphi^c(x) \mu(x)dx + \int_{\Omega} \varphi(y) \nu(y)dy \right\} \\
        \psi_* &\in \underset{\psi \in C(\Omega)} {\operatorname{argmax}} \left\{\int_{\Omega} \psi(x) \mu(x)dx +z \int_{\Omega} \psi^c(y) \nu(y)dy \right\}
    \end{align*}
さらに、$T_\varphi$は$\mu$を$\nu$に送る唯一の最適輸送写像であり、$T_\psi$は$\nu$を$\mu$に送る唯一の最適輸送写像である。
これを押し出し速度を用いて表すと、$T_{\varphi_* \#} \mu = \nu$および$T_{\psi_* \#} \nu = \mu$となる。

また、式(\ref{eq:wasserstein dual})から2-Wasserstein距離$W^2_2(\mu, \nu)$と関数$\varphi_*, \psi_*$との関係は次のようになる。

\[
\frac{1}{2\tau}W^2_2(\mu, \nu) = \int_{\Omega} \psi_*(x) \mu(x)dx + \int_{\Omega} \varphi_*(y) \nu(y)dy.
\]
\end{thm}




\section{双対問題(Dual Problem)}
\label{sect:Dual Problem}
(\ref{eq:JKOscheme})のように、$\rho_*$を最小化問題(\ref{eq:minimizer})の解とすることで、以下のような問題として表すことができる(一般化最適輸送問題)。
\begin{equation}
    \label{eq: GOT}
        \rho_* = \underset{\rho \in L^1(\Omega)} {\operatorname{argmin}} \left( U(\rho) + \frac{1}{2 \tau}W_2^2(\rho, \mu) \right), 
\end{equation}
ここで、$\mu \in L^1(\Omega)$で、非負の測度である。

以後本論文において、以下を仮定して考える。
\begin{ass}
\label{ass:2}

$\rho \in \mathcal{P}(\mathbb{R}^n) \setminus L^m(\mathbb{R}^n)$のとき、$U(\rho) = \infty$と定義されているとする。
さらに$U$は、properで、凸かつ、下半連続な汎関数 $U: L^1(\Omega) \to \mathbb{R} \cup  \{+ \infty\}$ である。
また、超線形成長条件(superlinear growth)を持つ$s: \mathbb{R} \to \mathbb{R} \cup  \{+ \infty\}$が存在し、以下の条件を満たす$:$
$$
    U(\rho) \geq \int_\Omega s(\rho(y)) \, dy
$$

これにより、非負の密度に対して関数$U(\rho)$が適切に定義される。
さらに任意の$A \in \mathbb{R}$に対して、集合$\{ \rho \in L^1(\Omega): U(\rho) < A\}$が弱収束位相において弱コンパクトであることを保証している。

もしくは、Definition \ref{dnf:1}から$\rho, \mu \in L^1(\Omega)$は$\int \rho = 1, \int \mu = 1$、かつ$\varphi, \psi \in C(\Omega)$より、$\rho, \mu$は絶対連続である。
よって、$\mathcal{P}$がヒルベルト空間の部分集合ではなく、
測度と連続関数の間の双対性を使って内積の代わりに$\int \varphi d \rho$の積分を扱うことができる。
さらに絶対連続であるとき、唯一の最適輸送写像$T_*$と逆写像$T_*^{-1}$が存在する。
これを押し出し測度を用いて表すと、
$T_{\varphi_* \#} \mu = \nu$および$T_{\psi_* \#} \nu = \mu$が存在する。
さらにKantorovich双対問題の最適輸送写像を見つけることができる。

\begin{comment}

確率測度 $\mu$ がルベーグ積分に関して絶対連続であるとき、測度 $\mu$ と連続関数 $f$ の積分は、次の式で表されます。

\[
\int_\Omega f(x) \, \mu(dx) = \int_\Omega f(x) \, \rho(x) dx
\]


ここで、$\rho$ は $\mu$ の密度関数です。

この式は、ルベーグ積分と測度の積分の関係から導かれます。

ルベーグ積分では、関数 $f$ がルベーグ可積分であることを条件に、次の式で定義されます。

\[
\int_\Omega f(x) \, dx = \sup \left\{ \sum_{k=1}^\infty \lambda_k f(x_k) \mid \{x_k\} \subset \Omega, \sum_{k=1}^\infty \lambda_k < \infty \right\}
\]

確率測度 $\mu$ がルベーグ積分に関して絶対連続であるとき、$\mu$ の密度関数 $\rho$ は存在し、$\mu(A) = \int_A \rho(x) dx$ が成立します。

したがって、次の等式が成立します。

\[
\int_\Omega f(x) \, \mu(dx) = \int_\Omega f(x) \, \rho(x) dx
\]
この式は、確率測度と連続関数の積分を計算する際によく使用されます。
\end{comment}

\end{ass}


最小化問題は、
\[
    U^*(\varphi) := \sup_{\rho \in \mathcal{P}} \left(\int \varphi \, d\rho - U(\rho) \right).
\]
を用いることで、
\begin{equation*}
    \min_{\rho \in \mathcal{P}} U(\rho) + \frac{1}{2\tau} W_2^2(\rho, \mu) = \sup_{(\varphi, \psi) \in \mathcal{C}} \left(\int \psi \, d\mu - U^*(- \varphi)\right)
\end{equation*}
と表すことができる(\ref{eq:dual})。

$\mu$が非負測度であるため、上式を満たすには、できるだけ大きな値を持つように$\psi$を選ぶ必要がある。
そこで与えられた \((\varphi, \psi) \in C\) に対して、\(\psi\) の c-変換(c-transform)は
$
    \psi^c(x) = \inf_{y \in \Omega} \left( \frac{1}{2\tau}|x-y|^2 - \psi(y)\right)
$
であるので、
\(\psi^c \geq \varphi\)が成り立つ。


なぜなら、
$
    \mathcal{C}  := \{(\varphi, \psi) \in C(\Omega) \times C(\Omega) : \varphi(x) + \psi(y) \leq \frac{1}{2 \tau} |x - y|^2 \}
$
であり、
$
  \varphi(x) \le \frac{1}{2\tau}|x-y|^2 - \psi(y)
$
なので、$\varphi(x)$の中での$\sup$が$\psi^c$になるためである。

また、\(\rho \geq 0\) の場合、\(- U^*(-\varphi)\) は \(\varphi\) に関して減少しない関数である。
よって、$-U^*(- \varphi) \le -U^*(- \psi^c)$である。


したがって、以下が満たされる。
\[
\sup_{(\varphi, \psi) \in \mathcal{C}} \left(\int \psi \, d\mu - U^*(- \varphi)\right) \le \sup_\psi \left(\int \psi \, d\mu - U^*(- \psi^c)\right)
\]
また、$(\varphi, \psi) \in \mathcal{C} \implies (\psi^c, \psi) \in \mathcal{C}$
であるので、
\[
\sup_{(\varphi, \psi) \in \mathcal{C}} \left(\int \psi \, d\mu - U^*(- \varphi)\right) \ge \sup_\psi \left(\int \psi \, d\mu - U^*(- \psi^c)\right)
\]
が成立する。よって、
\begin{equation}
  \label{eq:psi^c}
\sup_{(\varphi, \psi) \in \mathcal{C}} \left(\int \psi \, d\mu - U^*(- \varphi)\right) = \sup_\psi \left(\int \psi \, d\mu - U^*(- \psi^c)\right)
\end{equation}

同様に、\(\mu \geq 0\) であるため、
$
    \varphi^c(y) := \inf_{x \in \Omega} \left( \frac{1}{2\tau}|x-y|^2 - \varphi(x)\right)
$
を用いることで、

\begin{equation}
  \label{eq:phi^c}
  \sup_{(\varphi, \psi) \in \mathcal{C}} \left(\int \psi \, d\mu - U^*(- \varphi)\right) = \sup_\varphi \left(\int \varphi^c \, d\mu - U^*(- \varphi)\right)
\end{equation}
が成り立つ。

$c$-変換(\ref{dfn:backward-c-transform})(\ref{dfn:forward-c-transform})を利用して$\varphi$もしくは$\psi$のみの式にすることで、制約条件$\mathcal{C}$を排除することができる。
すなわち、(\ref{eq:dual})は(\ref{eq:psi^c})(\ref{eq:phi^c})から制約条件のない以下2つの汎関数の最大値を求める問題と考えることができる。
また、$J,I$はproper、上半連続、凹関数である。
\begin{equation}
    \label{eq:J}
    J(\varphi):= \int_{\Omega} \varphi^c(x) \,d\mu(x) - U^*(- \varphi)
\end{equation}

\begin{equation}
    \label{eq:I}
    I(\psi):= \int_{\Omega} \psi(x) \, d\mu(x) - U^*(- \psi^{c})
\end{equation}

すなわち、
\[
\sup_{(\varphi,\psi) \in \mathcal{C}} \int \psi(x) d\mu(x) - U^*(- \varphi) = \sup J(\varphi) = \sup I(\psi).
\]
が成立する(双対性)。

さらに、$\varphi, \psi$がc-凹である場合、$J$と$I$は以下のような一次変分を持つ:

\begin{equation}
    \label{eq:delta J}
    \delta J(\varphi) = \delta U^*(- \varphi) - T_{\varphi \#} \mu,
\end{equation}
\begin{equation}
    \label{eq:delta I}
    \delta I(\psi) = \mu - T_{\psi \#} \delta U^* (- \psi^c).
\end{equation}

\begin{thm}
    \label{thm: duality}
        $\mu \in L^1(\Omega)$であり、$U$がAssumption\ref{ass:2}を満たし、さらに$\delta U(\mu)$が定数関数でない場合、
        次の条件を満たす一意の密度$\rho_*$と$c$-共役な関数のペア$(\varphi_*, \psi_*)$が存在する。
    \begin{equation*}
        \rho_* = \underset{\rho \in L^1(\Omega)} {\operatorname{argmin}} \, U(\rho) + \frac{1}{2\tau} W_2(\rho, \mu), \quad \varphi_* \in \underset{\varphi \in C(\Omega)} {\operatorname{argmax}} \, J(\varphi), \quad \psi_* \in \underset{\psi \in C(\Omega)} {\operatorname{argmax}} \, I(\psi),
    \end{equation*}
    
    \begin{equation*}
    U(\rho_*) + \frac{1}{2\tau} W_2^2(\rho_*,\mu) = J(\varphi_*) = I(\psi_*) \quad (\text{双対性}), 
    \end{equation*}
    
    \begin{equation*}
        \rho_* \in \delta U^*(\varphi_*), \quad \phi_* \in \delta U(\rho_*), \quad \rho_* = T_{\varphi_* \#} \mu.
    \end{equation*}
    \end{thm}



\section{多孔質勾配方程式の導入}
\label{sect:多孔質勾配方程式の導入}
多孔質媒体方程式(PME)は、固定された \(m > 1\) に対して以下の偏微分方程式のことをいう:
\[
\rho_t = \frac{{\partial \rho}}{{\partial t}} = \gamma\Delta (\rho^m)
\]
ここで、本論文では非負の解 \(\rho \geq 0\) を考えている。
このPDEは、エネルギー関数
\[
    U(\rho) := \frac{\gamma}{{m-1}} \int \rho^m \, dx
\]
に基づくWasserstein勾配流として表現することがきる。
ここで、
\begin{align*}
    U^*(\varphi) &= \sup_{\rho \in P} \left( \int \varphi \, d\rho - U(\rho) \right).\\
                &= \sup_{\rho \in P} \int \left(- \frac{\gamma}{m-1}\rho^m + \rho\varphi\right) \, dx
\end{align*}
となる。

$U^*$は、実質的には$U$のLagrange-Fenchel変換を$-\varphi$に対して行ったものである(\ref{eq:dual})。
ここで、$L^1(\Omega) \subset \mathcal{P}$がヒルベルト空間の部分集合ではなく、
測度と連続関数の間の双対性を使って内積の代わりに$\int \varphi d \rho$の積分を扱うことに注意する。
これは$\rho, \mu$が絶対連続であることに基づいている。


$\delta U^*(\varphi)$は通常の設定における$\partial U^*(\varphi)$に類似しており、
その場合下半連続な(つまり閉集合な)凸関数$f$について、次の関係が成り立つ。

\[
    x \in \partial f^*(y) \iff z \cdot y - f(z) \text{が} z = x \text{において最大値を取る}
\]
言い換えると、
\[
    \partial f^*(y) = \operatorname{argmax}_x (x \cdot y - f(x))
\]
よって、$\delta U^*$を見つけるために, 以下の最大値を求める必要がある。
\[
    V(\rho) := \int \left(- \frac{\gamma}{m-1}\rho^m + \rho\varphi\right) \, dx.
\]

\begin{lem}
    \label{lem:rho_opt}
    $\varphi \in C(\Omega)$ と仮定し、以下のように定義されるとする.
    $$
        \rho_*(x) := \left( \frac{m-1}{m\gamma}(C + \varphi)_+ \right)^{\frac{1}{m-1}} 
    $$
    ただし、$C \in \mathbb{R}$は$\int \rho_* = 1$となるように設定する。
    $(s)_+ := max(s, 0)$と定義している。

    この時、$\rho_*$ は$L^1(\Omega)$上の関数$V$の最大化関数。
\end{lem}
\begin{proof}
    $C$の選び方により、 $\rho_* \in \mathcal{P}(\mathbb{R}^n)$であることがわかる。
    次に、以下を示す。
    $$
    V(\rho) \leq V(\rho_*) \qquad \text{for all } \, \rho \in \mathcal{P}(\mathbb{R}^n) \cap L^1(\mathbb{R}^n).
    $$
    ここで$\rho$を固定し、 
    $$
    \mu(x) := \rho(x) - \rho_*(x)
    $$
    とする。
    注意点として、$C$の選び方から、
    \begin{equation}
        \label{eq:intmu}
        \int \mu \, dx = 0
    \end{equation}
    であり、また$\mu(x) \geq 0$は$\rho_*(x) = 0$の場所、つまり$\varphi(x) \geq -C$の場所で成り立つ。

    よって、
    \begin{align*}
        V(\rho) - V(\rho_*) &= V(\rho_* + \mu) - V(\rho_*)\\
                            &= \int \left(- \frac{\gamma}{m-1} ((\rho_* + \mu)^m - \rho_*^{m}) + \mu \varphi \right) \, dx.
    \end{align*}
    ここで、関数 $s \mapsto s^m$ は $[0, \infty)$ 上で凸であるため、$s, t \geq 0$ に対して 
    $$
        (s + t)^m \geq s^m + ms^{m-1}t, \qquad s + t \geq 0
    $$ 
    が成り立ちます。この不等式を適用することで、さらに次のように簡略化できます:
    \begin{align*}
        V(\rho) - V(\rho_*) &\leq \int \left(- \frac{\gamma}{m-1} ((\rho_*^m + m \rho_*^{m-1} \mu) - \rho_*^{m}) + \mu \varphi \right) \, dx\\
                            &\leq \int \left(- \frac{m\gamma}{m-1} \rho_*^{m-1}\mu + \mu\varphi\right) \, dx\\
    \end{align*}

    $\rho_*$の式を利用することで、
    \begin{align*}
        \int \left(\frac{m\gamma}{m-1} \rho_*^{m-1}\mu + \mu\varphi\right) \, dx &= \int \left(-(C + \varphi)_+ \mu + \mu\varphi\right) \, dx \\
                                                                        &= \int \left(-(C + \varphi)_+ \mu + \mu \varphi + C \mu\right) \, dx \\
                                                                        &= \int \left(-(C + \varphi)_+ + \varphi + C\right)\mu \, dx \\  
    \end{align*}
    ただし、(\ref{eq:intmu})を利用する。

    ここで、以下を考える$:$

    \[
        - (C + \varphi)_+ + \varphi + C = - (C + \varphi)_- \begin{cases} = 0 & \text{if } \varphi > - C \\ 
                                                                    \leq 0 & \text{if } \varphi \leq - C \end{cases}.
    \]
    
    また、\(\mu(x) \geq 0\) は、\(\varphi(x) \geq C\) のとき成り立つ。
    したがって、
    \[
        V(\rho) - V(\rho_*) \leq \int (- (C + \varphi)_+ + \varphi + C) \mu \, dx \leq 0
    \]

    したがって、\(\rho_*\) は \(V\) の最大化点であることが証明された。
\end{proof}

以上から、多孔質勾配方程式において、$U^*$を満たす$\rho$は、
\begin{equation}
    \label{eq:PMErho}
    \rho_*(x) = \left( \frac{m-1}{m\gamma}(C + \varphi)_+ \right)^{\frac{1}{m-1}} 
\end{equation}
である。


\section{勾配上昇法}
\label{sect:勾配上昇法}
The back-and-forth methodで用いる勾配上昇法のために、実ヒルベルト空間$\mathcal{H}$の内積$\langle\cdot,\cdot\rangle_\mathcal{H}$とノルム$\|\cdot\|_\mathcal{H}$を使用する。

まず第一変分を定義する。
点$\varphi\in \mathcal{H}$に対して、有界線型写像$\delta F(\varphi): \mathcal{H} \to \mathbb{R}$が$F$の第1変分(フレシェ微分)であるとは、
    \[
        \lim_{\|h\|_{\mathcal{H}} \to 0} \frac{\|F(\varphi + h) - F(\varphi) - \delta F(\varphi)(h)\|_{\mathcal{H}}}{\|h\|_{\mathcal{H}}} = 0
    \]
が成り立つことをいう。

\begin{dfn}
    写像$\nabla_\mathcal{H} F: \mathcal{H} \to \mathcal{H}$が$\mathcal{H}$-勾配とは、
    \[
        \langle \nabla_\mathcal{H} F(\varphi), h \rangle_\mathcal{H} = \delta F(\varphi)(h)
    \]
    をすべての$(\varphi, h) \in \mathcal{H} \times \mathcal{H}$に対して満たすものを指す。
\end{dfn}

ここで、与えられた$\mathcal{H}$上の凹関数$J$に対し、勾配上昇法の反復式
\begin{equation}
    \label{eq:gradient ascent}
    \varphi_{k+1} = \varphi_k + \nabla_\mathcal{H} J(\varphi_k).
\end{equation}
を考える。
勾配上昇法式(\ref{eq:gradient ascent})は、次の変分形式でも同等に書くことができる。
\begin{equation}
    \label{eq:gradient ascent variational}
    \phi_{k+1} =  \underset{\varphi} {\operatorname{argmax}} J(\varphi_k) + \delta J(\varphi_k)(\varphi-\varphi_k) - \frac{1}{2}\|\varphi-\varphi_k\|_H^2.
\end{equation}
$J$を効率的に収束させるため、ノルムの取り方が重要である。
\begin{thm}
    \label{thm:chose norm}
    $J: \mathcal{H} \rightarrow \mathbb{R}$ を2階フレシェ微分可能な凹関数であり、maximizer $\varphi^*$ を持つとする。
    以下の条件がすべての $\varphi, h \in \mathcal{H}$ に対して成り立つとき,$J$ は「$1$-$smooth$」と呼ばれる:
    \begin{equation}
        \label{eq:1-smooth}
        \delta^2 J(\varphi)(h, h) \leq \|h\|_\mathcal{H}^2.
    \end{equation}
    $J$が$1$-$smooth$のとき、
    勾配上昇法 
    $$
        \varphi_{k+1} = \varphi_k + \nabla_\mathcal{H} J(\varphi_k)
    $$
    を初期点 $\varphi_0$ から開始すると、次の上昇特性(\ref{eq:ascent})を満たし、また収束率(\ref{eq:convergence})を持つ:
    \begin{equation}
        \label{eq:ascent}
        J(\varphi_{k+1}) \geq J(\varphi_k) + \frac{1}{2}\|\nabla_\mathcal{H} J(\varphi_k)\|_\mathcal{H}^2,
    \end{equation}
    \begin{equation}
        \label{eq:convergence}
        J(\varphi^*) - J(\varphi_k) \leq \frac{\|\varphi^* - \varphi_0\|_\mathcal{H}^2}{2k}.
    \end{equation}
\end{thm}
収束率である(\ref{eq:convergence})から、(\ref{eq:convergence})を満たすなるべく小さいノルムを取ることによって$J$を効率的に収束させることができる。



\section{ソボレフ(Sobolev)空間における勾配}
\label{sect:ソボレフ(Sobolev)空間における勾配}
$\Omega$を$\mathbb{R}^d$の開有界凸部分集合とする。
勾配上昇法は、Sobolev空間$H^1(\Omega)$に基づいたノルム$H$を使う。
このとき次のようにノルムを定義する。
\begin{equation}
    \label{eq:norm}
    \|h\|_H^2 = \int_{\Omega} \Theta_2 |\nabla h(x)|^2 + \Theta_1 |h(x)|^2 \, dx,
\end{equation}
ただし$\Theta_1 > 0$ and $\Theta_2 > 0$を定数とする。
次に勾配の計算方法を示す。
\begin{lem}[\cite{MR4238775}]
    \label{lem:H-gradient}
    $F = F(\varphi)$ がフレシェ微分可能な汎関数であり、任意の $\varphi$ に対して、任意の点 $h$ で評価される第1変分 $\delta F(\varphi)$ が関数 $f_{\varphi}$ に対する積分として表されるとします。
    つまり、
    \[
        \delta F(\varphi)(h) = \int_\Omega h(x) f_{\varphi}(x) dx
    \]

とします。ただし式 (\ref{eq:norm}) で $\| \cdot \|_H$ を定義している。
このとき、$F$ の $H$-勾配は次のように表される$:$
\[
    \nabla_H F(\varphi) = (\Theta_1 \mathrm{Id} - \Theta_2 \Delta)^{-1} f_{\varphi}.
\]
ここで、$\mathrm{Id}$ は単位作用素、$\Delta$ はラプラス作用素であり、ノイマン条件を伴って考える。
\end{lem}

Lemma\ref{lem:H-gradient}の意味は、$F$の$H$-勾配は$\delta F$を逆演算子$(\Theta_1 \text{Id} - \Theta_2\Delta)^{-1}$で事前調整することによって得られるということを表している。

つまり、$F$の$H$-勾配を計算する場合、$\delta F$を直接適用する代わりに、まず逆演算子$(\Theta_1 \text{Id} - \Theta_2\Delta)^{-1}$を$\delta F$に適用する。
この逆演算子を適用するプロセスは、「事前調整」と呼ばれ、収束特性を改善し、$H$-勾配の計算を効率化するために、$\delta F$を修正する役割を果たす。
すなわち、
\begin{equation*}
    f_\varphi = (\Theta_1 \text{Id} - \Theta_2 \Delta)\nabla_H F(\varphi)
\end{equation*}
を用いると、
\begin{align*}
    \delta F(\varphi)(h)   &= \int_\Omega h(x) f_\varphi(x) \, dx \\
                        &= \int_\Omega h(x)(\Theta_1 \text{Id} - \Theta_2 \Delta)\nabla_H F(\varphi)(x) \, dx \\
\end{align*}
となる。よって、
\begin{align}
    \label{eq:H-gradient}
    \delta F(\varphi)  &= (\Theta_1 \text{Id} - \Theta_2 \Delta)\nabla_H F(\varphi)(x),\notag\\
    \nabla_H F(\varphi) &= (\Theta_1 \text{Id} - \Theta_2 \Delta)^{-1} \delta F(\varphi).
\end{align}
となる。
よって、(\ref{eq:delta J}),(\ref{eq:delta I})を参照することにより、$\delta F(\varphi)$は$J, I$を用いて、以下のように表される。
\begin{equation}
    \label{eq:J-gradient}
    \nabla_H J(\varphi) = (\Theta_1 \text{Id} - \Theta_2 \Delta)^{-1} \left(\delta U^*(- \varphi) - T_{\varphi \#} \mu\right)
\end{equation}
\begin{equation}
    \label{eq:I-gradient}
    \nabla_H I(\varphi) = (\Theta_1 \text{Id} - \Theta_2 \Delta)^{-1}\left(\mu - T_{\psi \#} \delta U^* (- \psi^c)\right)
\end{equation}
