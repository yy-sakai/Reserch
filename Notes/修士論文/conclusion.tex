\chapter{おわりに} \label{ch:conclusion}
本論文の目的であった偏微分方程式の解法の一つであり特に非線形偏微分方程式の数値解法として有用なthe back-and-forth methodを用いて非線形である多孔質勾配方程式の解を求め、他の解法との比較を行うことができた。
back-and-forth methodは他の解放と比べて大きい時間ステップサイズであっても誤差を少なく近似することができた。
これは$\rho$を直接求めるのではなく、$\rho_*(x) = \delta U^*(- \varphi)$で$\rho$を再現するため、
不連続な$\rho$の台の境界でショックが起凝らないことと双対問題を解くことで制約がない状態で計算できるからである。

ただし、グリッドサイズに合わせ、適切な時間ステップ$\tau$とback-and-forth methodの反復を抜ける条件の$\|\delta U^*(- \varphi) - T_{\varphi \#} \mu \|_{L^1(\Omega)} < \varepsilon$を適切に取る必要がある。

今後の展望としては多孔質勾配方程式の$m$の値を大きくすること、2・3次元などで考えることなどが挙げられる。