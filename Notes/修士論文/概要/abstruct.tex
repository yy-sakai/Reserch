\documentclass[a4,11pt, twocolumn, dvipdfmx]{article}

\usepackage{amsthm, amsfonts, amsmath}
\usepackage{color}
\usepackage{graphicx}
\usepackage{listings,jvlisting}
\usepackage{here, latexsym, bm, ascmac, mathtools, multicol, tcolorbox, graphicx, comment, pgfplots}
\usepackage{algpseudocode}
\usepackage{algorithm}
\usepackage{tikz}
\usepackage{siunitx}
\usepackage{subcaption}
\usepackage{booktabs}
\usepackage{hyperref}
\usepackage{pxjahyper}
\usepackage{xcolor}
\usepackage{pgfplots}
\usepackage{mathtools}

\hypersetup{
    colorlinks=true,
    citecolor=blue,
    linkcolor=blue,
    urlcolor=orange,
}

\usetikzlibrary{intersections}
\usepgfplotslibrary{fillbetween}

%---------------------------------------------------
% ページの設定
%---------------------------------------------------
\setlength{\pdfpagewidth}{8.5in}  % DO NOT CHANGE THIS
\setlength{\pdfpageheight}{11in}  % DO NOT CHANGE THIS
\pagestyle{empty}
\setlength{\headheight}{0truemm}
\setlength{\parindent}{1zw}

\newtheorem{thm}{Theorem}[section]
\newtheorem{cor}{Corollary} [thm]
\newtheorem{lem}[thm]{Lemma}
\newtheorem{prop}[thm]{Proposition}
\theoremstyle{definition}
\newtheorem{dfn}{Definition}[section]
\newtheorem{ex}{Example}[section]




\lstset{
    basicstyle={\ttfamily},
    identifierstyle={\small},
    commentstyle={\smallitshape},
    keywordstyle={\small\bfseries},
    ndkeywordstyle={\small},
    stringstyle={\small\ttfamily},
    frame={tb},
    breaklines=true,
    columns=[l]{fullflexible},
    numbers=left,
    xrightmargin=0zw,
    xleftmargin=3zw,
    numberstyle={\scriptsize},
    stepnumber=1,
    numbersep=1zw,
    lineskip=-0.5ex
}


%---------------------------
%1段組
\begin{document}
\twocolumn
[
\begin{center}
{\huge PDE solver using optimal transport}
\end{center}
\begin{flushright}
\begin{tabular}{rr}
{\Large \ } &\ \\
2215011016&Yukito Sakai\\
{\Large \ } &\ \\
\end{tabular}
\end{flushright}
]


\section{Introduction}
The back-and-forth method (BFM) is presented as a new solution for partial differential equations\cite{MR4238775}. 
The back-and-forth method (BFM) was originally developed as a solution for the optimal transport problem. 
However, by extending its application to PDEs through the utilization of optimal transport, 
BFM becomes a powerful tool for solving PDEs.
This solution method is particularly effective for non-linear partial differential equations and is faster than previous methods and does not require stability conditions, 
allowing a wider range of problems to be solved.
This study compares the BFM-based solver with other solvers for the $m=2$ porous medium equation and shows that under certain conditions, the BFM solver can solve the non-linear porous media equation more efficiently.
\section{Background}
In this paper, we assume that $\Omega$ is a convex and compact subset of $\mathbb{R}^n$. 
We denote the set of probability measures on $\Omega$, which are non-negative measures with mass 1, as $\mathcal{P} (\Omega)$ . 
Additionally, we represent the space of continuous functions on $\Omega$ as $C(\Omega)$. 
For simplicity, we consider non-negative \(L^1(\mathbb{R}^n)\) functions with an integral equal to 1.

The Porous Medium Equation (PME) is expressed as
\[
    \frac{\partial \rho}{\partial t} = \gamma \Delta(\rho^m)
\]
for a fixed \(m > 1\). 
This partial differential equation can be expressed as a Wasserstein gradient flow based on the energy function
\[
    U(\rho) = \frac{\gamma}{m-1} \int_{\Omega} \rho^m \,dx.
\]
This allows the JKO scheme to find approximate solutions to the porous medium equation by discretizing in time and performing the following iterations based on the variational principle:
\[
    \min_{\rho \in P} \left(U(\rho) + \frac{1}{2\tau} W_2^2(\rho, \mu)\right)
\]
Here, Kantorovich's dual formula for the 2-Wasserstein distance is expressed as
\[
    \frac{1}{2\tau} W_2^2(\rho, \mu) = \sup_{(\varphi, \psi) \in C} \left(\int_{\Omega} \varphi \,d\rho + \int_{\Omega} \psi \,d\mu\right)
\]
where \(\mathcal{C}\) is defined by 
\(\mathcal{C} = \{(\varphi, \psi) \in C(\Omega) \times C(\Omega) \,|\, \varphi(x) + \psi(y) \leq \frac{1}{2\tau} |x-y|^2\}\) 
and \(\tau\) represents the time step within the scheme. 
Therefore, the above minimization problem is represented as follows:
\begin{align*}
    &\min_{\rho \in P} \left(U(\rho) + \frac{1}{2\tau} W_2^2(\rho, \mu)\right)\\
     &= \sup_{(\varphi, \psi) \in C} \left(\int_{\Omega} \psi \,d\mu - U^*(-\varphi)\right)
\end{align*}
Here, 
\(U^*(-\varphi) = \sup_{\rho \in P} \int_{\Omega} \left(-\frac{\gamma}{m-1} \rho^m + \rho\varphi\right) \,dx\) 
is the Legendre-Fenchel transform of \(U\) with respect to \(-\varphi\). 
Note that the Legendre-Fenchel transform of $f$ is defined as
\begin{equation*}
    f^*(p) := \sup_x \{p \cdot x - f(x) \}.
\end{equation*}
where $f \in C$ be a convex and differentiable function. 

Furthermore, by using the \(c\)-transform defined by 
\(\psi^c(x) = \inf_{y \in \Omega} \left(\frac{1}{2\tau} |x-y|^2 - \psi(y)\right)\), 
the minimization problem has the following duality:
\begin{align*}
    &\min_{\rho \in P} \left(U(\rho) + \frac{1}{2\tau} W_2^2(\rho, \mu)\right) \\
    &= \sup\left(\int_{\Omega} \varphi^c \,d\mu - U^*(-\varphi)\right) =: \sup J\\
    &= \sup\left(\int_{\Omega} \psi \,d\mu - U^*(-\psi^c)\right) =: \sup I
\end{align*}
Also, the solution to the generalized optimal transport problem \(\rho_*\) has the following properties:
\[
    \rho_*(x) = \delta U^*(-\varphi_*), \,\, \varphi_* \in U(\rho_*), \,\, \rho_*(x) = T_{\varphi\#} \mu
\]
where \(T_{\varphi\#} \mu\) is the pushforward measure with \(T\) being the map transporting from \(\mu\) to \(\rho\) ($T_{\varphi\#} \mu = \rho$).


\section{The back-and-forth method and JKO scheme }
BFM is an algorithm that iteratively performs gradient ascent updates in the \(\varphi\) space for the functional \(J(\varphi) \) and in the \(\psi\)-space for the functional \(I(\psi)\). 
Also, by swapping $\varphi$ and $\psi$ using the $c$-transform, it is possible to go back and forth between $J$ and $I$ and find the solution $\varphi_*$ of the BFM while maintaining symmetry.
The algorithm is based on the following ideas:
\begin{enumerate}
    \item Back-and-forth Scheme: Iteratively alternate between gradient ascent steps in $J$ and $I$.
    \item For the \(H^1\)-type norm \(H\) in Sobolev space, the gradient ascent steps are given by:
        \begin{equation*}
            \nabla_H J(\varphi) = (\Theta_1 \text{Id} - \Theta_2 \Delta)^{-1} \left[\delta U^*(- \varphi)-  T_{\varphi \#} \mu \right], \\
        \end{equation*}
        \begin{equation*}
            \nabla_H I(\psi) = (\Theta_1 \text{Id} - \Theta_2 \Delta)^{-1} \left[\mu - T_{\psi \#} (\delta U^*(\psi^c))\right].
        \end{equation*}
        The appropriate choices for \(\Theta_1\) and \(\Theta_2\) can be found in \cite{MR4238775}.
\end{enumerate}

\begin{algorithm}[tb]
    \caption{The back-and-forth scheme for solving $J(\varphi)$ and $I(\psi)$}
    \label{al:baf-method}
    \begin{algorithmic}
    \State \textbf{Input:} $\mu$ and $\varphi_0$, iterate:
    \State{
        \begin{align*}
            \varphi_{k + \frac{1}{2}} &= \varphi_k + \nabla_H J(\varphi_k)\\
            \psi_{k + \frac{1}{2}} &= (\varphi_{k + \frac{1}{2}})^c\\
            \psi_{k + 1} &= \psi_{k + \frac{1}{2}} + \nabla_H I(\psi_{k + \frac{1}{2}})\\
            \varphi_{k + 1} &= (\psi_{k + 1})^c
        \end{align*}
    }
    \State \textbf{return} $\varphi_*$
    \end{algorithmic}
\end{algorithm}


Furthermore, utilizing the obtained solution \(\varphi_*\) from BFM, 
the minimization problem for \(\rho_*\) is solved using $\rho_*(x) = \delta U^*(- \varphi)$.

\section{Experiments and Results}
The porous medium equation is solved using two different methods: 
The BBR scheme proposed by Berger, Brezis, Rogers \cite{M2AN_1979__13_4_297_0}, and the back-and-forth method explained in this paper. 
A comparison is made in terms of computational speed and errors with the Barenblatt solution. In all implementations, 
the parameters for the exact solution, Barenblatt solution, are set as $M = 0.5, h_0 = 15, \gamma = 10^{-3}$.

Verify under the conditions of a grid size of $512$ and $\varepsilon = 10^{-3}$. 
Graphs for multiple time step sizes $\tau$ at times $t = t_0, t_0 + 0.4, t_0 + 0.8, t_0 + 2.0$ are shown in Figures \ref{img:baf_m=2} and \ref{img:bbr_m=2}.

\begin{figure}[htbp]
    \centering
    \begin{tabular}{cc}
        \begin{minipage}[t]{0.45\hsize}
            \centering
            \includegraphics[width=\linewidth]{../../../images/baf_tau/t=0.png}
            \subcaption{$t=0$}
            \label{img:baf_0}
        \end{minipage} &
        \begin{minipage}[t]{0.45\hsize}
            \centering
            \includegraphics[width=\linewidth]{../../../images/baf_tau/t=0.4.png}
            \subcaption{$t = 0.4$}
            \label{img:baf_1}
        \end{minipage} \\
        
        \begin{minipage}[t]{0.45\hsize}
            \centering
            \includegraphics[width=\linewidth]{../../../images/baf_tau/t=0.8.png}
            \subcaption{$t = 0.8$}
            \label{img:baf_2}
        \end{minipage} &
        \begin{minipage}[t]{0.45\hsize}
            \centering
            \includegraphics[width=\linewidth]{../../../images/baf_tau/t=2.png}
            \subcaption{$t = 2$}
            \label{img:baf_3}
        \end{minipage}
    \end{tabular}
    \caption{Comparison between the numerical solution using the back-and-forth method and the exact solution Barenblatt solution.}
    \label{img:baf_m=2}
\end{figure}

\begin{figure}[htbp]
    \begin{tabular}{cc}
        \begin{minipage}[t]{0.45\hsize}
            \centering
            \includegraphics[width=\linewidth]{../../../images/bbr_tau/t=0.png}
            \subcaption{$t=0$}
            \label{img:bbr_0}
        \end{minipage} &
        \begin{minipage}[t]{0.45\hsize}
            \centering
            \includegraphics[width=\linewidth]{../../../images/bbr_tau/t=0.4.png}
            \subcaption{$t = 0.4$}
            \label{img:bbr_1}
        \end{minipage} \\

        \begin{minipage}[t]{0.45\hsize}
            \centering
            \includegraphics[width=\linewidth]{../../../images/bbr_tau/t=0.8.png}
            \subcaption{$t = 0.8$}
            \label{img:bbr_2}
        \end{minipage} &
        \begin{minipage}[t]{0.45\hsize}
            \centering
            \includegraphics[width=\linewidth]{../../../images/bbr_tau/t=2.png}
            \subcaption{$t = 2$}
            \label{img:bbr_3}
        \end{minipage} \\
    \end{tabular}
    \caption{Comparison between the numerical solution using BBR scheme and the exact solution Barenblatt solution.}
    \label{img:bbr_m=2}
\end{figure}
The BBR scheme, being an implicit method, allows for larger time step approximations compared to the explicit Euler method. 
However, as observed in Figure \ref{img:bbr_m=2}, there are discontinuities at the boundary of the support $(\text{supp} \, \rho = \{ x \in \Omega \,| \, \rho(x) > 0\})$ of the evolving function $\rho$, leading to constraints. 
Therefore, without choosing an appropriate size for the time step $\tau$, the presence of abrupt changes in the gradient $\nabla\rho$ can result in shocks at the boundary.

On the other hand, the back-and-forth method can approximate the solution even with larger $\tau$, avoiding shocks at the discontinuous boundary of $\rho$'s support, as evident in Figure \ref{img:baf_m=2}. 
This is attributed to the indirect computation of $\rho$ by reconstructing it through $\rho_*(x) = \delta U^*(- \varphi)$.

\section{Conclusion}
The main objective of this paper was to explore solutions to nonlinear porous gradient equations, particularly employing the back-and-forth method as a numerical approach for nonlinear partial differential equations. 
I also compared the results with other solution methods.
The back-and-forth method demonstrated the ability to approximate the solution with minimal errors even for large time step sizes compared to other methods. 
This is attributed to the reconstruction of \(\rho\) through \(\rho_*(x) = \delta U^*(- \varphi)\), avoiding shocks at the discontinuous boundary of \(\rho\), and the capability to compute without constraints by solving the dual problem.

However, it is essential to choose appropriate time steps, $\tau$, and conditions to exit the iterations of the back-and-forth method, such as $\|\delta U^*(- \varphi) - T_{\varphi \#} \mu \|_{L^1(\Omega)} < \varepsilon$, in alignment with the grid size.

For future prospects, increasing the value of $m$ in the porous medium gradient equation and exploring scenarios in two or three dimensions are potential directions to consider.
\bibliographystyle{plain}
\bibliography{../ref}  % BibTeXファイルの指定

\end{document}