%%%%%%%%%% Beamerの初期設定 %%%%%%%%%%%
% beamerを使用する初期設定
\documentclass[dvipdfmx, 12pt]{beamer}

% 使用するパッケージ 
\usepackage{amsthm, amsfonts, amsmath}
\usepackage{color}
\usepackage{graphicx}
\usepackage{listings,jvlisting}
\usepackage{here, latexsym, bm, ascmac, mathtools, multirow, multicol, tcolorbox, graphicx, comment, pgfplots}
\usepackage{algpseudocode}
\usepackage{algorithm}
\usepackage{tikz}
\usepackage{siunitx}
\usepackage{subcaption}
\usepackage{booktabs}
\usepackage{pxjahyper}
\usepackage{xcolor}
\usepackage{pgfplots}
\usepackage{mathtools}
%デザインの選択 (省略可)
\usetheme{Boadilla}
%\usecolortheme[RGB={0, 100, 125}]{structure}
%%%%%%%%%%%%%%%%%%%%%%%%%%%%%%%%%%%%%
\usepgfplotslibrary{fillbetween}
\setbeamertemplate{footline}{}

\usetikzlibrary{intersections}
\usepgfplotslibrary{fillbetween}
\setbeamertemplate{footline}[frame number]
\beamertemplatetextbibitems


\hypersetup{
    colorlinks=true,
    citecolor=blue,
    linkcolor=blue,
    urlcolor=orange,
}


%%%%%%%%%% Beamerの基本的なコード %%%%%%%%%%
% 属性
\title{最適輸送法を用いた\\偏微分方程式のソルバー}
\subtitle{PDE solver using optimal transport}
\author[坂井 幸人]{坂井 幸人}
\institute[数物科学専攻 2年]{数物科学専攻 2年}
\date{\today}

% スライドの始まり
\begin{document}

% タイトルページ
\frame{\maketitle}


% 目次ページ
\begin{frame}{目次}
    \tableofcontents
\end{frame}


\section{研究目的}
\begin{frame}{研究背景}
    \begin{itemize}
        \item $\,\,\,\, U: $内部エネルギー
        \item $\nabla \varphi: $ $U$によって生成される圧力勾配
        \item $\,\,\,\,\, \, \rho: $質量密度(確率測度) $(\rho \ge 0, \|\rho\|_{L_1} = 1)$
    \end{itemize}

    \begin{align*}
        \begin{split}
            \partial_t \rho - \nabla \cdot (\rho \nabla \varphi) = 0, \\
            \varphi = \delta U(\rho).
        \end{split}
    \end{align*}


    \vspace\baselineskip 
    $\implies $ 一般的に剛性かつ非線形であるため数値計算が困難

    \vspace\baselineskip 
    $\implies $ 効率的かつ正確にシュミレートしたい
    

\end{frame}

\begin{frame}{研究背景}
    \begin{align*}
        \begin{split}
            \partial_t \rho - \nabla \cdot (\rho \nabla \varphi) = 0, \\
            \varphi = \delta U(\rho).
        \end{split}
    \end{align*}
    \[
        U(\rho) = \frac{\gamma}{m-1} \int_{\Omega} \rho^m \,dx
    \]

    \[
        \frac{\partial \rho}{\partial t} = \gamma \Delta(\rho^m) \quad (m > 1, \gamma > 0)
    \]
\end{frame}

\begin{frame}{研究目的}
    The back-and-forth method\cite{MR4238775}を用いた偏微分方程式のソルバーが多孔質勾配方程式(porous medium equation)
    \[
        \frac{\partial \rho}{\partial t} = \gamma \Delta(\rho^m) \quad (m > 1, \gamma > 0)
    \]
    を他のソルバーより効率的に、幅広い条件で解けることを示す。
\end{frame}

\section{前提知識}
\begin{frame}{前提知識}
    \begin{block}{最適輸送問題(Mongeの問題(1871))}
        ある砂山から砂山(測度$\mu$)と同じ体積の穴(測度$\nu$)に砂を運ぶ(写像$T$).
        輸送にかかるコストは重さと移動距離に依存する時,コストを最小にする方法を求めよ.
    \end{block}
    \begin{figure}[htbp]
        \begin{center}
            \includegraphics[width=120mm]{../images/transport_map1.JPG}
            \caption{transport map}
        \end{center}
    \end{figure}
\end{frame}

\begin{frame}{}
    多孔質勾配方程式について考える:
    \[
        \frac{\partial \rho}{\partial t} = \gamma \Delta(\rho^m) \quad (m > 1, \gamma > 0)
    \]
    ただし内部エネルギー$U$は
    \[
        U(\rho) = \frac{\gamma}{m-1} \int \rho^m \,dx
    \]
    である。

    \vspace\baselineskip 
    $\implies $時間離散化によって近似解$\rho$を求めたい

    \vspace\baselineskip 
    $\implies $多孔質勾配方程式をWasserstein距離を用いた
    
    Wasserstein勾配流として表現する

\end{frame}


\begin{frame}{}
    \[
    \frac{\partial \rho}{\partial t} = \gamma \Delta(\rho^m) \quad (m > 1, \gamma > 0)
    \]
    \vspace\baselineskip 
    \begin{itemize}
        \item $\tau > 0$
        \item $t_n := n \tau$
    \end{itemize}
    \vspace\baselineskip 
    最小化問題
    \[
        \rho_n \in \underset{{\substack{\rho \ge 0 \\ \|\rho\|_{L^1} = 1}}}{\operatorname{argmin}}\, \left(U(\rho) + \frac{1}{2\tau} W_2^2(\rho, \rho_{n-1})\right)
    \]
    2-Wasserstein距離
    \[
        \frac{1}{2\tau} W_2^2(\rho, \mu) = \sup_{\varphi(x) + \psi(y) \leq \frac{1}{2 \tau} |x - y|^2} \left( \int \varphi d\rho + \int \psi d\mu \right)
    \]
\end{frame}

\begin{frame}{}
    \(c\)-変換
    \[
        \psi^c(x) := \inf_{y} \left(\frac{1}{2\tau} |x-y|^2 - \psi(y)\right)
    \]
    を定義することで、
    \[
        \varphi(x) + \psi(y) \leq \frac{1}{2 \tau} |x - y|^2
    \]
    の制約をなくし、一つの関数で表すことができる。

    
\end{frame}


\begin{frame}{}

    内部エネルギー$U$のLegendre変換
    \[
      U^*(\varphi) := \sup_{{\substack{\rho \ge 0 \\ \|\rho\|_{L^1} = 1}}} \left(\int \varphi \, d\rho - U(\rho) \right)
    \]
    \vspace\baselineskip
    を用いることで、最小化問題は以下の双対性を持つ:
    \begin{align*}
        \min_{{\substack{\rho \ge 0 \\ \|\rho\|_{L^1} = 1}}} \left(U(\rho) + \frac{1}{2\tau} W_2^2(\rho, \mu)\right) 
        &= \sup\left(\int \varphi^c \,d\mu - U^*(-\varphi)\right) =: \sup J\\
        &= \sup\left(\int \psi \,d\mu - U^*(-\psi^c)\right) =: \sup I
    \end{align*}    
\end{frame}


\section{The back-and-forth method}
\begin{frame}{The back-and-forth method}
    \begin{align*}
         J(\varphi) &:= \int \varphi^c \,d\mu - U^*(-\varphi)  \\
        \nabla_H J(\varphi) &= (\Theta_1 \text{Id} - \Theta_2 \Delta)^{-1} \left[\delta U^*(- \varphi)-  T_{\varphi \#} \mu \right]\\
        I(\psi)&:= \int \psi \,d\mu - U^*(-\psi^c)\\
        \nabla_H I(\psi) &= (\Theta_1 \text{Id} - \Theta_2 \Delta)^{-1} \left[\mu - T_{\psi \#} (\delta U^*(\psi^c))\right]\\
        \psi &\leftarrow \varphi^c\\
        \varphi &\leftarrow  \psi^c
    \end{align*}
\end{frame}


\begin{comment}
\end{comment}


% ブロック環境




%Algorithm
\begin{frame}{Algorithm}
    \begin{block}{Algorithm: the back-and-forth method}
        Input $\mu$ and $\varphi_0$, iterate until $\|\delta U^*(- \varphi) - T_{\varphi \#} \mu \|_{L^1(\Omega)} <  \forall \varepsilon$:
        \begin{align*}
            \varphi_{k + \frac{1}{2}} &= \varphi_k + \nabla_H J(\varphi_k)\\
            \psi_{k + \frac{1}{2}} &= (\varphi_{k + \frac{1}{2}})^c\\
            \psi_{k + 1} &= \psi_{k + \frac{1}{2}} + \nabla_H I(\psi_{k + \frac{1}{2}})\\
            \varphi_{k + 1} &= (\psi_{k + 1})^c
        \end{align*}
    \end{block}
    \[
        \nabla_H J(\varphi) = (\Theta_1 \text{Id} - \Theta_2 \Delta)^{-1} \left[\delta U^*(- \varphi)-  T_{\varphi \#} \mu \right]
    \]
    \[
        \nabla_H I(\psi) = (\Theta_1 \text{Id} - \Theta_2 \Delta)^{-1} \left[\mu - T_{\psi \#} (\delta U^*(\psi^c))\right]
    \]\\

    \begin{itemize}
        \item $H:$重み付きSobolev空間$H^1$に基づくノルム
    \end{itemize}
    $\implies $BFMによって最適な関数$\varphi_{n*}$を求める

\end{frame}

\begin{frame}{pushforward measure}
    \[
        \nabla_H J(\varphi) = (\Theta_1 \text{Id} - \Theta_2 \Delta)^{-1} \left[\delta U^*(- \varphi)-  T_{\varphi \#} \mu \right]
    \]
    \[
        \nabla_H I(\psi) = (\Theta_1 \text{Id} - \Theta_2 \Delta)^{-1} \left[\mu - T_{\psi \#} (\delta U^*(\psi^c))\right]
    \]
    

    \(T_{\varphi\#} \mu\)は\(\mu\)から\(\rho\)へ輸送する写像\(T\)によるpushforward measureである(\(T_{\varphi\#} \mu = \rho\))。
    \begin{figure}[htbp]
        \begin{center}
            \includegraphics[width=100mm]{../images/transport_map2.JPG}
            \caption{pushforward measure}
        \end{center}
    \end{figure}
\end{frame}

\begin{frame}{BFMの解から最小化問題の解の復元}
    多孔質勾配方程式
    \[
        \frac{\partial \rho}{\partial t} = \gamma \Delta(\rho^m) \quad (m > 1, \gamma > 0)
    \]
    の最小化問題
    \[
        \rho_n \in \underset{{\substack{\rho \ge 0 \\ \|\rho\|_{L^1} = 1}}}{\operatorname{argmin}}\, \left(U(\rho) + \frac{1}{2\tau} W_2^2(\rho, \rho_{n-1})\right)
    \]
    は、BFMから得られた解\(\varphi_{n*}\)を利用して、
    \[
        \rho_{n*}(x) = \delta U^*(- \varphi_{n*}) = \left( \frac{m - 1}{m \gamma} \max(- \varphi_{n*}, 0)\right)^{\frac{1}{m-1}}
    \]
    で求めることができる
\end{frame}


\section{実装と結果}
\begin{frame}{実装}
    多孔質勾配方程式$(m=2, \gamma = 10^{-3})$
    \[
        \frac{\partial \rho}{\partial t} = \gamma \Delta(\rho^m)
    \]
    \begin{itemize}
        \item the back-and-forth methodを用いたソルバー
        \item Berger, Brezis, Rogers\cite{M2AN_1979__13_4_297_0}によって提案されたBBRスキームのソルバー
        \item Euler 陽解法
    \end{itemize}
    を用いて計算する

\vspace\baselineskip
    ただし厳密解はBarenblatt solutionを用いる
    \[
        \rho(t,x)= \max\left(\frac{1}{t^{\frac{1}{3}}}\left(C  - \frac{1}{12} \frac{ |x|^2}{t^{\frac{2}{3}}} \right), 0\right)
    \]

\end{frame}


\begin{frame}{The back-and-forth methodの数値解}
    \begin{figure}[htbp]
        \centering
        \begin{tabular}{cc}
            \begin{minipage}[t]{0.35\textwidth}
                \centering
                \includegraphics[keepaspectratio, width=\textwidth]{../../../images/baf_tau/t=0.png}
            \end{minipage} &
            \begin{minipage}[t]{0.35\textwidth}
                \centering
                \includegraphics[keepaspectratio, width=\textwidth]{../../../images/baf_tau/t=0.4.png}
            \end{minipage} \\
            \begin{minipage}[t]{0.35\textwidth}
                \centering
                \includegraphics[keepaspectratio, width=\textwidth]{../../../images/baf_tau/t=0.8.png}
            \end{minipage} &
            \begin{minipage}[t]{0.35\textwidth}
                \centering
                \includegraphics[keepaspectratio, width=\textwidth]{../../../images/baf_tau/t=2.png}
            \end{minipage}
        \end{tabular}
        \caption{時刻\(t = 0, 0.4, 0.8, 2.0\)における時間ステップ\(\tau\)に対するグラフ}
    \end{figure}
\end{frame}

\begin{frame}{BBRスキームの数値解}
    \begin{figure}[htbp]
        \centering
        \begin{tabular}{cc}
            \begin{minipage}[t]{0.35\textwidth}
                \centering
                \includegraphics[keepaspectratio, width=\textwidth]{../../../images/bbr_tau/t=0.png}
            \end{minipage} &
            \begin{minipage}[t]{0.35\textwidth}
                \centering
                \includegraphics[keepaspectratio, width=\textwidth]{../../../images/bbr_tau/t=0.4.png}
            \end{minipage} \\
            \begin{minipage}[t]{0.35\textwidth}
                \centering
                \includegraphics[keepaspectratio, width=\textwidth]{../../../images/bbr_tau/t=0.8.png}
            \end{minipage} &
            \begin{minipage}[t]{0.35\textwidth}
                \centering
                \includegraphics[keepaspectratio, width=\textwidth]{../../../images/bbr_tau/t=2.png}
            \end{minipage}
        \end{tabular}
        \caption{時刻\(t = 0, 0.4, 0.8, 2.0\)における時間ステップ\(\tau\)に対するグラフ}
    \end{figure}
\end{frame}

\begin{frame}{計算速度と誤差比較}
    \begin{table}[hbtp]
        \centering
        \caption{Errors and calculation times(grid size 512, $\varepsilon = 10^{-3}$)}
        \label{tab:grid512_1}
        \resizebox{\textwidth}{!}{%
            \begin{tabular}{llllll} 
                \hline
                \multirow{2}{*}{$\tau$} & \multirow{2}{*}{$N_\tau$} & \multicolumn{2}{c}{the back-and-forth method} & \multicolumn{2}{c}{BBR scheme} \\
                \cline{3-6}
                & &  Error & Time(s) & Error & Time(s) \\
                \hline \hline  
                0.4  & 5 & \num{8.68e-02} & 0.069 & \num{7.19e-01} &  0.000342 \\  
                0.2  & 10 & \num{5.74e-02} & 0.139 & \num{6.37e-01} &  0.00067 \\ 
                0.1  & 20 & \num{3.61e-02} & 0.162 & \num{4.97e-01} &  0.00131 \\ 
                0.05  & 40 & \num{2.15e-02} & 0.17 & \num{2.98e-01} &  0.00275 \\
                0.025  & 80 & \num{1.25e-02} & 0.198 & \num{1.16e-01} &  0.00495 \\
                0.0125  & 160 & \num{7.66e-03} & 0.214 & \num{3.77e-02} &  0.0102 \\ 
                0.00625  & 320 & \num{5.06e-03} & 0.245 & \num{1.02e-02} &  0.0202 \\ 
                0.0001  & 20000 & \num{1.81e-02} & 3.7 & \num{1.25e-04} &  1.54 \\ 
                \hline 
            \end{tabular} 
        }
    \end{table}
    
\end{frame}
\begin{comment}
\begin{frame}{計算速度と誤差比較2}
    \begin{table}[hbtp]
        \centering
        \caption{Errors and calculation times(grid size 4000, $\varepsilon = 10^{-6}$)}
        \label{tab:grid512_1}
        \resizebox{\textwidth}{!}{%
            \begin{tabular}{llllll} 
                \hline
                \multirow{2}{*}{$\tau$} & \multirow{2}{*}{$N_\tau$} & \multicolumn{2}{c}{the back-and-forth method} & \multicolumn{2}{c}{BBR scheme} \\
                \cline{3-6}
                & &  Error & Time(s) & Error & Time(s) \\
                \hline \hline  
                0.4  & 5  & \num{0.07922714669663385} & 0.345 & \num{8.12e-01} & 0.00192\\  
                0.2  & 10 &\num{5.73e-02} & 0.675 & \num{8.12e-01} & 0.00391\\ 
                0.1  & 20 & \num{3.59e-02} & 1.33 &\num{7.81e-01} & 0.00778\\ 
                0.05  & 40  & \num{2.14e-02} & 2.65 &\num{7.10e-01} & 0.0156\\
                0.025  & 80  & \num{1.23e-02} & 5.29 &\num{5.81e-01} & 0.0308\\
                0.0125  & 160  & \num{6.82e-03} & 10.6 &\num{4.02e-01} & 0.0647\\ 
                0.00625  & 320  & \num{3.67e-03} & 21.2 &\num{1.91e-01} & 0.127 \\ 
                0.0001  & 20000  & \num{5.75e-04} & 122 &\num{5.00e-05} & 7.99\\ 
                \hline 
            \end{tabular} 
        }
    \end{table}
    
\end{frame}
\end{comment}



\begin{frame}{まとめ}
    \begin{enumerate}
        \item 大きい$\tau$でも近似ができる
        \begin{itemize}
            \item $\rho_*(x) = \delta U^*(- \varphi_*)$で$\rho$を再現する
            \item $c$-変換によって制約条件がない        
        \end{itemize}
        \item grid sizeに対して適切な$\tau$と$\varepsilon$を取る必要がある
        \begin{itemize}
            \item BFMループの終了条件 $\|\delta U^*(- \varphi) - T_{\varphi \#} \mu \|_{L^1(\Omega)} <  \forall \varepsilon$を満たさないため
        \end{itemize}
    \end{enumerate}
    \vspace{\baselineskip} 

    今後の展望としては多孔質勾配方程式の$m$の値を大きくすること、2・3次元などで考えることなどが挙げられる
    
\end{frame}

\begin{frame}{参考文献}
    \bibliographystyle{plain}
    \bibliography{../ref}  % BibTeXファイルの指定
\end{frame}

\end{document}




