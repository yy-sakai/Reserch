\documentclass{jsarticle}
%
\usepackage{type1cm}
\usepackage{amsthm}
\usepackage{color}

\usepackage[dvipdfmx]{graphicx}
\usepackage{listings,jvlisting}
\usepackage{float}
\usepackage{here, amsmath, latexsym, amssymb, bm, ascmac, mathtools, multicol, tcolorbox, subfig, graphicx, comment, pgfplots}
%

% 「%」は以降の内容を「改行コードも含めて」無視するコマンド
\usepackage[%
 dvipdfmx,% 欧文ではコメントアウトする
 setpagesize=false,%
 bookmarks=true,%
 bookmarksdepth=tocdepth,%
 bookmarksnumbered=true,%
 colorlinks=true,%
 citecolor=green,%
 urlcolor=magenta,%
 linkcolor=blue,%
 pdftitle={},%
 pdfsubject={},%
 pdfauthor={},%
 pdfkeywords={}%
]{hyperref}
% PDFのしおり機能の日本語文字化けを防ぐ((u)pLaTeXのときのみかく)
\usepackage{pxjahyper}


\newtheorem{thm}{Theorem}[section]
\newtheorem{cor}{Corollary} [thm]
\newtheorem{lem}[thm]{Lemma}
\newtheorem{prop}[thm]{Proposition}
\theoremstyle{definition}
\newtheorem{dfn}{Definition}[section]
\newtheorem{ex}{Example}[section]
\newtheorem{rem}{Remark}[section]

\renewcommand{\labelenumi}{(\roman{enumi})}

%
\lstset{
  basicstyle={\ttfamily},
  identifierstyle={\small},
  commentstyle={\smallitshape},
  keywordstyle={\small\bfseries},
  ndkeywordstyle={\small},
  stringstyle={\small\ttfamily},
  frame={tb},
  breaklines=true,
  columns=[l]{fullflexible},
  numbers=left,
  xrightmargin=0zw,
  xleftmargin=3zw,
  numberstyle={\scriptsize},
  stepnumber=1,
  numbersep=1zw,
  lineskip=-0.5ex
}

\title{Note  "A fast approach to optimal transport: the back-and-forth method"}

\author{坂井幸人}

\date{\today}

\begin{document}
\maketitle

\section{Introduction}
狭凸関数の最適輸送問題を効率よく解く解法。

$n$個の点を持つ離散グリッド上の最適輸送マップを$O(n log(n))$の操作と$O(n)$のストレージで計算。

一連のソースからシンクへの質量輸送の最も費用効率の良い方法を見つけることを目的としている。

この理論は、1942年にKantorovichによって近代化され、最適輸送と線形計画法の間に重要な関連性を見出しました。近年、最適輸送への関心は爆発的に高まっています。

これは、2次コスト最適輸送問題と、統計力学や流体力学で発生する偏微分方程式(PDE)の多様なクラスとの深い関連性が発見されたことに一因があります。

最適マップの計算は非常に困難な課題でした。私たちの知る限りでは、これまでに最適輸送問題を解決するための既知のすべての方法は、問題サイズに関して線形スケーリングしない[3,5]、最適マップを正確に計算できない[10]、または限られた確率密度のクラスにしか適用できない[4]という問題があります。\par
\vspace\baselineskip 

本論文の目的は、効率的かつ正確なアルゴリズムを提供し、最適輸送マップを計算することです。

新しい最適輸送問題の解法である「the back-and-forth method」を提案します。
確定的凸コスト関数を持つ二つの確率密度関数$\mu$と$\nu$を、$n$点のグリッドに離散化した場合、前後法は一回の繰り返しあたり$O(n log(n))$の計算時間と$O(n)$のストレージスペースを使用して最適マップを計算します。
全ての実験で$\epsilon$の精度を達成するために必要なイテレーション数は、$O(max( \|\mu\|_\infty, \|\nu\|_\infty) log(1/\epsilon))$のように成長します。

この方法は非常に速く収束します。
\color{red}
特筆すべきことに、他の多くの方法とは異なり、前後法は確率密度の正の下限を必要としません。
そのため、大部分の領域でゼロになる密度であっても最適輸送問題を解くことができます。
前後法の効率性と柔軟性により、以前の手法では扱えなかった最適輸送問題を解くことができます。

\color{black}


\subsection{Overall approach}

\begin{thm}[カントロヴィッチ双対問題]
  二つの確率測度$\nu,\mu$と、$x$から$y$への輸送にかかるコスト関数$c(x, y)$が与えられたとする。
  $\nu$を$\mu$に輸送する最適なコストCは、Kantorovich双対問題の値によって与えられる:

  \begin{equation*}
      C = \sup_{\phi, \psi} \int \phi  d\nu + \int \psi  d\mu 
  \end{equation*}

  ただし,Kantorovichポテンシャル$\phi(y), \psi(x)$は, 以下の不等式制約を満たす$2$つのスカラー関数である。

  \begin{equation}
      \phi(y) + \psi(x) \le c(x, y).
  \end{equation}
  \vspace{\baselineskip} 
  最適なコスト$C$は輸送する砂の体積を最大化させる問題として扱うことができる.
\end{thm}
$\mu \to \nu$への最適マップが存在すると、双対問題の最大値 $\phi_*(y), \psi_*(x)$が復元できる.

そのとき, 
\begin{equation*}
  \phi_*(y) + \psi_*(x) = c(x, y).
\end{equation*}

したがって、最適なマップを計算するには、双対問題を解くだけで十分。

$\phi_*$と$\psi_*$が双対問題の最大値をとるとき、以下の関係が成立することに基づいている。
\begin{equation*}
  \phi_*(y) = \psi_*^c(y) = \inf\{c(x, y) - \psi_*(x) \}
\end{equation*}
\begin{equation*}
  \psi_*(x) = \phi_*^c(x) = \inf\{c(x, y) - \phi_*(y) \}
\end{equation*}

ここで、連続関数 $\phi: \Omega \to \mathbb{R}$に対し, そのc-変換 $\phi^c: \Omega \to \mathbb{R}$を次のように定義する.
  \begin{equation*}
    \phi^c(x) := \inf_{y \in \Omega} \{ c(x, y) - \phi(y) \}
  \end{equation*}
従って、式(1)の制約を取り除くためには、$\phi$を$\phi^c$に置き換えるか、$\psi$を$\psi^c$に置き換えるかのどちらかができます。


このことから、以下の2つの関数が得られます。
\begin{align*}
  J(\phi) &= \int \phi  d \nu + \int \phi^c  d \mu, &
  I(\psi) = \int \psi^c  d \nu + \int \psi  d \mu, 
\end{align*}

カントロヴィッチ双対問題
$C = \sup_{\phi, \psi} \int \phi  d\nu + \int \psi  d\mu $
はc-変換を用いて,$J$と$I$それぞれの$\sup$と表すことができる.

なお、$J$は「$\phi$空間」で問題を定式化し、$I$は「$\psi$空間」で問題を定式化している.

The back-and-forthメソッドは、$\phi$-空間での$J$の勾配上昇の更新と$\psi$-空間での$I$の勾配上昇の更新を行うことで、Kantorovichの双対問題を解決します。 
勾配は$\dot{H}^1$距離(式(2)を参照)に関して取られます。
勾配ステップの間に、一方の空間の情報はc-transformを取ることによって、他の空間に逆伝播されます(アルゴリズム1を参照)。\\

The back-and-forth methodの利点は、最適解のペア$(\phi_*, \psi_*)$の特定の特徴が、片方の空間で他方の空間よりも簡単に構築できる場合があるということです。
たとえば、$\psi = \phi^c$のHessianは、$\phi$のHessianの逆行列と非常に密接に関連しています。
したがって、一方の空間で大きなHessian固有値は、他方の空間で小さなHessian固有値に対応し、より小さな特徴をより少ない勾配ステップで構築することができます。
2つの空間を行ったり来たりすることで、より良い状態になります。
つまり、より小さな特徴を構築するのに最適な空間でそれを行う機会が常にあります。
その結果、バックアンドフォース法は、$\phi$空間だけまたは$\psi$空間だけに作用するvanillaの勾配上昇法よりもはるかに迅速に収束します。
実際、特定の例では、追加の$2〜4$回の反復で誤差が$10,000$倍に減少することがあります(表1、2を参照)。\\


$\dot{H}^1$距離における勾配上昇ステップの選択がアルゴリズムの安定性を維持するために重要である。
実際、勾配上昇ステップを$\dot{H}^1$よりも弱いHilbert空間で取ると、双対問題の値の増加を保証することはできないと思われます(詳細については3.1節および命題1を参照)。\\

計算効率に関しては、the back-and-forth methodは$J$と$I$(つまり$\phi$空間と$\psi$空間)で交互に$\dot{H}^1$勾配上昇イテレーションを実行し、c-transforms を計算することに相当します。
重要なことに、$J$の導関数(および対称性から$I$の導関数)は単純な形式を取ることができ、効率的に計算できます[12]。
さらに、多くのコスト関数に対して、c-transforms を非常に効率的に計算できます。
離散グリッド上では、正確な c-transforms は $O(n log(n))$ の操作で計算できます[9,19](詳細についてはSect. 4.1を参照)。


\subsection{Future work and paper outline}

この研究は最適輸送問題に焦点を当てていますが、the back-and-forth methodは多くの他の問題にも応用できると予想されます。
たとえば、the back-and-forth methodは簡単にWasserstein勾配流を計算するために適応できます。
これにより、大規模な重要で興味深いPDEシミュレーションが可能になります。
[6、16、17、21] など、いくつかの例を挙げることができます。
また、この方法は平均場ゲームと呼ばれる急速に発展する分野から生じる計算問題を解決するために有用である可能性があります[15、18]。
今後の研究でこれらの応用を探究することを期待しています。

本論文の残りの部分は次のように構成されています。
第2節では、最適輸送と勾配ベースの最適化スキームに必要な背景情報を再確認します。
第3節では、the back-and-forth methodを紹介し、その安定性と効率性の理由を示します。
第4節では、アルゴリズムの数値実装について説明し、そのパフォーマンスを示すために様々な実験を行います。


\section{Background}

\subsection{Optimal transport and the c-transform}
$\mathbb{R}^d$次元空間の凸でコンパクトな部分集合を$\omega$とする。
$\omega$上のコストとは、連続関数$c: \omega \times \omega \rightarrow \mathbb{R}$のことである。
最適輸送理論では、かなり一般的なコストが考慮されるが、本論文では、以下の場合に焦点を当てる。

\begin{equation*}
  c(x, y) = h(y - x),
\end{equation*}

ただし、$h: \mathbb{R}^d \rightarrow \mathbb{R}$は厳密に凸で偶関数である。
$\omega$上に支持される2つの確率測度$\mu$と$\nu$が与えられた場合、最適輸送問題のモンジュ形式は、次のように定義される。

\begin{equation*}
  C(\mu, \nu) = \inf_T \int_\Omega c(x, T(x)) d \mu (x),
\end{equation*}

ただし、infimumは$\mu$を$\nu$に輸送するような写像$T: \omega \to \omega$対して実行される。
すなわち、$T_\# \mu = \nu$となるものである。\\

$\bold{最適輸送問題(Mongeの問題(1871))}$
  ある砂山から砂山(測度$\mu$)と同じ体積の穴(測度$\nu$)に砂を運ぶ(写像$T$).
  輸送にかかるコストは重さと移動距離に依存する時,コストを最小にする方法を求めよ.\\


写像 T による押し出し測度 $T_\# \mu$ は、可測部分集合 $A \subset \omega$ に対して $T_\# \mu(A) = \mu(T^{-1}(A))$ で定義されます。
また、連続なテスト関数 $f : \Omega \to \mathbb{R}$ に対する押し出し測度の積分を定義することで、押し出し測度を特徴付けることもできます:
{
  \color{red}
\begin{equation}
  \label{def:pushforward_int}
  \int_{\Omega} f(y)d(T_\# \mu)(y) = \int_\Omega f(T(x)) d \mu(x).
\end{equation}
}

\begin{figure}[htbp]
  \begin{center}
    \includegraphics[width=120mm]{images/transport_map2.JPG}
    \caption{transport map}
  \end{center}
\end{figure}

\hypertarget{絶対連続2}
本論文では、確率測度$\mu$と$\nu$が\hyperlink{絶対連続1}{ルベーグ測度に関して絶対連続}である特別な場合に焦点を当てます。
そのため、しばしば測度とその密度関数を同一視します。
{
  \color{teal}
この仮定の下では、$\mu$を$\nu$に移す最適写像$T_*$が唯一存在し、その逆写像$T^{-1}$は$\nu$を$\mu$に移す最適写像です[7,12]。
}

さらに、最適輸送問題のKantorovich双対定式化を解くことで、最適写像を見つけることができます。

双対形式は、pushforward制約条件のためのラグランジュ乗数を導入することで導出することができます。

pushforward制約条件$T_\# \mu = \nu$はすべての連続関数$\phi$について以下が成り立つ場合にのみ成り立ちます。

\begin{equation*}
  \int_{\Omega} \phi(T(x)) d \mu(x) = \int_{\Omega}\phi(y) d \nu(y)
  \quad \left(\Leftrightarrow 
  \int_{\Omega} \phi(T(x)) \mu(x) dx = \int_{\Omega}\phi(y) \nu (y)dy \right).
\end{equation*}

従って
\begin{align*}
  C(\mu, \nu) &= \inf_{T_\# \mu = \nu} \int_\omega c(x, T(x)) d \mu(x)\\
              &= \inf_{T_\# \mu = \nu} \sup_\phi \int_\omega c(x, T(x)) d \mu(x) - \phi(T(x)) d \mu(x) + \int_\omega \phi(y) d \nu(y).
\end{align*}


$\mu$が絶対連続であるとき、infimumとsupremumは入れ替え可能である。
$T$に関する項でまとめると、
\begin{align*}
  C(\mu, \nu) &= \sup_\phi \inf_{T_\# \mu = \nu} \int_\omega \left( c(x, T(x)) - \phi(T(x)) \right) d \mu(x) + \int_\omega \phi(y) d \nu(y).
\end{align*}


このように、ある点$x_0$での$T$の最適な選択は、他の任意の点$x$での選択と完全に切り離されていることが明確になった。
したがって、infimumを積分の内側に移動でき、$\inf_{T(x)} c(x,T(x)) - \psi (T(x))$という操作が双対問題に重要な役割を果たすことが明らかである。
この操作は$c$-変換と呼ばれ、最適輸送の中心にあります。
$c$-変換は、関数$\phi$を別の関数$\phi^c$にマップするものであり、凸解析のlegendre変換(凸共役)の一般化と見ることができます。

\begin{dfn}
  \label{dfn:c-transform}
  連続関数 $\phi: \omega \to \mathbb{R}$に対して、その$c$-transform $\phi^c: \omega \to \mathbb{R}$を以下のように定義する。
  \begin{equation*}
    \phi^c(x) = \inf_{y \in \omega} c(x, y) - \phi(y)
  \end{equation*}
  また, $\phi$ が c-凹関数とは, $\phi = \psi^c$ となる連続関数$\psi: \Omega \to \mathbb{R}$ が存在することをいう.
  さらに関数の組 $(\phi, \psi)$が c-共役であるとは, $\phi = \psi^c$ かつ $\psi = \phi^c$のときをいう.
\end{dfn}


ここで、Kantorovichの双対汎関数を導入する準備が整いました。

\begin{align*}
  J(\phi) &= \int \phi  d \nu + \int \phi^c  d \mu, &
  I(\psi) = \int \psi^c  d \nu + \int \psi  d \mu, 
\end{align*}
以上の議論から、以下が成り立ちます。

\begin{equation*}
  C(\mu, \nu) = \sup_\phi J(\phi) = \sup_\psi I(\psi)
\end{equation*}


\begin{lem}
  \label{lem:c-transform}
  $\left[ 12, 13 \right]$
  連続関数 $\phi: \Omega \to \mathbb{R}$ について、$c$-transform は次の特性を持ちます(\hyperlink{proof:lem:c-transform}{Proof}):
  \begin{enumerate}
  \item 任意の $x \in \Omega$ について、$\phi(x) \le \phi^{cc}(x)$ が成り立ち、$\phi(x) = \phi^{cc}(x)$ であるための必要十分条件は、$\phi$ が $c$-concave であることである。
  その結果、任意の連続関数$\phi$ について、$\phi^{ccc}(x) = \phi^{c}(x)$ が成立する。
  
  \item $\phi$ が $c$-concave の場合、最小化問題 $\inf_{y \in \Omega} c(x, y) - \phi(y)$ はほとんど全ての(almost every) $x$ について唯一の最小値を与える関数(最小化関数) $T_\phi(x)$ を持つ。
  すなわち、$c(x, y) - \phi(y)$が最小値を取る時、 $y = T_\phi(x)$である。
  さらに、次の明示的な式が成り立ちます:
  \begin{equation*}
    T_\phi(x) = x - (\nabla h)^{-1}(\nabla \phi^c(x)),
  \end{equation*}
  ただし $c(x, y) = h(y - x)$ である。
  
  \item $\phi$ が $c$-concave で $u$ が $\omega$ 上の連続関数である場合、ほとんど全ての $x \in \Omega$ について、以下が成り立ちます。
  \begin{equation*}
    \label{eq:c-transform lim}
    \lim_{\epsilon \to 0} \frac{(\phi + \epsilon u)^c(x) - \phi^c(x)}{\epsilon} = - u(T_\phi(x)) 
  \end{equation*}
  \end{enumerate}
\end{lem}

以下は改めて$J$を定義している。
{\color{teal}
\begin{thm}
  $\left[ 7,12,13 \right]$ 
  二重双対問題
  \begin{equation*}
    \sup_\phi \int_\omega \phi  d \nu + \int_\omega \phi^c  d \mu =: J(\phi),
  \end{equation*}
  は、以下の性質を持ちます。

  \begin{enumerate}
    \item $\phi$に関して$J$はconcaveです。
    
    \item $\mu$がルベーグ測度に関して絶対連続である場合、
    $J$は$c$-concaveな関数$\phi_*$によって最大化され、
    $T_{\phi_*}$は$\mu$を$\nu$にプッシュする一意な最適マップです。
    
    \item $\nu$もまた絶対連続である場合、$T\phi_*$はほとんどどこでも(almost everywhere)逆写像を持ち、
    $T^{-1}$は$\nu$を$\mu$にプッシュする最適マップです。
  \end{enumerate}
\end{thm}
}
定理1から、最適マップは凸最大化問題を解くことで計算できることがわかります。
したがって、勾配上昇法などの最適化の標準的なテクニックを用いて、最適輸送問題を解くことができます。
ただし、{\color{red}勾配上昇法を効果的に使用するためには、適切な距離とステップサイズを選択する必要があります。}
そのため、以下ではHilbert空間における勾配上昇法を解説します。

\subsection{Gradient ascent}
凸関数に対する一定ステップサイズ勾配上昇法の初等的な結果について説明。
$(\mathcal{H} , \left\lVert \cdot  \right\rVert_\mathcal{H})$を可分なヒルベルト空間とし、 $F$が滑らかな凸汎関数
\begin{equation*}
  F: \mathcal{H} \to \mathbb{R}
\end{equation*}
であると仮定します。

\begin{dfn}
  点$\phi \in \mathcal{H}$ において、有界線型写像$\delta F_\phi: \mathcal{H} \to \mathbb{R}$が$F$のフレシェ微分(第1変分)であるとは、以下の条件を満たすこととする。
  \begin{equation*}
    \lim_{\left\lVert h \right\rVert_{\mathcal{H}} \to 0} \frac{\left\lVert F(\phi + h) - F(\phi) - \delta F_\phi (h) \right\rVert_{\mathcal{H}}}{\left\lVert h \right\rVert_{\mathcal{H}}} = 0.
  \end{equation*}
\end{dfn}


\begin{dfn}
  \label{dfn:inner product}
  $\langle \cdot, \cdot \rangle_\mathcal{H}$をHilbert空間$\mathcal{H}$上で定義された内積とする。
  写像$\nabla_\mathcal{H} F: \mathcal{H} \to \mathcal{H} $が$F$の$\mathcal{H}$-勾配であるとは、任意の$(\phi, h) \in \mathcal{H} \times \mathcal{H}$に対して、以下の式が成立することを言う。

  \begin{equation*}
    \langle \nabla_\mathcal{H} F(\phi), h \rangle_\mathcal{H} = \delta F_\phi(h)
  \end{equation*}

\end{dfn}

The back-and-forth methodでは、$\dot{H}^1$-勾配を使用します。ここで、$\dot{H}^1(\Omega)$は以下のように定義されます。

\begin{equation}
\dot{H}^1(\Omega) = \left\{ \varphi: \Omega \rightarrow \mathbb{R} : \int_\Omega \varphi(x) dx = 0 \quad and \quad \int_{\Omega} |\nabla \varphi|^2 dx < \infty \right\},
\end{equation}


$H^1(\Omega)$上の内積は、以下のように定義されます。

\begin{equation*}
\langle \varphi_1, \varphi_2 \rangle_{\dot{H}^1(\Omega)} = \int_{\Omega} \nabla \varphi_1(x) \cdot \nabla \varphi_2(x) dx.
\end{equation*}

双対空間$\dot{H}^{-1}$も述べます。2つの確率密度関数$\rho_1$と$\rho_2$に対して、次のように定義します。

\begin{equation}
  \label{eq:H1metric}
  \left\lVert \rho_2 - \rho_1 \right\rVert_{\dot{H}^{-1}}^2 = \int_{\omega} |\nabla \phi(x)|^2 dx
\end{equation}
  
ただし、$\phi$ はラプラス方程式 $-\Delta \phi = \rho_2 - \rho_1$ かつ zero Neumann 境界条件を満たす $\dot{H}^1(\omega)$ の唯一の解である。(詳細は [22, Section 5.5.2] を参照)。

以下の補題は、$\dot{H}^1$ 勾配が特に単純な形式を持つことを示している。証明は積分部分法の直接的な応用である。

\begin{lem}
  \label{lem:variation inf}
  $F: \dot{H}^1(\omega) \rightarrow \mathbb{R}$が、$\forall \phi \in \dot{H}^1(\omega)$に対して、
  任意の$h \in \dot{H}^1(\omega)$で評価される一次変分$\delta F_\phi$が関数$f_\phi$に対する積分として表される、つまり
  $$\delta F_\phi(h) = \int_{\Omega} h(x) f_\phi(x) dx$$
  このとき、$F$の$\dot{H}$-勾配は次の形式を持ちます。
  $$
  \nabla_{\dot{H}^1} F(\phi) = (- \Delta)^{-1} \bar{f_\phi},
  $$
  ここで、$(-\Delta)^{-1}$はzero Neumann境界条件を持つ負ラプラシアンの逆演算子を表し、
  $\bar{f_\phi} = f_\phi - \frac{1}{|\Omega|} \int_\Omega f_\phi$.
\end{lem}

\begin{proof}
  jacobs the back-and-forth method for wasserstein gradient flows.pdfを参照。
\end{proof}

以下は、一般的なHilbert空間上での勾配法について説明したものです。
勾配法は、Fréchet-微分可能な汎関数 $F:\mathcal{H} \to \mathbb{R}$ を最大化することを目的として、以下の反復式で表されます。

$$
\phi_{n+1} = \phi_n + \sigma \nabla_\mathcal{H} F(\phi_n),
$$
ここで、ステップサイズ $\sigma>0$ は定数です。最大値に収束するためには、つまり
$$
F(\phi_n) \to \sup F,
$$
収束するためには、勾配写像 $\nabla_\mathcal{H} F$ の連続性を制御する必要があります。
\hypertarget{定理2.4}
次の勾配法の収束定理は、最適化の基本的な考え方の1つです。


\begin{thm}([20]を参照)
  \label{thm:ascent property}
  (\hyperlink{proof:thm:ascent property}{Proof})
  $F:H \rightarrow \mathbb{R}$ をFrechet微分可能な凹関数とし、ある定数$\sigma>0$が存在して、
  $$
  F(\phi) \geq F(\hat{\phi}) + \delta F_{\hat{\phi}} (\phi -\hat{\phi}) - \frac{1}{2 \sigma} \| \phi - \hat{\phi} \|_\mathcal{H}^2
  $$
  全ての$\phi,\hat{\phi}\in\mathcal{H}$について成り立つと仮定する。
  このとき、勾配降下法の反復
  $$
  \phi_{n+1} = \phi_n + \sigma \nabla_\mathcal{H} F(\phi_n)
  $$
  \hypertarget{上昇性}
  は、以下の上昇性の性質を持つ:
  $$
  F(\phi_{n+1}) - F(\phi_n) \geq \frac{\sigma}{2} \|\nabla_\mathcal{H} F(\phi_n)\|_\mathcal{H}^2
  $$
  {\color{teal}
  さらに、$F$が唯一の極大点$\phi_*$を持ち、$\sup_n \|\phi_n \|_\mathcal{H} < \infty$である場合、列$\{\phi_n\}_{n=0}^{\infty}$は$\phi_\ast$に弱収束する。
  }
\end{thm}

\vspace\baselineskip
定理2の証明については,"Appendix"を参照。これらの道具を手に入れたので,我々は交互法を導入する準備が整った。


\section{The back-and-forth method}
\label{sect:the back-and-forth method}

$\omega \subset \mathbb{R}^d$を凸かつコンパクトな領域とし、次の形式のコスト$c: \omega \times \omega \to \mathbb{R}$を考える。
$$
c(x,y) = h(y - x).
$$
ただし、$h: \mathbb{R}^d \to \mathbb{R}$は厳密に凸で偶関数である。
確率密度関数$\mu$と$\nu$が$\omega$で\hyperlink{支持されている}{支持されている}とき、最適輸送問題の双対形式である次式を考えることを考えます。
\begin{equation}
  C(\mu,\nu) = \sup_{\phi}\int_{\omega}\phi(y)\nu(y)dy + \int_{\omega}\phi^c(x)\mu(x)dx,
\end{equation}
ここで、supremumは連続関数$\phi:\omega \to \mathbb{R}$について実行され、c-transformは$\phi^c(x) = \inf_{y\in\omega}(c(x,y)-\phi(y))$で定義されます。

双対問題の最大化解を計算する効率的なアルゴリズムを開発するのが目的。
双対問題の最大化解を計算するための2つの等価な形式で双対関数を考える。


この節での目的は、デュアル問題の最大化関数を効率的に計算するアルゴリズムを開発することです。以下では、デュアル関数を2つの等価な形式で考えます。

\begin{equation}
  J(\phi) = \int_\omega \phi \, d\nu + \int_\omega \phi^c \, d\mu 
\end{equation}

\begin{equation}
  I(\psi) = \int_\omega \psi^c \, d\nu + \int_\omega \psi \, d\mu
\end{equation}

$J$と$I$の関数は基本的に同じであることに注意してください。
異なる点は、$\mu$と$\nu$の役割が逆転することです。
これ以上進めるには、$J$と$I$の変分の式が必要です。

\begin{lem}
  $\left[11,12 \right]$ 
  \label{lem:variation}
  (\hyperlink{proof:lem:variation}{Proof})
  連続関数 $\phi: \omega \rightarrow \mathbb{R}$ および $\psi: \omega \rightarrow \mathbb{R}$ 上で定義される関数 J および I を考える。
  $\phi$ が $c$-凹関数である場合、$J$ の第一変分は、以下のように表される。
  $$
  \delta J_\phi = \nu - T_{\phi \#} \mu,
  $$
  同様に、$\psi$ が c-凹関数である場合、$I$ の第一変分は以下のように表される。
  $$
  \delta I_\psi = \mu - T_{\psi \# }\nu.
  $$
  ここで、任意の c-凹関数 $\phi: \omega \rightarrow \mathbb{R}$ に対して、
  $$
  T_\varphi(x) = x - (\nabla h)^{-1}(\nabla \varphi^c(x)).
  $$
  になることを思い出す。
\end{lem}

\vspace\baselineskip
最適輸送問題を解決するためのアプローチ、すなわち. the back-and-forth method を紹介する準備が整った。
アルゴリズム1に概要を示す。アルゴリズムは、次の2つの主要なアイデアに基づいている:

\begin{enumerate}
  \item $\dot{H}^1$ metricの勾配上昇ステップであり、(Lemmma \ref{lem:variation inf}, \ref{lem:variation}を参照)
        $$
        \nabla_{\dot{H}^1} J(\phi) = (- \Delta)^{-1} (\nu - T_{\phi \#} \mu),
        $$
        \begin{equation}
          \nabla_{\dot{H}^1} I(\psi) = (- \Delta)^{-1} (\mu - T_{\psi \#} \nu).
        \end{equation}
  
  \item back-and-forth update schemeは$J$と$I$の勾配上昇ステップを交互に行うものである。
\end{enumerate}

ステップサイズ$\sigma$に関する情報については、セクション4.2を参照してください。


\subsection{$\dot{H}^1$-gradientascent}
\label{sect:H1-gradient ascent}
The back-and-forth methodの主要なステップは、$\dot{H}^1$メトリックにおける勾配上昇ステップであり、
\begin{align*}
\phi_{n + \frac{1}{2}} &= \phi_n + \sigma \nabla_{\dot{H}^1} J(\phi_n), \\
\psi_{n + 1} &= \psi_{n + \frac{1}{2}} + \sigma \nabla_{\dot{H}^1} I(\psi_{n + \frac{1}{2}}).
\end{align*}

The back-and-forth methodの収束性を得るためには、これらのステップがそれぞれ双対関数$J$および$I$の値を増加させるかどうかを知る必要があります。
以下では、対称性により、$J$について成立する性質は$I$についても成立するため、$J$の更新ステップに焦点を当てます。\\



前のセクションの\hyperlink{定理2.4}{定理2.4}を思い出すと、一般的なHilbert空間$\mathcal{H} $において、$\mathcal{H}$-勾配上昇ステップ

$$\phi_{n+1}=\phi_n+\sigma\nabla_H J(\phi_n)$$
は、不等式

\begin{equation}
  \label{eq:variation ascent}
  J(\phi)\geq J(\hat{\phi})+\delta J _{\hat{\phi}}(\phi-\hat{\phi})-\frac{1}{2\sigma}\|\phi-\hat{\phi}\|_{\mathcal{H}}^2
\end{equation}
が$\sigma>0$と任意の$\phi,\hat{\phi}\in H$について成立するならば、勾配上昇ステップは\hyperlink{上昇性}{上昇性}を持ちます。すなわち、

$$
J(\phi_{n+1})\geq J(\phi_n) + \frac{\sigma}{2}\|\nabla_\mathcal{H}  J(\phi_n)\|_\mathcal{H}^2
$$

が成立します。
注意すべきは、この不等式は、$\mathcal{H} $に関連するノルムが強くなると(つまり、右辺がより負になると)、満たしやすくなることです。\\

{\color{teal}
不等式を確立するために、次の式の下限を求める必要があります。
$$
J(\phi) - J(\hat{\phi}) - \delta_{\hat{\phi}} J (\phi - \hat{\phi})
$$
この式は、$\hat{\phi}$ 周りで1次のTaylor展開で $J$ を近似することによる誤差と認識できます。

{\color{teal}
進展するための最も簡単な方法は、凹不等式 $J(\hat{\phi}) \le J(\phi) + \delta J_{\phi}(\hat{\phi} - \phi)$ を使って、以下の下限を得ることです。}
$$
J(\phi) - J(\hat{\phi}) - \delta_{\hat{\phi}} J(\phi-\hat{\phi}) \ge (\delta J_{\phi} - \delta J_{\hat{\phi}}) (\phi - \hat{\phi})
$$


Lemma \ref{lem:variation}の一次変分の明示的な式を使って、以下のように書くことができます。
\begin{align*}
  (\delta J_\phi - \delta J_{\hat{\phi}})(\phi - \hat{\phi})  &= \int_\omega (\phi - \hat{\phi})(\nu - T_{\phi \#} \mu) - \int_\omega (\phi - \hat{\phi})(\nu - T_{\hat{\phi} \#} \mu)\\
                                                              &= \int_\omega (\phi - \hat{\phi})\left((\nu - T_{\phi \#} \mu) - (\nu - T_{\hat{\phi} \#} \mu)\right)\\
                                                              &= \int_\omega (\phi - \hat{\phi})(T_{\hat{\phi} \#} \mu - T_{\phi \#} \mu)
\end{align*}


ここで、不等式(\ref{eq:variation ascent})は以下の式を示すことができれば成立します。  (\hyperlink{proof:eq:variation int ineq}{Proof})


\begin{equation}
  \label{eq:variation int ineq}
  \int_\Omega (\phi - \hat{\phi})(T_{\hat{\phi} \#} \mu - T_{\phi \#} \mu) \geq - \frac{1}{2\sigma} \|\phi - \hat{\phi}\|_\mathcal{H}^2
\end{equation}

私たちの目標は、この不等式が成り立つようなヒルベルト空間$\mathcal{H}$とパラメータ$\sigma$を見つけることです。\\

任意のヒルベルト空間 $H$ に対して、以下のコーシー・シュワルツの不等式が成立します。

\begin{equation}
  \label{eq:cauchy}
  \int_\Omega (\phi - \hat{\phi})(T_{\hat{\phi} \#} \mu - T_{\phi \#} \mu) \geq - \|\phi - \hat{\phi}\|_{\mathcal{H}} \|T_{\hat{\phi} \#} \mu - T_{\phi \#} \mu\|_{\mathcal{H}^*},
\end{equation}
ここで、$\mathcal{H}^*$ は $L^2$ 内積に関する $\mathcal{H}$  の双対空間です(すなわち、$\mathcal{H}$  を $L^2(\omega)$ といういわゆる "pivot空間" に関して双対化しています)。


適切なヒルベルト空間 $\mathcal{H}$ を選ぶことで、
\begin{equation}
  \label{eq:ineq of T ond inner product}
  \|T_{\hat{\phi} \#} \mu - T_{\phi \#} \mu\|_{\mathcal{H}^*} \leq \frac{1}{2 \sigma} \|\phi - \hat{\phi}\|_\mathcal{H} ,
\end{equation}

式(\ref{eq:ineq of T ond inner product})が成立することを示せば、コーシー・シュワルツの不等式(\ref{eq:cauchy})と組み合わせることで式(\ref{eq:variation ascent})が得られます
(注意:より正確な議論により、式(\ref{eq:ineq of T ond inner product})の右辺にある $\frac{1}{2}$ の因子は除去できますが、ここでは気にしません)。
再度強調しますが、$\mathcal{H}$ に関連付けられたノルムが強くなるにつれて、不等式(\ref{eq:ineq of T ond inner product})はより簡単に満たされるようになります。
実際、$\mathcal{H}$ が強くなると $\mathcal{H}^*$ は弱くなり、左辺が小さくなる一方で、右辺が大きくなります。\\
{\color{teal}
不等式(\ref{eq:ineq of T ond inner product})は、関数$\phi \mapsto T_{\phi \#}$が$\mathcal{H} \to \mathcal{H}$へのLipschitz連続写像である必要があることを示しています。
}
したがって、上昇特性を持つ勾配上昇スキームを得るために、$\phi$と$\hat{\phi}$が$\mathcal{H}$で近ければ、$T_{\phi \#} \mu$と$T_{\hat{\phi} \#} \mu$が$\mathcal{H}^*$で近くなるHilbert空間$\mathcal{H}$を選ぶ必要があります。
これは容易な課題ではありません。
たとえば、$\phi$が$c$-凸でない場合、写像$\phi \mapsto T_\phi$ がwell-definedではない場合があり、必ずしも特定の連続性を持たない場合があります。\\


上記の複雑さに直接対処しようとすると、どのように進めればいいかは明確ではありません。
進展をするために、二次コストかつあるwell-definedな$c$-凹関数に制限したとき、Hilbert空間$\dot{H}^1(\Omega)=\{\phi:\Omega\to\mathbb{R} \mid \int_\Omega \phi = 0 \text{ かつ } \|\nabla \varphi\|_{L^2}<\infty\}$が勾配上昇を保持できることを示します。


結果を述べる前に、$\dot{H}^1$の選択肢はおそらく弱めることができません。
実際、$T\phi$の式は$\nabla\phi^c$に依存するため、$\phi$と$\hat{\phi}$の勾配を何らかの制御を持つ必要があり、
(\ref{eq:variation ascent})または(\ref{eq:ineq of T ond inner product})を示す可能性があります。\\
}

\begin{dfn}
  \label{dfn:Bregman divergence}
  $\mathcal{H}$をヒルベルト空間とし、Fréchet微分可能な関数$F: \mathcal{H} \to \mathbb{R}$を考えます。
  Fに関連するBregman divergence はすべての$\phi_1, \phi_2 \in \mathcal{H}$に対して次のように定義されます:
  $$
    F(\phi_2|\phi_1) = F(\phi_2) - F(\phi_1) - \delta F_{\phi_1}(\phi_2 - \phi_1),
  $$
  同様に、$\omega$が$\mathbb{R}^d$の閉凸集合であり、$f: \omega \rightarrow \mathbb{R}$が微分可能な関数である場合、すべての$x_1, x_2 \in \omega$に対して次のように定義します:
  $$
    f(x_2|x_1) = f(x_2) - f(x_1) - \nabla f(x_1) \cdot (x_2 - x_1),
  $$
\end{dfn}
上記の定義から示唆されるように、私たちはBregman距離を、Kantorovichポテンシャル($\Omega$上で定義されたもの)および双対関数(ヒルベルト空間$\dot{H}^1(\Omega)$上で定義されたもの)の両方において使用します。
次の結果は、Bregman距離の理論でよく知られている性質をまとめたものです。

\begin{lem}
  \label{lem:convex conjugate}
  $f$を$\omega$上で定義された凸で微分可能な関数とし、その凸共役を$f^*(x) = \sup_y x\cdot y- f(y),$と表すと、以下のようになります:
  \begin{enumerate}
    \item 任意の$\lambda \geq 0$に対して、$f$が$\lambda$-強凸関数 $\Leftrightarrow$ $f(x_2|x_1) \geq \frac{\lambda}{2} |x_2 - x_1|^2$.
    \item $p_i = \nabla f(x_i)$とするとき、すべての$x_1, x_2 \in \omega$に対して、$f(x_2|x_1) = f^*(p_1|p_2)$が成り立ちます。
    \item 任意の$\lambda > 0$に対して、$f$が$\lambda$-強凸である場合、すべての$x_1, x_2 \in \omega$に対して、$f(x_1|x_2) \leq \frac{\lambda^{-1}}{2} |p_2 - p_1|^2$が成り立ちます。ここで、$p_i = \nabla f(x_i)$とします。
    \item $\omega$内のある固定された$x_0$を選びます。$f(u) = f(x_0 + u|x_0)$および$g(v) = f^*(p_0 + v|p_0)$とします。
          ここで、$x_1, x_2 \in \omega$、$p_0 = \nabla f(x_0)$とします。すると、$f^*(v) = g(v)$が成り立ちます。
  \end{enumerate}
\end{lem}


関数$\xi$を$\omega$上で定義された凸で微分可能な関数としたとき、そのLegendre変換(凸共役)\,$\xi^*$は以下のように定義できる.
$$
    \xi^*(p) := \sup_x p \cdot x - \xi(x) 
$$

ここで、双対汎関数$J$は$J(\varphi) =  \int \phi\, \nu  + \int \phi^c \, \mu$で定義していたことを思い出す。
{\color{teal}
これは凹な汎関数であり、第一項 $\int \phi\, \nu$ は$\phi$に対して線形であるため、$J$の凸性には影響しません。
}
そのため、$F$を次のように定義します。
\begin{equation}
  \label{eq:F(phi)}
    F(\phi) = - \int \phi^c \, \mu,
\end{equation}
これは、本質的には線形項のみが異なるため、$J$と同じ凸性を持つ凸関数です.
例えば、任意のポテンシャル $\phi_1$ と $\phi_2$ に対して、次の式が成立することを直接確認できます。
{\color{teal}
\begin{equation}
  \label{eq:J-F}
  J(\phi_2|\phi_1) = -F(\phi_2|\phi_1).
\end{equation}
}
最後に、$F$の凸共役は次のように定義されます。
\begin{equation}
    F^*(\rho) := \int_\omega \phi \, \rho - F(\phi)
\end{equation}

\begin{lem}
  \label{lem:stability estimates}
  $\phi_1$と$\phi_2$を$\omega$上で定義される二回微分可能な実数値関数とし、以下の条件を満たすとします(\hyperlink{proof:lem:stability estimates}{Proof}):
  $$
    (1-\lambda^{-1})I \leq D^2\phi_i(x) \leq (1-\lambda)I, \quad i = 1, 2, \quad (\lambda > 0)
  $$

  また、$\rho_i = T_{\phi_i \#} \mu$を関連する質量密度とすると ($i = 1, 2$)、以下の不等式が成り立ちます:
  \begin{equation}
    \label{eq:F_upper bound}
    F(\phi_2 | \phi_1) \leq \frac{1}{2}\|\mu\|\lambda^{-(d+1)}\|\phi_2 - \phi_1\|^{2}_{\dot{H}^1}. 
  \end{equation}
  \begin{equation}
    \label{eq:F_lower bound}
    F(\phi_2 | \phi_1) \geq \frac{1}{2}\|\mu\|^{-1}\lambda^{(d+1)}\|\rho_2 - \rho_1\|_{\dot{H}^1}^2.
  \end{equation}
\end{lem}

\begin{prop}
  \label{prop:勾配上昇保持}
  (\hyperlink{proof:prop:勾配上昇保持}{Proof})
  2次コスト$h(y-x)=\frac{1}{2}|x-y|^2$を持つKantorovich双対問題$\sup_{\phi}J(\phi)$を考える。
  $\lambda>0$が存在して、
  $$
  \phi_n, \phi_{n + \frac{1}{2}} \in S_{\lambda}:=\{\phi:(1-\lambda^{-1})I \leq D^2\phi(x) \leq (1-\lambda)I\}
  $$
  が任意の$n$に対して成り立つとき、ステップサイズ
  $\sigma = \|\mu\|^{-1}\lambda^{-d+1}$を持つ勾配上昇ステップ
  $$
  \phi_{n+\frac{1}{2}} = \phi_n + \sigma \nabla_{\dot{H}^1} J(\varphi_n)
  $$
  は、以下の上昇性を満たします。
  $$
  J(\phi_{n + \frac{1}{2}})-J(\phi_n) \geq \frac{1}{2}{\|\mu\|^{-1}\lambda^{d+1}}\|\nabla_{\dot{H}^1} J(\phi_n)\|_{\dot{H}^1}^2.
  $$
  さらに、質量密度はターゲット$\nu$に近づきます。
  $$
  \|\rho_{n + \frac{1}{2}} - \nu\|_{\dot{H}^{-1}}^2 - \|\rho_n - \nu\|_{\dot{H}^{-1}}^2 \leq -\|\rho_{n + \frac{1}{2}} - \rho_n\|_{\dot{H}^{-1}}^2,
  $$
  ここで、$\rho_n = T_{\phi_n \#}\mu$および$\rho_{n+\frac{1}{2}} = T_{\phi_{n+\frac{1}{2}} \#}\mu$を表し、$\dot{H}^{-1}$距離は(\ref{eq:H1metric})で定義されます。

\end{prop}


証明はAppendixを参照。

\begin{rem}
  命題\ref{prop:勾配上昇保持}はすべての反復子$\phi_n$が上下ヘシアン境界を持つと仮定すると、$\dot{H}^1$上で下降特性が成立することを示しています。
  残念ながら、単純な上昇スキーム$\phi_{n+1} = \phi_n + \sigma \nabla_{\dot{H}^1} J (\phi_n)$を通じてこの性質が維持されるとは期待できません。
  加えて、勾配の更新によって、いずれはイテレーションが望ましい領域の外に押し出される可能性があります。
  実際、勾配更新ステップによって$c$-凹性が維持されることすら保証できません。
\end{rem}


\begin{rem}
  前述の問題を回避する方法の1つは、反復解を集合
  $$
  S_{\lambda} = \{\phi \,\, | \,\, (1-\lambda^{-1})I \leq D^2\phi(x) \leq (1-\lambda)I \}
  $$
  に射影するステップを追加することです。
  $S_{\lambda}$は凸集合であるので、射影ステップは上昇特性に干渉しません。
  残念ながら、$E_\lambda$への射影を計算するのはコストがかかります(半定値計画問題を解く必要があるため)。
  さらに、双対問題の最大化は、任意の$\lambda > 0$に対して$S_\lambda$内に存在するわけではありません
  (ただし、二次コストの場合は$c$-凹性により$S_0$内に存在する必要があります)。
  このような場合、任意の正確な解を得るために$\lambda \to 0$として、$\lambda$を$0$に近づける必要があります。
\end{rem}

\vspace\baselineskip 

上記のRemarkで言及された問題のいくつかを回避するために、我々は従来の勾配上昇法の代わりに、$J$を最大化するステップと$I$を最大化するステップを交互に繰り返す前後法を用いることにします。



\subsection{Back-and-forth updates}

Kantorovichの双対問題は、以下の形式で書くことができます。

\begin{equation}
  \label{eq:Kantorovich問題}
  C(\mu,\nu)=\sup_{\phi,\psi} \left\{\int_\omega \phi(y) \nu(y) \,d\nu(y)+\int_\omega \psi(x) \mu(x)\,d\mu(x)\right\},
\end{equation}
ここで、上限は、すべての$x,y\in \omega$に対して
$$
\phi(y)+\psi(x)\leq h(y-x)
$$
を満たす連続関数$\phi$と$\psi$の上で取られます。
この式は、ポテンシャル$\phi$と$\psi$が対称的な役割を果たすことを強調しています。
以下のように、双対汎関数を次の形式で表すことが便利です。

\begin{equation}
  \label{eq:双対汎関数}
  D(\phi, \psi) = \int_\omega \phi \,\nu\, dy + \int_\omega \psi \,\mu\, dx - \iota_C(\phi,\psi)
\end{equation}
ここで、制約条件は凸指示関数 $\iota_C$ によって符号化されます。凸集合
$$
\mathcal{C}  = \{(\phi,\psi) \,| \, \forall x,y \in \omega,\, \phi(y) + \psi(x) \leq h(y-x)\}
$$
上では、$\iota_C$ は0をとり、他のペア$(\varphi, \psi)$に対しては$+ \infty$をとります。

\color{teal}
Kantorovich問題(\ref{eq:Kantorovich問題})は、 双対汎関数$D$(\ref{eq:双対汎関数})を用いて、
$C(\mu, \nu) = \sup_{\phi, \psi} D(\phi, \psi)$
と簡潔に書くことができます。

また、対称的な表現(\ref{eq:双対汎関数})から、$\phi$または$\psi$を排除することで、関数$J$と$I$を求めることができます。
具体的には、
$$
\phi^c = \arg\max_\psi D(\phi, \psi), \quad \psi^c = \arg\max_\phi D(\phi, \psi)
$$
を見つけることで、次式を得る。
$$
J(\phi) = D(\phi, \phi^c) = \sup_\psi D(\phi, \psi), \quad I(\psi) = D(\psi^c, \psi) = \sup_\phi D(\phi, \psi).
$$
\color{black}
vanilla勾配上昇法 $\phi_{n+1} = \phi_n + \sigma\nabla_{\dot{H}^1} J(\phi_n)$ は、
(任意の) $\phi$-空間にフォーカスし、関数 $J$ の勾配ステップを反復します。
また、代わりに $\psi$-空間で作業し、関数 $I$ 上の勾配上昇スキームを書くこともできます。
おそらくより良いアイデアは、$\phi$-空間と $\psi$-空間を交互に行き来することです。
これは、アルゴリズム1で示されている back-and-forth 方法の中心的なアイデアです。
現在の反復$(\phi_n, \psi_n)$は以下のように更新されます:
\begin{align*}
  \phi_{n+\frac{1}{2}} &= \phi_n + \sigma \nabla_{\dot{H}^1} J(\phi_n),\\
  \psi_{n+\frac{1}{2}} &= (\phi_{n+\frac{1}{2}})^c,\\
  \psi_{n+1} &= \psi_{n+\frac{1}{2}} + \sigma \nabla_{\dot{H}^1} I(\psi_{n+\frac{1}{2}}), \\
  \quad \phi_{n+1} &= (\psi_{n+1})^c.
\end{align*} 


The back-and-forthのアプローチは、前のサブセクションで特定した問題を既に解決しています。 
$\phi$と$J$に関しての純粋な勾配上昇スキーム$\phi_{n + 1} = \phi_n + \sigma \nabla_{\dot{H}^1} J(\phi_n)$(または同様に$\psi$と$I$に関して)、その繰り返し(反復)が$c$-concaveであることが保証されません.
$\phi$または$\psi$が$c$-concaveでない場合、勾配$\nabla_{\dot{H}^1} J(\phi)$または$\nabla_{\dot{H}^1} I(\psi)$がwell-definedでない可能性があります。
{\color{red}
the back-and-forth methodでは、勾配を取るたびに、対象の関数が常に$c$-concaveであるため、問題がありません。
実際、$\phi_n = \psi_{n -1}^c$および$\psi_{n + \frac{1}{2}} = \phi^c_{n + \frac{1}{2}}$であるため、$\nabla_{\dot{H}^1} J(\phi_n)$および$\nabla_{\dot{H}^1} I(\psi_{n + \frac{1}{2}})$は常にwell-definedです。\\
}

さらに、the back-and-forthの更新は純粋な勾配上昇法よりも悪化することはありません。
実際、$c$-transformを行う中間ステップは、双対問題の値を増加させることしかできません。
{\color{teal}
Lemma \ref{lem:c-transform}の性質$(i)$から、以下が成り立ちます。

$$
J(\phi_{n + \frac{1}{2}}) \leq J(\phi_{n + \frac{1}{2}}^{cc}) = I(\psi_{n + \frac{1}{2}}),
$$

$$
I(\psi_{n+1}) \leq I(\psi_{n + 1}^{cc}) = J(\phi_{n+1}).
$$
}

したがって、中間の$c$-transformのステップは双対的な機能の値を増加させるのに役立ちます。
これをSect. \ref{sect:H1-gradient ascent}の以前の結果と組み合わせると、The back-and-forth methodに関する以下の結果を得ることができます。\\

\color{teal}
\begin{prop}
  \label{prop:不等式連鎖}
  (\hyperlink{proof:prop:不等式連鎖}{Proof})
  二次コスト最適輸送問題に対するthe back-and-forth method(Algorithm 1)を考えます。
  すべての繰り返し$\phi_n$, $\phi_{n + \frac{1}{2}}$, $\psi_n$, $\psi_{n + \frac{1}{2}}$が、
  関数の集合
  $$
  S_{\lambda}=\{\varphi:(1-\lambda^{-1})I\leq D^2\varphi(x)\leq (1-\lambda)I\},
  $$
  に属している$\lambda > 0$が存在するとします。このとき、すべての整数$n$に対して次の不等式連鎖が成り立ちます。
  $$
  D(\phi_{n+1},\psi_{n+1})\geq D(\phi_{n + \frac{1}{2}},\psi_{n + \frac{1}{2}})\geq D(\phi_n,\psi_n),
  $$
  同値的に、$I$と$J$の値は交互に増加します。
  $$
  I(\psi_{n+1})\geq I(\psi_{n + \frac{1}{2}})\geq J(\phi_{n + \frac{1}{2}})\geq J(\phi_n)\geq I(\psi_n).
  $$
\end{prop}

\color{black}
Proposition \ref{prop:不等式連鎖}の簡単な証明は「付録」に記載されています。

\begin{comment}
  Proposition 2の簡単な証明を以下に示します。

  1. 全ての反復$\varphi_n$、$\varphi_{n+1/2}$、$\psi_n$、$\psi_{n+1/2}$が関数集合$S_{\lambda}=\{\varphi:(1-\lambda^{-1})I\leq D^2\varphi(x)\leq(1-\lambda)I\}$に属すると仮定します。
  2. 反復の定義を使用して、$\varphi_{n+1}$と$\psi_{n+1}$を$\varphi_n$、$\varphi_{n+1/2}$、$\psi_n$、$\psi_{n+1/2}$の式で表します。
  3. 関数集合$S_{\lambda}$に関する仮定を使用して、$\varphi_{n+1}$と$\varphi_{n+1/2}$の差、および$\psi_{n+1}$と$\psi_{n+1/2}$の差を境界します。
  4. 差の境界と二次コスト関数の性質を使用して、不等式$D(\varphi_{n+1},\psi_{n+1})\geq D(\varphi_{n+1/2},\psi_{n+1/2})\geq D(\varphi_n,\psi_n)$を導出します。
  5. 原始問題と双対問題の双方の関係性と二次コスト関数の性質を使用して、不等式$I(\psi_{n+1})\geq I(\psi_{n+1/2})\geq J(\varphi_{n+1/2})\geq J(\varphi_n)\geq I(\psi_n)$を導出します。

  注意:これは一般的な概要であり、実際の証明には追加の手順と技術的な詳細が必要な場合があります。
\end{comment}

\vspace\baselineskip 

The back-and-forthの更新を行っても、反復関数の凸性の境界条件を仮定しない限り、厳密な収束証明を行うことはできないことに注意。
しかし、私たちが完全に理解していない理由で、the back-and-forth methodは、比較的大きなステップサイズでも非常に安定しています(JまたはIのvanilla上昇法はより不利に振る舞います)。
さらに、the back-and-forth methodは最大値に非常に迅速に収束します。
この発見には、特定の特徴を "$\phi$-空間"または "$\psi$-空間"のどちらかで構築することが容易であるためであると考えられます。
実際、二次コストのための$c$-共役関数Def.\ref{dfn:c-transform}のペア$(\phi, \psi)$がある場合、次の関係が成立します。
$$
D^2 \phi(x - \nabla\phi(x)) = (D^2 \psi(x))^{-1}
$$
すなわち、二次$c$-transformはヘシアンの上限を反転させます。
空間を交互に切り替えることで、より小さい空間で特徴を構築し、より迅速に解に収束することができます。


\section{Numerical implementation and results}
\subsection{The fast c-transform}


定格グリッド上での$c$-変換の計算は、次の最適化問題を各グリッド点$x$に対して解く必要があります。
$$
\phi^c(x) = \inf_y h(y - x) - \phi(y)
$$
一次元の場合、コストの厳密な凸性は、$y$に関する導関数
$$
h^\prime(y - x) - \phi^\prime(y)
$$
が$x$の減少関数であることを意味します。
この結果、最小化する $y(x)=\underset{y}{\operatorname{argmin}} \, h(y-x) - \phi(y)$ は $x$ に関して単調に増加することがわかります。
つまり、$x_1 \leq x_2$ のとき、$y(x_1) \leq y(x_2)$ です。
この観察結果は、$n$個の点における1次元の$c$-transformを $O(n \log(n))$ の操作で計算するための分割統治アルゴリズムの設計に利用できます [9]。


このアルゴリズムのキーとなるアイデアは、点$x = \frac{k}{n},\frac{k + 2}{n}$について最小値を持つ$y(x)$がわかっている場合、中央の点の最小値が他の$2$点の間に「閉じ込められる」ということです。つまり、

$$
y\left(\frac{k}{n} \right) \leq y\left(\frac{k + 1}{n} \right) \leq y\left(\frac{k + 2}{n} \right),
$$
となります。
したがって、偶数のグリッドポイントの$c$-transformと、奇数のグリッドポイントの$c$-transformを別々に計算し、上記の交錯性を使用して完全な解を再構築することができます。
具体的なアルゴリズムや詳細については、[9]を参照してください。\\

このため、各次元で分解するコスト関数 
$$
h(y-x) = \sum_{i = 1}^{d}  h_i(y_i - x_i),
$$ に限定すると、$1$-次元の $c$-transform を反復適用することで $d$-次元の $c$-transform を計算することができます。
例えば、2次元の場合は次のようになります。

\begin{align*}
  \inf_y h(y - x) - \phi(y) &= \inf_{y_2} \inf_{y_1} \left\{ h_2(y_2 - x_2) + h_1(y_1 - x_1) - \phi(y_1, y_2) \right\}  \\
                            &= \inf_{y_2} \left\{ h_2(y_2 - x_2) + \phi(\cdot, y_2)^c(x_1) \right\}.
\end{align*}
ここで、$\tilde{\phi}(x_1, y_2) = -\phi(\cdot, y_2)^c(x_1)$ と書くと、最後の式は$y_2$に関する$c$-transformとなります。
$$
  \inf_{y_2} h_2(y_2 - x_2) - \tilde{\phi}(x_1, y_2).
$$

したがって、$2$次元の場合、全体の$c$-transformを計算するためには、
まずすべての水平線に沿って1次元のc-transformを計算し、次にすべての垂直線に沿って計算します。
このアイデアは任意の次元に一般化され、多次元高速フーリエ変換(FFT)を計算するために使用される完全に同じメカニズムです。\\

極めて重要で特別な$h(y-x)=\frac{1}{2}|y-x|^2$ の場合、$c$-transform はより効率的に計算できます。
$h$の2乗部分を計算すると、次のようになります。
\begin{align*}
  \phi^c(x) &= \inf_y \left[h(y - x) - \phi(y) \right]\\
            &= \inf_y \left[\frac{1}{2}|y-x|^2 - \phi(y) \right]\\
            &=\frac{1}{2}|x|^2+\inf_y\left[-x\cdot y+\frac{1}{2}|y|^2-\phi(y)\right].
\end{align*}
ここで、$\varphi(y)=\frac{1}{2}|y|^2-\phi(y)$ と設定し、Legendre 変換を $\varphi^*(x)$ で表すと,
\begin{equation}
  \label{eq:legendre}
  \varphi^*(x)=\sup_y x\cdot y-\varphi(y),
\end{equation}
そして、
$$\phi^c(x)=\frac{1}{2}|x|^2-\varphi^*(x).$$
したがって、2次の$c$-transformを計算する代わりにLegendre変換を計算することで$c$-transformが計算できます。

$n$個の点集合に対する一次元のLegendre変換は、$O(n)$の操作で計算することができます[19]。
[19]のアイデアは、Legendre変換の2つの非常に重要な性質、つまり$\varphi^* = \varphi^{***} = (\varphi^{**})^*$であり、
$\varphi^{**}$が$\varphi$のconvex hullであることを利用しています.
したがって、(\ref{eq:legendre})を計算することは、
\begin{equation}
  x - (\varphi^{**})^\prime(y)
\end{equation}
の値を求めることに相当し、ここで符号が正から負に変わる場所が$y(x)$の値です。

$\varphi^{**}$が凸であるため、勾配$(\varphi^{**})^\prime(y)$は増加します。
したがって、(13)の各$x$に対応する符号変化$y(x)$を1回の$(\varphi^{**})^\prime(y)$値のスキャンで見つけることができます。
$n$個の定格の点を持つ1次元グリッドでは、スキャンステップは$O(n)$の操作を要し、
重要なことに、convex hull $\varphi^{**}$も$O(n)$の操作で計算できます[19]。
また、$c$-transformの場合と同様に、多次元レジェンドル変換も各次元で問題を分解することで計算できます。
最後に、正確なアルゴリズムと詳細については、[19]を参照してください。

\subsection{Step sizes}

\ref{sect:the back-and-forth method}章で見たように、任意の関数$\phi$における勾配マッピングのLipschitz定数を決定することは容易ではありません。
したがって、固定したステップサイズの勾配上昇法において最適な$\sigma$を選択する方法は明確ではありません。
そのため、Armijo-Goldstein型のアップデートルール[2]を使用して、アルゴリズム中に$\sigma = \sigma_n$を更新します。\\

各勾配ステップ後に、$J(\phi_{n+1})-J(\phi_n)$ を2乗ノルム$\| \nabla_{\dot{H}^1} J(\phi) \|_{\dot{H}^1}^2$ ($I$および$\psi, \psi$の場合についても同様)に比較します。
パラメータ$0<\beta_1<\beta_2<1, \alpha_1>1, \alpha_2<1$ が与えられた場合、以下を確認します。

$$
- \sigma_n \beta_2 \|\nabla_{\dot{H}^1} J(\phi_n)\|^2 \leq J(\phi_n)-J(\phi_{n+1}) \leq - \sigma_n \beta_1 \|\nabla_{\dot{H}^1} J(\phi_n)\|_{\dot{H}^1}^2.
$$

もし上の不等式が成立しない場合、
我々は$\sigma_n$を減らすために$\sigma_{n+1}=\alpha_2\sigma_n$を取ります。
下の不等式が成立しない場合、$\sigma_n$を増やすために$\sigma_{n+1}=\alpha_1\sigma_n$を取ります。
全ての実験において、$\beta_1 = \frac{1}{4},\beta_2 = \frac{3}{4}, \alpha_1 = \frac{5}{4}, \alpha_2 = \frac{4}{5}$とし、$\sigma = 8 \min(\|\mu\|_{L^\infty}^{-1}, \|\nu\|_{L^\infty}^{-1})$を初期値として選択します。\\

注意すべきは、我々はbacktrackingを行いません。
新しいの値、すなわち$\sigma = \sigma_{n+1}$は単に次の更新ステップで使用されます。
実際、関数における$J$または$I$を評価するには$c$-transformの計算が必要であるため、特定の反復回数でステップサイズを最適化しようとする価値はありません。


\subsection{Experiments}


\section{Details}
\hypertarget{絶対連続1}
$\bold{ルベーグ測度に関して絶対連続}$
確率測度$\mu$が\hyperlink{絶対連続2}{ルベーグ積分に関して絶対連続}であるとき、ルベーグ積分 $=$ リーマン積分になる。すなわち、
$d \mu(x) = \mu(x)dx$であり、

\begin{equation*}
  \int_{\Omega} \phi(T(x)) d \mu(x) = \int_{\Omega}\phi(y) d \nu(y)
  \Leftrightarrow 
  \int_{\Omega} \phi(T(x)) \mu(x) dx = \int_{\Omega}\phi(y) \nu (y)dy.
\end{equation*}\\


\hypertarget{支持されている}
$\bold{支持されている(supported)}$

$\mu$ supported in $\omega$ $\Longleftrightarrow$ 
$
\text{supp} \, \mu = \overline{\{x \in \omega \, | \, \mu(x) \ne 0 \}} 
$


\subsection*{Proof}


\begin{proof}[Proof of Lemma \ref{lem:c-transform}]
  \hypertarget{proof:lem:c-transform}{}

  Proof of $(\mathrm{i})$
  \begin{align*}
            \phi^c(x)    &= \inf_{y\in\Omega} \left(\frac{1}{2 \tau}|x - y|^2 - \phi(y)\right)\\
            \phi^{cc}(y) &= \inf_{x\in\Omega} \left(\frac{1}{2 \tau}|x - y|^2 - \phi^c(x)\right)\\
                      &= \inf_{x\in\Omega} \left(\frac{1}{2 \tau}|x - y|^2 - \inf_{z\in\Omega} \left(\frac{1}{2 \tau}|x - z|^2 - \phi(z)\right)\right)\\
                      &\ge  \inf_{x\in\Omega} \left(\frac{1}{2 \tau}|x - y|^2 - \left(\frac{1}{2 \tau}|x - y|^2 - \phi(y)\right)\right)\\
                      &= \phi(y).
  \end{align*}

次に$\phi^{cc} = \phi$の必要十分条件は$\phi$が$c$-concaveであることを示す。

$(\Leftarrow)$

$\phi: c$-concaveすなわち、$\exists \psi \, s.t. \, \phi = \psi^c$のとき、$\phi^{cc} = \phi$を示す。

$\exists \psi \, s.t. \, \phi = \psi^c$と仮定する。
$\psi := \phi^c$とおくと、
$$
    \phi = \psi^c = (\phi^c)^c= \phi^{cc}
$$

$(\Rightarrow)$

$\phi^{cc} = \phi$のとき、$\phi: c$-concaveすなわち、$\exists \psi \, s.t. \, \phi = \psi^c$を示す。

$\phi^{cc} = \phi$と仮定すると、
$$
    \phi = \phi^{cc} = (\phi^c)^c
$$
よって、$\phi = (\phi^c)^c$となる$\phi^c$が存在する。したがって、$\phi$は$c$-concaveである。


最後に、$\phi^{ccc} = \phi$を示す。
まず、$\phi^{ccc} \geq \phi^c$を示す。
\begin{align*}
    \phi^{ccc}(x)   &= \inf_{w \in \Omega} \left( \frac{1}{2 \tau} |x - w|^2 - \phi^{cc}(w) \right)\\
                    &= \inf_{w \in \Omega} \left( \frac{1}{2 \tau} |x - w|^2 - \inf_{z \in \Omega}\left( \frac{1}{2 \tau} |w - z|^2 - \phi^c(z)\right)\right)\\
                    &= \inf_{w \in \Omega} \left( \frac{1}{2 \tau} |x - w|^2 + \sup_{z \in \Omega}\left( - \frac{1}{2 \tau} |w - z|^2 + \phi^c(z)\right)\right)\\
                    &\geq \inf_{w \in \Omega} \left( \frac{1}{2 \tau} |x - w|^2 + \left( - \frac{1}{2 \tau} |w - x|^2 + \phi^c(x)\right)\right)\\
                    &= \phi^c(x)\\
\end{align*}
次に、$\phi^{ccc} \leq \phi^c$を示す。
\begin{align*}
    \phi^{ccc}(x)   &= \inf_{w \in \Omega} \left( \frac{1}{2 \tau} |x - w|^2 - \phi^{cc}(w) \right)\\
                    &\leq \inf_{w \in \Omega} \left( \frac{1}{2 \tau} |x - w|^2 - \phi(w) \right)\\
                    &= \phi^c(x)\\
\end{align*}
よって、$\phi^{ccc} = \phi^c$が示された。


Proof of $(\mathrm{ii})$.

$T_\phi(x) = x - (\nabla h)^{-1} (\nabla \phi^c(x))$が成り立つことを示します。

$\phi^c(x) = \inf_{y \in \Omega} c(x, y) - \phi(y) =  \inf_{y \in \Omega} h(y - x) - \phi(y)$より、$\phi^c$は$x$についての関数である。
よって$h(y - x) - \phi(y)$最小値は$\frac{\partial }{\partial x} (h(y - x) - \phi(y)) = \nabla (h(y - x) - \phi(y)) = 0$となる点の値である。
なぜなら、$h(y - x) - \phi(y)$は凸関数であるためである。($\phi$が凹関数より、$- \phi$は凸関数のため)
よって、もちろん$\phi^c(x)$も凸関数。よって、

$$
  \nabla (h(y - x) - \phi(y)) = - \nabla h(y - x) 
$$
$T_\phi$の定義から、$y = T_\phi$のとき、$h(y - x) - \phi(y)$が$\inf$をとる。
すなわち、$y = T_\phi$のとき、$\phi^c(x)$となる。
よって、上の式を書き換えると
$$
    \nabla \phi^c(x) = - \nabla h(T_\phi(x) - x) 
$$
よって、
\begin{eqnarray*}
    &\nabla \phi^c(x) = - \nabla h(T_\phi(x) - x) \\
    \iff& T_\phi(x) - x = - (\nabla h)^{-1}(\nabla \phi^c(x))\\
    \iff& T_\phi(x) = x - (\nabla h)^{-1}(\nabla \phi^c(x))
\end{eqnarray*}

\end{proof}



証明します。

まず、$\phi$が$c$-concaveであると仮定します。また、$u$は$\Omega$上の連続関数です。

写像$T_\phi(x) = \underset{y \in \Omega}{\operatorname{argmin}}\left(\frac{1}{2 \tau} |x - y|^2 - \phi(y)\right)$を考えます。

式(1)の最小化問題において、$y = T_\phi(x)$が最小値を取る点です。したがって、$T_\phi(x)$を最小化問題の制約条件に代入すると、
$$\frac{1}{2 \tau} |x - T_\phi(x)|^2 - \phi(T_\phi(x)) \leq \frac{1}{2 \tau} |x - y|^2 - \phi(y)$$
$$\frac{1}{2 \tau} |x - T_\phi(x)|^2 - \phi^c(x) \leq \frac{1}{2 \tau} |x - y|^2 - \phi^c(y)$$
$$\frac{1}{2 \tau} |x - T_\phi(x)|^2 - \phi^c(x) \leq \frac{1}{2 \tau} |x - T_\phi(x)|^2 - \phi^c(T_\phi(x))$$
$$\phi^c(T_\phi(x)) \leq \phi^c(x)$$

したがって、$\phi^c$が最小値を取る座標は$T_\phi(x)$で表されます。

次に、$u$の連続性を考慮して、以下の極限を考えます。
$$\lim_{\epsilon \to 0} \frac{(\phi + \epsilon u)^c(x) - \phi^c(x)}{\epsilon}$$

$\epsilon \to 0$の極限を取ると、$(\phi + \epsilon u)^c(x)$は$\phi^c(x)$に収束します。

したがって、極限の中身を展開すると、
$$\lim_{\epsilon \to 0} \frac{(\phi + \epsilon u)^c(x) - \phi^c(x)}{\epsilon} = \lim_{\epsilon \to 0} \frac{\phi^c(x) - \phi^c(x)}{\epsilon} + \lim_{\epsilon \to 0} \frac{(\epsilon u)^c(x)}{\epsilon}$$

第一項は$0$となります。

第二項を評価するために、$(\epsilon u)^c(x)$の定義を考えます。$(\epsilon u)^c(x)$は$c$-transformの定義により、
$$(\epsilon u)^c(x) = \sup_{y \in \Omega}\left\{c(x, y) - \epsilon u(y)\right\}$$

$c(x, y) = \frac{1}{2 \tau} |x - y|^2$であることから、
$$(\epsilon u)^c(x) = \sup_{y \in \Omega}\left\{\frac{1}{2 \tau} |x - y|^2 - \epsilon u(y)\right\}$$

この最適化問題の最大値を取る点を$y_\epsilon$とします。

$T_\phi(x)$が最小化問題の制約条件を満たすことから、
$$\frac{1}{2 \tau} |x - T_\phi(x)|^2 - \phi(T_\phi(x)) \leq \frac{1}{2 \tau} |x - y|^2 - \phi(y)$$
$$\frac{1}{2 \tau} |x - T_\phi(x)|^2 - \phi^c(x) \leq \frac{1}{2 \tau} |x - y|^2 - \phi^c(y)$$

$y = y_\epsilon$として、この不等式を書き直すと、
$$\frac{1}{2 \tau} |x - T_\phi(x)|^2 - \phi^c(x) \leq \frac{1}{2 \tau} |x - y_\epsilon|^2 - \phi^c(y_\epsilon)$$

右辺の不等式は$\epsilon \to 0$の極限を取ると、$y_\epsilon$が$T_\phi(x)$に収束することから、
$$\frac{1}{2 \tau} |x - T_\phi(x)|^2 - \phi^c(x) \leq \frac{1}{2 \tau} |x - T_\phi(x)|^2 - \phi^c(T_\phi(x))$$
$$\phi^c(T_\phi(x)) \leq \phi^c(x)$$

したがって、$\phi^c(x)$が最小値を取る座標は$T_\phi(x)$です。

$u$が$\Omega$上で連続な関数であることから、$u(y_\epsilon)$は$u(T_\phi(x))$に収束します。

したがって、$\epsilon \to 0$の極限を取ると、
$$\lim_{\epsilon \to 0} \frac{(\phi + \epsilon u)^c(x) - \phi^c(x)}{\epsilon} = - u(T_\phi(x))$$

以上が証明です。


\begin{proof}[Proof of Theorem \ref{thm:ascent property}]
  \hypertarget{proof:thm:ascent property}{}
  まず、与えられた不等式において、$\phi = \phi_{n+1}$および$\hat{\phi} = \phi_n$とすると、不等式は以下のように書き直せる。
  $$
    F(\phi_{n+1}) \geq F(\phi_n) + \delta F_{\phi_n} (\phi_{n+1} - \phi_n) - \frac{1}{2\sigma} \|\phi_{n+1} - \phi_n\|_\mathcal{H}^2
  $$
  勾配降下法の反復によって$\phi_{n+1} = \phi_n + \sigma \nabla_\mathcal{H} F(\phi_n)$であることを考慮すると、$\phi_{n+1} - \phi_n = \sigma \nabla_\mathcal{H} F(\phi_n)$です。
  これを上記の不等式に代入すると、
  {\color{teal}
  \begin{align*}
    F(\phi_{n+1})   &\geq F(\phi_n) + \delta F_{\phi_n} (\sigma \nabla_\mathcal{H} F(\phi_n)) - \frac{1}{2\sigma} \|\sigma \nabla_\mathcal{H} F(\phi_n)\|_\mathcal{H}^2 \\
                    &= F(\phi_n) + \sigma \delta F_{\phi_n} (\nabla_\mathcal{H} F(\phi_n)) - \frac{\sigma^2}{2\sigma} \| \nabla_\mathcal{H} F(\phi_n)\|_\mathcal{H}^2\\
                    &= F(\phi_n) + \sigma \left\langle \nabla_\mathcal{H} F(\phi_n), \nabla_\mathcal{H} F(\phi_n)\right\rangle - \frac{\sigma}{2} \| \nabla_\mathcal{H} F(\phi_n)\|_\mathcal{H}^2 \qquad (\because \text{Def. } \ref{dfn:inner product})\\
                    &= F(\phi_n)  + \sigma \| \nabla_\mathcal{H} F(\phi_n)\|_\mathcal{H}^2 - \frac{\sigma}{2} \| \nabla_\mathcal{H} F(\phi_n)\|_\mathcal{H}^2\\
  \end{align*} 
  }
  よって、
  $$
    F(\phi_{n+1}) - F(\phi_n) \geq \frac{\sigma}{2} \| \nabla_\mathcal{H} F(\phi_n)\|_\mathcal{H}^2
  $$

  以上より、勾配降下法の反復によって$F(\phi_{n+1}) - F(\phi_n)$は$\|\nabla_\mathcal{H} F(\phi_n)\|_\mathcal{H}^2$以上になることが示されました。
  {\color{teal}
  次に、$F$が唯一の極大点$\phi_*$を持ち、列$\{\phi_n\}_{n=0}^\infty$が$\mathcal{H}$の有界な部分集合であると仮定します。
  このとき、列$\{\phi_n\}$から弱収束点$\tilde{\phi}$を取り出すことができます。
  また、弱上半連続性により、$F(\tilde{\phi}) \geq \lim_{n \to \infty} F(\phi_n)$が成り立ちます
  ($F$が下に凸かつ実数値であることを仮定しているため、弱上半連続性は自動的に満たされます)。
  値$F(\phi_n)$は$F(\phi_*)$で上に有界であるため、
  $\lim_{n \to \infty} \|\nabla_\mathcal{H} F(\phi_n)\|_\mathcal{H} = 0$
  です。
  したがって、以下の不等式の連鎖を導くことができます。
  $$
    F(\phi_*) \leq F(\phi_n) + \langle \nabla_\mathcal{H} F(\phi_n), \phi_* - \phi_n \rangle \leq F(\phi_n) + 2R \|\nabla_H\mathcal{H} F(\phi_n)\|_\mathcal{H},
  $$
  ここで、最初の不等式は$F$が凹であることから、
  2つ目の不等式は
  $R = \max(\|\phi_*\|_\mathcal{H}, \sup_n \|\phi_n\|_\mathcal{H})$
  を選ぶことにより得られます。
  右辺の極限を取ると、$F(\tilde{\phi}) \geq F(\phi_*)$となります。
  したがって、$\tilde{\phi} = \phi_*$となります(最大化点は唯一であるため)。
  }
\end{proof}

\begin{proof}[Proof of Lemma \ref{lem:variation}]
  \hypertarget{proof:lem:variation}{}
  \begin{align*}
    J(\phi) &= \int_\omega \phi \, d\nu + \int_\omega \phi^c \, d\mu \\
    J(\phi + \epsilon u) &= \int_\omega (\phi + \epsilon u) \, d\nu + \int_\omega (\phi + \epsilon u)^c \, d\mu 
  \end{align*}
  より、
  \begin{align*}
    J(\phi + \epsilon u) - J(\phi)  &= \int_\omega (\phi + \epsilon u)(x) \, d\nu(x) + \int_\omega (\phi + \epsilon u)^c(x) \, d\mu(x) - \left( \int_\omega \phi(x) \, d\nu(x) + \int_\omega \phi^c(x) \, d\mu(x) \right)\\
                                    &= \int_\omega (\phi(x) + \epsilon u(x)) - \phi(x) \, d\nu(x) + \int_\omega (\phi + \epsilon u)^c(x) - \phi^c(x) \, d\mu(x) \\
                                    &= \int_\omega \epsilon u(x) \, d\nu(x) + \int_\omega (\phi + \epsilon u)^c(x) - \phi^c(x) \, d\mu(x) 
  \end{align*}
  ここで、Lemmma \ref{lem:c-transform}の$(\mathrm{iii})$を用いると、
  \begin{align*}
    \lim_{\epsilon \to 0} \frac{J(\phi + \epsilon u) - J(\phi)}{\epsilon} &= \lim_{\epsilon \to 0} \int_\omega \frac{ \epsilon u(x)}{\epsilon} \, d\nu(x) + \lim_{\epsilon \to 0} \int_\omega \frac{(\phi + \epsilon u)^c(x) - \phi^c(x)}{\epsilon} \, d\mu(x) \\
                                                        \delta J_\phi(u)  &= \int_\omega u(x) \, d\nu(x) + \int_\omega \lim_{\epsilon \to 0} \frac{(\phi + \epsilon u)^c(x) - \phi^c(x)}{\epsilon} \, d\mu(x) \\
                                                                          &= \int_\omega u(x) \, d\nu(x) - \int_\omega u(T_\phi(x)) \, d\mu(x) \\
                                                                          &= \int_\omega u(x) \, d\nu(x) - \int_\omega u(y) \, d\, T_{\phi \#} \mu(y) & (\because (\ref{def:pushforward_int}))\\
                                                                          &= \int_\omega u(x) \nu(x) \, dx - \int_\omega u(x) T_{\phi \#} \mu(x) \, dx\\
                                                                          &= \int_\omega u(x) \left(\nu(x) - T_{\phi \#} \mu(x) \right)\, dx\\
                                                                          &= \int_\omega u \left(\nu - T_{\phi \#} \mu \right)
  \end{align*}
  よって、
  \[
    \delta J_\phi = \int_\omega \nu - T_{\phi \#} \mu
  \]
\end{proof}

\begin{proof}[Proof of (\ref{eq:variation int ineq})]
  \hypertarget{proof:eq:variation int ineq}{}
  不等式(\ref{eq:variation ascent})を示すためには、以下の式変形が必要です。
  \begin{align*}
    \int_\Omega (\phi - \hat{\phi})(T_{\hat{\phi} \#} \mu - T_{\phi \#} \mu)  &= \int_\Omega T_{\phi \#}(\phi - \hat{\phi}) d\mu - \int_\Omega T_{\hat{\phi} \#}(\phi - \hat{\phi}) d\mu \\
                                                                              &= \langle T_{\phi \#}(\phi - \hat{\phi}), \phi - \hat{\phi} \rangle_{\mathcal{H}} - \langle T_{\hat{\phi} \#}(\phi - \hat{\phi}), \phi - \hat{\phi} \rangle_{\mathcal{H}} \\
                                                                              &= \langle T_{\phi \#}(\phi - \hat{\phi}) - T_{\hat{\phi} \#}(\phi - \hat{\phi}), \phi - \hat{\phi} \rangle_{\mathcal{H}} \\
                                                                              &= \langle T_{\hat{\phi} \#}^*(T_{\hat{\phi} \#} \mu - T_{\phi \#} \mu), \phi - \hat{\phi} \rangle_{\mathcal{H}} \\
                                                                              &= \langle (T_{\hat{\phi} \#} - I)^*(T_{\hat{\phi} \#} \mu - T_{\phi \#} \mu), \phi - \hat{\phi} \rangle_{\mathcal{H}} \\
                                                                              &= \langle (T_{\hat{\phi} \#} - I)(T_{\hat{\phi} \#} \mu - T_{\phi \#} \mu), \phi - \hat{\phi} \rangle_{\mathcal{H}} \\
                                                                              &\geq - \|(T_{\hat{\phi} \#} - I)(T_{\hat{\phi} \#} \mu - T_{\phi \#} \mu)\|_{\mathcal{H}} \|\phi - \hat{\phi}\|_{\mathcal{H}} \\
                                                                              &= - \|T_{\hat{\phi} \#} \mu - T_{\phi \#} \mu\|_{\mathcal{H}} \|\phi - \hat{\phi}\|_{\mathcal{H}} \\
                                                                              &= - \frac{1}{\sigma} \|\phi - \hat{\phi}\|_{\mathcal{H}}^2,
  \end{align*}
  ここで、2行目から3行目への変形で、ヒルベルト空間 $\mathcal{H}$ 上で内積をとりました。また、3行目から4行目への変形では、$T_{\hat{\phi} \#} - I$ が $\mathcal{H}$ 上の自己共役な線形作用素であることを用いました。最後の行では、前節での定義に従って、$\sigma = \|T_{\hat{\phi} \#} \mu - T_{\phi \#} \mu\|_{\mathcal{H}}$ とおきました。
  したがって、不等式(\ref{eq:variation ascent})は以下のように得られます。
  $$
  \int_\Omega (\phi - \hat{\phi})(T_{\hat{\phi} \#} \mu - T_{\phi \#} \mu) \geq - \frac{1}{\sigma} \|\phi - \hat{\phi}\|_{\mathcal{H}}^2 \geq - \frac{1}{2\sigma} \|\phi - \hat{\phi}\|_{\mathcal{H}}^2,
  $$
  最後の不等号は $\sigma \leq \frac{1}{2}$ で成立するためです。従って、$\sigma \leq \frac{1}{2}$ の場合には、不等式(8)が成立することが示されました。
\end{proof}

\begin{proof}[Proof of Lemma \ref{lem:stability estimates}]
  \hypertarget{proof:lem:stability estimates}{}

  $\bold{Step 1.}$
  証明は、次の等式を立証することから始めます。
  \begin{equation}
    \label{eq:F(phi2|phi1)}
    F(\phi_2 | \phi_1) = \int_{\Omega} \left[\frac{1}{2} \left| T_{\phi_1}(x) - T_{\phi_2}(x) \right|^2 - \phi_2\left( T_{\phi_1}(x)) |T_{\phi_2}(x)\right) \right] \mu(x)dx.
  \end{equation}
  このために、まずFact 3の単純な変形を利用すると、$F$の導関数は
  \begin{equation}
    \label{eq:delta F}
    \delta F_\phi = T_{\phi \#} \mu
  \end{equation}  
  であることがわかります。
  \begin{align*}
    F_\phi &= - \int_\Omega \phi^c \, d \mu\\
    F_{\phi + \epsilon u} &= - \int_\Omega (\phi + \epsilon u)^c \, d \mu
  \end{align*}
  \begin{align*}
    \lim_{\epsilon \to 0} \frac{F_{\phi + \epsilon u} - F_\phi}{\epsilon} &= - \lim_{\epsilon \to 0} \int_\Omega \frac{(\phi + \epsilon u)^c -  \phi^c}{\epsilon} \, d \mu\\
                                                          \delta F_\phi(u) &= \int_\omega u(T_\phi(x)) \, d\mu(x) \\
                                                                          &= \int_\omega u(y) \, d\, T_{\phi \#} \mu(y) & (\because (\ref{def:pushforward_int}))\\
                                                                          &= \int_\omega u(x) T_{\phi \#} \mu(x) \, dx\\
                                                                          &= \int_\omega u(x) T_{\phi \#} \mu(x)\, dx\\
                                                                          &= \int_\omega u T_{\phi \#} \mu
  \end{align*}
  したがって、$F$に関連するBregman divergenceは以下のように書くことができます。
  \[
    F(\phi_2 | \phi_1) = \int_{\omega} \left[-\phi_2^c(x) + \phi_1^c(x) - \phi_2(T_{\phi_1}(x)) + \phi_1(T_{\phi_1}(x))\right] \mu(x)dx.
  \]
  上記の式は、次の等式を利用して簡略化できます。($\phi^c(x) = \inf_y \left[h(y - x) - \phi(y) \right], h(y - x) = \frac{1}{2}|y-x|^2$).
  \[
    -\phi^c(x) = \phi(T_{\phi}(x)) - \frac{1}{2} \left|T_{\phi}(x) - x\right|^2,
  \]
  これはc-transformと$T_{\phi}$の定義から直ちに導かれます。(Lemma \ref{lem:c-transform} $(\mathrm{ii})$ )したがって、
  \[
    F(\phi_2 | \phi_1) = \int_{\omega} \left[\frac{1}{2} |T_{\phi_1}(x) - x|^2 -\frac{1}{2} |T_{\phi_2}(x) - x|^2 + \phi_2(T_{\phi_2}(x)) - \phi_2 (T_{\phi_1}(x))\right] \mu(x)dx.
  \]
  さらに進めるために、次の等式を利用します。
  \begin{equation}
    \label{eq:nabla phi}
    \nabla \phi(T_\phi(x)) = T_\phi(x) - x,
  \end{equation}
  これもc-transformの定義から直ちに導かれます。
  $F(\phi_2 | \phi_1)$の式に
  \[
    -(T_{\phi_2}(x) - x) \cdot (T_{\phi_1}(x) - T_{\phi_2}(x)) + \nabla \phi_2(T_{\phi_2}(x)) \cdot (T_{\phi_1}(x) - T_{\phi_2}(x))
  \]
  (これは常に0です)を加えることで、所望の等式(\ref{eq:F(phi2|phi1)})を得ることができます。

  $\bold{Step 2.}$

  不等式 
  \begin{equation}
    \label{eq:upper bound of F}
    F(\phi_2|\phi_1) \leq \frac{\lambda^{-1}}{2} \int_\omega |\nabla \phi_2(y) - \nabla \phi_1(y)|^2 \rho_1(y) \, dy,
  \end{equation}
  を証明するために、関数 
  \[
    f_2(x) = \frac{1}{2}|x|^2 - \phi_2(x)
  \] 
  を導入します。
  ただし、\(\rho_1 = T_{\phi_1\#}\mu\) と定義します。
  ここで、\(\phi_2\) に対する仮定により、\(f_2\) は \(\lambda\)-強凸関数です。
  よって、式 (\ref{eq:F(phi2|phi1)}) を以下のように書き換えることができます:
  \[
    F(\phi_2|\phi_1) = \int_\Omega f_2(T_{\phi_1}(x) | T_{\phi_2}(x))\,\mu(x) \,dx.
  \]
  Lemma \ref{lem:convex conjugate}$(\mathrm{iii})$を用いると、任意の \(t_1, t_2 \in \omega\)  に対して、
  \[
    f_2(t_1|t_2) \leq \frac{\lambda^{-1}}{2}|\nabla f_2(t_1) - \nabla f_2(t_2)|^2
  \]
  を満たす。
  再び、恒等式 (\ref{eq:nabla phi}) を用いると、次のようになります:
  \[
    \nabla f_2(T_{\phi_1}(x)) = x + \nabla\phi_1(T_{\phi_1}(x)) - \nabla\phi_2(T_{\phi_1}(x)),
  \]
  一方で、\(\nabla f_2(T_{\phi_2}(x)) = x\) です。
  このことから、以下の上界が導かれる。
  \[
    f_2(T_{\phi_1}(x) | T_{\phi_2}(x)) \leq \frac{\lambda^{-1}}{2} \left|\nabla\phi_1(T_{\phi_1}(x)) - \nabla\phi_2(T_{\phi_1}(x))\right|^2.
  \]
  これは、測度の変換後の形において、求める不等式(\ref{eq:upper bound of F})を導くことを意味します。

  $\bold{Step 3.}$

  次のステップでは、$\rho_1$の$L^{\infty}$ノルムを推定します:
  \begin{equation}
    \label{eq:L infty norm}
    \| \rho_1 \| \leq \lambda^{-d} \|\mu\|.
  \end{equation}
  ここで、$\phi_1$の強凸性から$T_{\phi_1}$は単射であることが言えます。
  $\rho_1$は$T_{\phi_1 \#} \mu$の密度であるため、変数変換の公式
  \[
    |\det D T_{\phi_1}(x)| \rho_1(T_{\phi_1}(x)) = \mu(x)
  \]
  が成立します。
  この公式は単射かつ微分可能な写像に対して有効です。
  次に、$DT_{\phi_1}$が対称かつ正定値行列であり、その固有値が$\lambda$以下であることを示します。
  これには以下のように設定します:
  \[
    f_1(x) = \frac{1}{2} |x|^2 - \phi_1(x)
  \]
  ここで、$\phi_1$に対する仮定から \(D^2 f_1(x) \leq \lambda^{-1}\) となります。
  恒等式 (\ref{eq:nabla phi}) を用いると、 \(x = \nabla f_1(T_{\phi_1}(x))\) と書けます。
  この式は逆に \(\nabla f_1^*(x) = T_{\phi_1}(x)\) となります。
  $f_1$のヘッセ行列の固有値が$\lambda^{-1}$以下であるため、その共役凸関数は$\lambda$-強凸です。
  したがって、
  \[
    \det D T_{\phi_1}(x) \geq \lambda^d,
  \]
  が任意の \(x \in \omega\) に対して成り立ちます。これにより、不等式 (\ref{eq:L infty norm}) が直接導かれます。
  これを式 (\ref{eq:upper bound of F}) と組み合わせることで、不等式 (\ref{eq:F_upper bound}) の証明が完了します。

  $\bold{Step 4.}$

  上限の不等式(\ref{eq:F_upper bound})が証明されたので、直接下限の不等式 (\ref{eq:F_lower bound}) が導かれます。
  実際、不等式(\ref{eq:F_upper bound})を次のように書き直します:
  \[
    F(\phi + h|\phi_1) \leq \frac{C}{2}\|h\|^2_{\dot{H}^1}.
  \]
  ただし、$h = \phi_2 - \phi_1, C = \|\mu\| \lambda^{-(d+1)}$.
  両辺に共役作用素を作用させると、不等号の向きが逆転します。
  すると次のような不等式が得られます:
  \[
    F(\phi_1 + \cdot \,|\phi_1)^*(u) \geq \frac{C^{-1}}{2} \|u\|^2_{\dot{H}^{-1}}, 
  \]

  ここで $ \frac{1}{2}\| \cdot \|^2_{\dot{H}^1}$ ノルムと $ \frac{1}{2}\| \cdot \|^2_{\dot{H}^{-1}} $ ノルムは互いに共役であることがわかります。
  次に、Lemma \ref{lem:convex conjugate}の性質 $(\mathrm{iv})$ であるBregmanダイバージェンスの性質を$F$関数に適用し、次の不等式を得ます:
  \[
    F^*(\phi_1 + u | \phi_1) \geq \frac{C^{-1}}{2} \|u\|^2_{\dot{H}^{-1}}, 
  \]
  ただし、$\rho_1 = \delta F_{\phi_1} = T_{\phi_1 \#} \mu$です。これは、\( u = \rho_2 - \rho_1 \) のときに正確に求めたい不等式です。

\end{proof}

\begin{proof}[Proof of Proposition \ref{prop:勾配上昇保持}]
  \hypertarget{proof:prop:勾配上昇保持}{}
  $\bold{上昇性質.}$
  連続する2つの反復 $\phi_n$ と $\phi_{n+1}$ を考えます。
  Lemma \ref{lem:variation}によれば、$J$の導関数は $\delta J_\phi = \nu - T_{\phi \#} \mu$ の形を取る。
  したがって、以下のように表されます。
  \[
  J(\phi_{n+1}) - J(\phi_n) = J(\phi_{n+1}|\phi_n) + \int_\Omega (\phi_{n+1} - \phi_n)(\nu - T_{\phi_n \#} \mu).
  \]
  ここで、Bregmanダイバージェンス $J(\cdot|\cdot)$ を導入しました(Definition \ref{dfn:Bregman divergence}を参照)。
  式(\ref{eq:F(phi)})によって定義される $F$ について、先に述べたように(\ref{eq:J-F}) $J(\phi_{n+1}|\phi_n) = -F(\phi_{n+1}|\phi_n)$ です。
  さらに、勾配ステップ $\phi_{n+1} - \phi_n = \sigma(-\Delta)^{-1}(\nu - T_{\phi_n \#} \mu)$ は次のように表されます:
  \[
    \int_\omega (\phi_{n + 1} - \phi_n)(\nu - T_{\phi_n \#} \mu) = \sigma^{-1}\|\phi_{n + 1} - \phi_n\|_{\dot{H}^1}^2.
  \]
  これらの2つの式を組み合わせると、次のようになります。
  \[
    J(\phi_n) - J(\phi_{n+1}) = -F(\phi_{n+1}|\phi_n) + \sigma^{-1}\|\phi_{n+1} - \phi_n\|_{\dot{H}^1}^2,
  \]
  そして、$\sigma = \frac{\lambda^{d+1}}{\|\mu\|}$ とLemma \ref{lem:stability estimates}との組み合わせにより、以下の上昇性質が得られます:
  \[
  J(\phi_n) - J(\phi_{n+1}) \geq \frac{1}{\|\mu\|}\lambda^{-(d+1)}\|\phi_{n+1} - \phi_n\|_{\dot{H}^1}^2.
  \]

  $\bold{Decrease\,\, in\,\,} \bm{H^{-1}}\,\, \bold{norm}$

  簡単な計算により、二次の項を展開し再配列することで次の式が得られます。
  \[
    \|\rho_{n+1} - \nu\|_{H^{-1}}^2 - \|\rho_n - \nu\|_{H^{-1}}^2 = \|\rho_{n+1} - \rho_n\|_{H^{-1}}^2 - 2 \int_\omega (-\Delta)^{-1}(\nu - \rho_n)(\rho_{n+1} - \rho_n).
  \]
  ここで、反復 $\phi_{n+1}$ は $\phi_{n+1} = \phi_n + \sigma(-\Delta)^{-1}(\nu - \rho_n)$ によって定義されるため、次のようになります。
  \[
    \|\rho_{n+1} - \nu\|_{H^{-1}}^2 - \|\rho_n - \nu\|_{H^{-1}}^2 = \|\rho_{n+1} - \rho_n\|_{H^{-1}}^2 - 2\sigma^{-1}\int_\omega (\phi_{n+1} - \phi_n)(\rho_{n+1} - \rho_n)
  \]
  次に、式(\ref{eq:delta F})で説明されているように、式(\ref{eq:F(phi)})で定義される汎関数 $F$ の導関数は $\delta F_\phi = \rho = T_{\phi \#} \mu$ となります。
  {\color{teal}
  また、$F$ は凸汎関数であるため、この関係は以下のように逆になります。
  \[
    \phi = \delta F^*_\rho,
  \]
  }
  ここで、$F^*$ は $F$ の凸共役を表し、$F^*_{\rho} = F^*(\rho) = \sup_\phi \int \phi \rho - F(\phi)$ で定義されます。
  \begin{align*}
    F^*(\rho + \epsilon u) - F^*(\rho)  &= \sup_\phi \int_\Omega \phi (\rho + \epsilon u) \, dx - \int_\Omega \phi \rho \, dx  - F(\phi) + F(\phi) \\
                                        &= \sup_\phi \int_\Omega \phi \rho \, dx  + \int_\Omega \phi \epsilon u \, dx - \int_\Omega \phi \rho \, dx\\
                                        &= \sup_\phi \int_\Omega \phi \epsilon u \, dx
  \end{align*}
  \begin{align*}
    \lim_{\epsilon \to 0} \frac{F^*(\rho + \epsilon u) - F^*(\rho)}{\epsilon}  &= \sup_\phi \lim_{\epsilon \to 0} \int_\Omega \frac{ \phi \epsilon u }{\epsilon} \, dx\\
                                                          \delta F^*_{\rho}(u) &= \sup_\phi \int_\Omega u \phi \, dx \\
                                                          \delta F^*_{\rho} &= \phi
  \end{align*}


  したがって、以下のように書くことができます。
  \begin{align*}
      \|\rho_{n+1} - \nu\|_{H^{-1}}^2 - \|\rho_n - \nu\|_{H^{-1}}^2 &= \|\rho_{n+1} - \rho_n\|_{H^{-1}}^2 - 2\sigma^{-1} \left( \delta F^*_{\rho_{n + 1}} - \delta F^*_{\rho_n} \right) (\rho_{n+1} - \rho_n)\\
                                                                    &= \|\rho_{n+1} - \rho_n\|_{H^{-1}}^2 - 2\sigma^{-1} \left(F^*(\rho_{n+1}|\rho_n) + F^*(\rho_n|\rho_{n+1})\right)\\
                                                                    &= - 2 \sigma^{-1}\left[ F^*(\rho_{n + 1} | \rho_n) + F^*(\rho_n | \rho_{n + 1}) - \frac{\sigma}{2} \|\rho_{n+1} - \rho_n \|_{H^{-1}}^2 \right].
  \end{align*}
  Lemma \ref{lem:stability estimates} により、$F^*(\rho_{n+1}|\rho_n) + F^*(\rho_n|\rho_{n+1}) \geq \sigma\|\rho_{n+1} - \rho_n\|^2_{H^{-1}}$ であるため、証明を結ぶのに十分です。
\end{proof}

\begin{proof}[Proof of Proposition \ref{prop:不等式連鎖}]
  \hypertarget{proof:prop:不等式連鎖}{}
  ある反復 $\phi_n$ を固定します。
  すると、$J(\phi_n) = D(\phi_n, \phi_n^c)$ です。
  Proposition \ref{prop:勾配上昇保持} により、
  $$
    J(\phi_{n + \frac{1}{2}}) \geq J(\phi_n)
  $$
  が成り立ちます。
  なぜなら、$\phi_n \to \phi_{n + \frac{1}{2}}$ のステップは $J$ の勾配上昇ステップであり、
  また $J(\phi_{n + \frac{1}{2}}) = D(\phi_{n + \frac{1}{2}}, (\phi_{n + \frac{1}{2}})^c) = D(\phi_{n + \frac{1}{2}}, \psi_{n + \frac{1}{2}})$ です。
  次に、$c$-変換を行うと常に $D$ の値が増加するか、あるいは横ばいになることが容易にわかります。
  したがって、
  $$
    D\left((\psi_{n + \frac{1}{2}})^c, \psi_{n + \frac{1}{2}}\right)\geq D(\phi_{n + \frac{1}{2}}, \psi_{n + \frac{1}{2}}).
  $$
  が成り立ちます。
  したがって、$I(\psi_{n + \frac{1}{2}}) \geq J(\phi_{n + \frac{1}{2}})$ を示しました。
  残りの不等式も同様の方法で扱うことができます。



\end{proof}

\end{document}

